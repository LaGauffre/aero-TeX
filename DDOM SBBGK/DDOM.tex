\subsection{Dynamic Discrete Ordinate Method}
	
\begin{frame}
	\frametitle{DDOM}
	To circumbent the problems professor C.T. Hsu, et al. \cite{Hsu201239} noted that the exponential in (\ref{eq:newdistf}) can be made independent of $\vec{u}(\vec{x},t)$ and $T(\vec{x},t)$. by using the transformation,
	\begin{equation}
	\vec{C}^{*}  = \frac{(\vec{c}-\vec{u})}{\sqrt{\frac{2 k_B T}{m}}} = \frac{(\vec{c}-\vec{u})}{a}
	\label{eq:transformation}
	\end{equation}
	Then the Semiclassical equilibrium distribution becomes,
	\begin{equation}
	f^{*eq} = f^{eq}(\vec{x},\vec{C}^{*},t) = \frac{1}{(\frac{1}{z})\exp{(\vec{C}^{*}}^2)+\theta}
	\label{eq:newdistf}
	\end{equation}
	Where $\vec{C}^{*}$ will be defined as the abscissas of our GH quadrature.
	
\end{frame}

\begin{frame}
	\frametitle{DDOM}
	Define variable $a = \sqrt{\frac{2 k_B T}{m}}$ and (\ref{eq:transformation}) can be re-written as in ref. \cite{Hsu201239},
	\begin{equation}
	\vec{c_\sigma}(\vec{x},t)  = a(\vec{x},t)\vec{C_\sigma}^{*}+\vec{u}(\vec{x},t) 
	\end{equation}
	To substitute $f(x,\vec{c},t)$ with $f^*(x,\vec{C^*},t)$ in BE, implies to substitute the total variation of the last term respect to $x$ and $t$ as if $C^*$ is a dependent variable of $c$, in oder to preserver the physical meaning of the original BE.
\end{frame}

\begin{frame}
	\frametitle{DDOM}
	Taking the total diferenciations of $f^*$ respecto to $x$ and $t$ we get,
			
	\begin{align*}
		\frac{df^*}{dt} &= \frac{\partial{f^*}}{\partial{t}} + 
				\frac{\partial{f^*}}{\partial{C^*}} \frac{\partial{C^*}}{\partial{a}} \frac{\partial{a}}{\partial{t}} + 
				\frac{\partial{f^*}}{\partial{C^*}} \frac{\partial{C^*}}{\partial{u}} \frac{\partial{u}}{\partial{t}}
		\\		
		\\ \frac{df^*}{dx} &= \frac{\partial{f^*}}{\partial{t}} + 
				\frac{\partial{f^*}}{\partial{C^*}} \frac{\partial{C^*}}{\partial{a}} \frac{\partial{a}}{\partial{x}} + 
				\frac{\partial{f^*}}{\partial{C^*}} \frac{\partial{C^*}}{\partial{u}} \frac{\partial{u}}{\partial{x}}
	\end{align*}
\end{frame}

\begin{frame}
	\frametitle{DDOM}
	Substituting know partial factors,
			
	\begin{align*}
		\frac{df^*}{dt} &= \frac{\partial{f^*}}{\partial{t}} + 
				\frac{\partial{f^*}}{\partial{C^*}} \frac{-C^*}{a} \frac{\partial{a}}{\partial{t}} + 
				\frac{\partial{f^*}}{\partial{C^*}} \frac{ -1 }{a} \frac{\partial{u}}{\partial{t}}
		\\		
		\\ \frac{df^*}{dx} &= \frac{\partial{f^*}}{\partial{t}} + 
				\frac{\partial{f^*}}{\partial{C^*}} \frac{-C^*}{a} \frac{\partial{a}}{\partial{x}} + 
				\frac{\partial{f^*}}{\partial{C^*}} \frac{ -1 }{a} \frac{\partial{u}}{\partial{x}}
	\end{align*}
\end{frame}

\begin{frame}
Then Boltzmann - BGK equation using DDOM becomes,
	\begin{equation}
		\begin{split}
			\frac{\partial{f^*_\sigma}}{\partial{t}} &+ (a\vec{C_\sigma}^{*}+\vec{u})\bullet\frac{\partial{f^*_\sigma}}{\partial{\vec{x}}} 
			+\vec{F}\bullet\frac{\partial{f^*_\sigma}}{\partial{\vec{C_\sigma}^{*}}}
			\\ &-\frac{1}{a}\frac{\partial f^*}{\partial C^*}C^*\left(\frac{\partial a}{\partial t} +c \frac{\partial a}{\partial x} \right )
			-\frac{1}{a}\frac{\partial f^*}{\partial C^*}\left(\frac{\partial u}{\partial t} +c \frac{\partial u}{\partial x} \right ) 
			\\ &= -\frac{1}{\tau}(f^*_\sigma-f^{*eq}_\sigma)
		\end{split}
	\end{equation}
	Note that $f^*_\sigma = f(\vec{x},\vec{C_\sigma}^{*},t)$ and $f^{*eq}_\sigma=f^eq(\vec{x},\vec{C_\sigma}^{*},t)$
\end{frame}

\begin{frame}
	\frametitle{DDOM}
	Again, we wish to compute the first four moments fo the the distribution function,
	\begin{eqnarray}
	\int f^* d^3 c  &=& \int f^{*eq}d^3c = \rho \\
	\int \vec{c} f^* d^3 c  &=& \int \vec{c} f^{*eq}d^3c = \rho \vec{u} \nonumber \\
	\int \frac{\vec{c}^2}{2} f^* d^3 c  &=& \int \frac{1}{2}\vec{c}^2 f^{*eq}d^3c = \rho E \nonumber \\
	\int \frac{(\vec{c}-\vec{u})^2}{2} f^* d^3 c  &=& \int \frac{1}{2}(\vec{c}-\vec{u})^2 f^{*eq}d^3c = \rho e \nonumber
	\end{eqnarray}
	but observe that a change of integration variable is necessary to keep their physical meaning intact.
\end{frame}

\begin{frame}
	\frametitle{DDOM}
	The term $(\frac{\partial C^*}{\partial c} )^3$ is introduced to perform this change of variables inside the moment integrals
	\begin{eqnarray}
		\int f^* \left(\frac{\partial C^*}{\partial c} \right )^3 d^3 c  &=& \rho \\
		\int \vec{c} f^* \left(\frac{\partial C^*}{\partial c} \right )^3 d^3 c &=& \rho \vec{u} \nonumber \\
		\int \frac{\vec{c}^2}{2} f^* \left(\frac{\partial C^*}{\partial c} \right )^3 d^3 c  &=& \rho E \nonumber \\
		\int \frac{(\vec{c}-\vec{u})^2}{2} f^* \left(\frac{\partial C^*}{\partial c} \right )^3 d^3 c &=& \rho e \nonumber
	\end{eqnarray}
	We identify the term $\left(\frac{\partial C^*}{\partial c} \right )$ as the Jacobian, J. 
\end{frame}

\begin{frame}
	\frametitle{DDOM}
	The moment integrals become,
	\begin{eqnarray}
		\sum_\sigma J^3 W_\sigma \exp(c_\sigma^2) f^*(\vec{x},\vec{c_\sigma},t) &=&  \rho, \\
		\sum_\sigma J^3 \vec{c_\sigma} W_\sigma \exp(c_\sigma^2) f^*(\vec{x},\vec{c_\sigma},t)&=& \rho \vec{u}, \nonumber \\
		\sum_\sigma J^3 \frac{\vec{c_\sigma}^2}{2} W_\sigma \exp(c_\sigma^2) f^*(\vec{x},\vec{c_\sigma},t)&=& \rho E, \nonumber \\
		\sum_\sigma J^3 \frac{(\vec{c_\sigma}-\vec{u})^2}{2} W_\sigma \exp(c_\sigma^2) f^*(\vec{x},\vec{c_\sigma},t) &=& \rho e  \nonumber
	\end{eqnarray}
	Where J is the Jacobian. Here $J = \sqrt{\frac{2k_B T(\vec{x},t)}{m}}$ .
\end{frame}

\subsection{Numerical Results with DDOM}
\begin{frame}
	\frametitle{Numerical Tests}
	\begin{itemize}
	\item Based on our DDOM description of the moments and our initial observations with conventional DOM, we developed a 'snipet' code in Matlab to evaluate and compare both methods in 1d and 2d cases.
	\item Our finding and conclusions will follow our preliminary resutls.
	\end{itemize}	
\end{frame}

\begin{frame}
	\frametitle{1d Equations}
	Neglecting any external force, the Boltzmann-BKG equation in 1d becomes,
	\begin{equation}
	\frac{\partial f(x,c,t)}{\partial t} + c \frac{\partial f(x,c,t)}{\partial x} = -\frac{f - f^{eq}}{\tau}
	\end{equation}
	and its companion semiclassical equilibrium distribution function becomes,
	\begin{equation}
	f^{eq} (x,c,t) = \frac{1}{(1/z) exp( \frac{ m \left | c - u \right |^2}{\sqrt{2 k_B T}} )+\theta}
	\end{equation}
\end{frame}

\begin{frame}
	\frametitle{1d Equations}
	substituting $f(x,c,t)$ by $f^*(x,C^*,t)$, where and evaluating the total derivates of $f*$ we get our Dynamics Boltzmann-BGK equation,
	
	\begin{equation}
	\frac{\partial f^*}{\partial t} + c_x \frac{\partial f*}{\partial x} 
		- \frac{1}{a} \frac{\partial f^*}{\partial C^*} C^* \frac{Da}{Dt}
		- \frac{1}{a} \frac{\partial f^*}{\partial C^*} \frac{Du}{Dt}
		= -\frac{f - f^{eq}}{\tau}
	\end{equation}
	
	Where operator $D/Dt = d/dt + c d/dx$ is identified as the material derivate. Observer that, 
	\begin{equation}
		c(x,t) = a(x,t) C^* + u(x,t)
	\end{equation}
	
\end{frame}

\begin{frame}
	\frametitle{Normalization of 1d Equations}
	But we first need to normalize our governing equations. Following Muljadi and Yang \cite{Yang2013}.
	\begin{align*}
	&V_\infty = \sqrt{\frac{2k_BT_\infty}{m}}, &t_\infty = \frac{L}{V_\infty}
	\end{align*}
	Here L is identified as the characterisitc length
	\begin{align*}
	&(\hat t, \hat \tau) = \frac{(t,\tau)}{T_\infty}, &(\hat u_x,\hat c_x,) = \frac{(u_x,c_x)}{V_\infty}, \\
	&\hat T = \frac{T}{T_\infty}, \;\; \hat x = \frac{x}{L}, &\hat \rho = \rho / \left(\frac{m^2V_\infty^2}{h^2} \right ), \\
	&\hat {\rho u_x} = \rho u_x / \left(\frac{m^2V_\infty^3}{h^2} \right ), &\hat E = E / \left(\frac{m^3V_\infty^4}{h^2} \right ), &\;\; \hat f = f.
	\end{align*}
\end{frame}

\begin{frame}
	\frametitle{Normalization of 1d Equations}
	Neglecting any external force, the normalized Boltzmann-BKG equation in 1d becomes,
	\begin{equation}
	\frac{\partial \hat {f^*}}{\partial \hat t} + \hat c_x \frac{\partial \hat {f*}}{\partial \hat x} 
		- \frac{1}{\hat a} \frac{\partial \hat {f^*}}{\partial \hat {C^*}} \hat {C^*} \frac{D \hat a}{D \hat t}
		- \frac{1}{\hat a} \frac{\partial \hat {f^*}}{\partial \hat {C^*}} \frac{D \hat u}{D \hat t}
		= -\frac{\hat f - \hat {f^{eq}}}{\hat \tau}
	\end{equation}
	The normalized one-dimensional semiclassical equilibrium distribution function becomes,
	\begin{equation}
	\hat f^{eq} (\hat x,\hat c,\hat t) = \frac{1}{(1/z) exp(\left | \hat c - \hat u \right |^2 / \hat T)+\theta}
\end{equation}
	
\end{frame}

\begin{frame}
	\frametitle{Normalization of 1d Equations}
	our normalized definition of $\vec{C}^*$ 
	\begin{align*}
	&\hat C^* = \frac{(\hat c-\hat u)}{\sqrt{\hat T}} & \hat J = \left(\frac{\partial \hat {C^*}}{\partial \hat c} \right ) = \sqrt(\hat T) 
	\end{align*}
	The Semiclassical equilibrium distribution becomes,
	\begin{equation}
	\hat f^{eq} (\hat x,\hat c,\hat t) = \frac{1}{(1/ \hat z) exp(\hat {C^*})+\hat \theta}
	\label{eq:normadistf1d}
	\end{equation}
	We will drop the hat notation in the following slides while working with normalized equations.
\end{frame}

\begin{frame}
	\frametitle{Apply DOM to D-BE}
	Applying Discrete Ordinate Method, we render the following set of dynamic Boltzmann-BGK equations,
	\begin{equation}
	\frac{\partial {f_\sigma^*}}{\partial t} + c_\sigma \frac{\partial f_\sigma^*}{\partial x} 
		- \frac{1}{a} \frac{\partial f_\sigma^*}{\partial C_\sigma^*} C_\sigma^* \frac{Da}{Dt}
		- \frac{1}{a} \frac{\partial f_\sigma^*}{\partial C_\sigma^*} \frac{Du}{Dt}
		= -\frac{f_\sigma - f_\sigma^{eq}}{\tau}
	\end{equation}
	Note that $c_\sigma(x,t) = a(x,t) C_\sigma^* + u(x,t)$. We also must render their companion semi-classical equilibrium distribution functions,
	\begin{equation}
	f_\sigma^{eq} (x,c_\sigma,t) = \frac{1}{(1/z) exp(C_\sigma^*)^2 + \theta}
	\end{equation}
	
\end{frame}

\begin{frame}
	\frametitle{DDOM}
	The moment integrals become,
	\begin{eqnarray}
		\sum_\sigma J W_\sigma \exp(c_\sigma^2) f^*(x,c_\sigma,t) &=&  \rho, \\
		\sum_\sigma J c_\sigma W_\sigma \exp(c_\sigma^2) f^*(x,c_\sigma,t) &=& \rho \vec{u}, \nonumber \\
		\sum_\sigma J \frac{c_\sigma^2}{2} W_\sigma \exp(c_\sigma^2) f^*(x,c_\sigma,t) &=& \rho E, \nonumber \\
		\sum_\sigma J \frac{(c_\sigma-u)^2}{2} W_\sigma \exp(c_\sigma^2) f^*(x,c_\sigma,t) &=& \rho e  \nonumber
	\end{eqnarray}
	Where J is the Jacobian. Here $J = \sqrt{T(x,t)}$ .
\end{frame}

\begin{frame}
	\frametitle{Matlab 'Snipet'}
	\begin{figure}
	\centering
	\includegraphics[height=5.5cm]{snipet_DDOM_vs_DOM}%
	\caption{Matlab 'snipet' of the 1d case}
	\end{figure}	
\end{frame}

\begin{frame}
	\frametitle{Tests}
	We propose 3 cases to be evaluated:
	\begin{itemize}
	\item case 1: Sod's left semiclassical IC
	\item case 2: Sod's right semiclassical IC
	\item case 3: large displacement-low temperature semiclassical IC
	\end{itemize}
\end{frame}

\begin{frame}
	\frametitle{Tests}
	\centering
	\begin{tabular}{r | c}
	z = 0.0394; u = 0.00; t = 3.2 &  z = 0.0394; u = 0.00; t = 3.2 \\ \hline
	z = 0.3000; u = 1.00; t = 0.1 & \\
	\end{tabular}
\end{frame}

\begin{frame}
	\frametitle{Tests}
	\begin{tabular}{r c}
	\centering
	\includegraphics[scale= 0.11]{MB_DDOMvsDOM1} & \includegraphics[scale= 0.11]{MB_DDOMvsDOM2} \\ 
	\includegraphics[scale= 0.11]{MB_DDOMvsDOM3} & {3 selected cases Graphically}\\
	\end{tabular}
\end{frame}

\begin{frame}
	\frametitle{Number of points for decided acuaracy}
	The required number of GH quadrature points to achieve an specific order of accuaracy, is tabulated as follows
	\begin{center}
	\begin{tabular}{| l | l | l | l |}
	%\centering
	\hline
	Order        & MB & FD & BE \\ \hline
	$O(h^2)$ &  -   &    3   &   3   \\ \hline
	$O(h^4)$ &  -   &  5-7  & 9-12 \\ \hline
	$O(h^8)$ &  3  & 19-20 & 18-20 \\ \hline
	\end{tabular}
	\end{center}
	%\caption{GH quadrature points for desired $O(h^n)$ presicion}
\end{frame}

\begin{frame}
	\frametitle{Normalization of 2d Equations}
	Again, normalize our governing equations as in \cite{Yang2013}.
	\begin{equation}
	V_\infty = \sqrt{\frac{2k_BT_\infty}{m}}, \;\;\;\;  t_\infty = \frac{L}{V_\infty}
	\end{equation}
	Here L is identified as the characterisitc length
	\begin{align*}
	&(\hat t, \hat \tau) = \frac{(t,\tau)}{T_\infty}, &(\hat u_x,\hat u_y,\hat v_x,\hat v_y) = \frac{(u_x,u_y,v_x,v_Y)}{V_\infty}, \\
	&(\hat x,\hat y) = \frac{(x,y)}{L}, &\hat T = \frac{T}{T_\infty}, \;\; \hat \rho =\rho / \left(\frac{m^2V_\infty^2}{h^2} \right ), \\
	&\hat{\rho u_x} = \rho u_x / \left(\frac{m^2V_\infty^3}{h^2} \right ), &\hat E = E / \left(\frac{m^3V_\infty^4}{h^2} \right ), \;\; \hat f = f.
	\end{align*}
\end{frame}

\begin{frame}
	\frametitle{Normalization of 2d Equations}
	Neglecting any external force, The normalized Boltzmann-BKG equation in 2d becomes,
	\begin{equation}
	\frac{\partial \hat f(\hat x,\hat v_x,t)}{\partial \hat t} + \hat v_x \frac{\partial \hat f(\hat x,\hat v_x,t)}{\partial \hat x} +  \hat v_x \frac{\partial \hat f(\hat x,\hat v_x,t)}{\partial \hat y} =  -\frac{\hat f - \hat f^{eq}}{\hat \tau}
	\end{equation}
	The normalized two-dimensional semiclassical equilibrium distribution function becomes,
	\begin{equation}
	\hat f^{eq} (\hat x,\hat v_x,\hat t) = \frac{1}{(1/z) \exp{((\left|\hat v_x - \hat u_x \right|^2 + \left|\hat v_y - \hat u_y \right|^2) / \hat T)}+\theta} 
	\end{equation}
	We will drop the hat notation in the following slides while working with normalized equations.
\end{frame}




