\subsection{Dynamic Discrete Ordinate Method}
	
\begin{frame}
	\frametitle{DDOM}
	To circumbent the problems professor C.T. Hsu, et al. \cite{ISI:000303761300021,Hsu201239} noted that the exponential in the equilibrium distribution function (\ref{eq:classical_feq}) can be made independent of $\vec{u}(\vec{x},t)$ and $T(\vec{x},t)$. by using the transformation,
	\begin{equation}
	\vec{C}^{*}  = \frac{(\vec{c}-\vec{u})}{\sqrt{\frac{2 k_B T}{m}}} = \frac{(\vec{c}-\vec{u})}{a}
	\label{eq:transformation}
	\end{equation}
Where $\vec{C}^{*}$ will be defined as the abscissas of our GH quadrature.
\end{frame}

\begin{frame}
	\frametitle{Graphically speacking}
	
		\begin{figure}
			\centering
				\includegraphics[width=0.90\textwidth]{MB_DDOMvsDOM2}
			\caption{Quadrature point distribution on thermaly normalized distribution with galilean transformation vs. traditional method}
			\label{fig:MB_DDOMvsDOM2}
		\end{figure}
	
\end{frame}

\begin{frame}
	\frametitle{Graphically speacking}
		
		\begin{figure}
			\centering
				\includegraphics[width=0.90\textwidth]{MB_DDOMvsDOM1}
			\caption{Quadrature point distribution on thermaly normalized distribution with galilean transformation vs. traditional method}
			\label{fig:MB_DDOMvsDOM1}
		\end{figure}
		
\end{frame}

\begin{frame}
	\frametitle{Graphically speacking}

		\begin{figure}
			\centering
				\includegraphics[width=0.90\textwidth]{MB_DDOMvsDOM3}
			\caption{Quadrature point distribution on thermaly normalized distribution with galilean transformation vs. traditional method}
			\label{fig:MB_DDOMvsDOM3}
		\end{figure}
		
\end{frame}


\begin{frame}
	\frametitle{DDOM}
	Define variable $a = \sqrt{\frac{2 k_B T}{m}}$ and (\ref{eq:transformation}) can be re-written as in ref. \cite{Hsu201239},
	\begin{equation}
	\vec{c_\sigma}(\vec{x},t)  = a(\vec{x},t)\vec{C_\sigma}^{*}+\vec{u}(\vec{x},t) 
	\end{equation}
	To substitute $f(\vec{x},\vec{c},t)$ with $f^*(x,\vec{C^*},t)$ in BE, implies to substitute the total variation of the last term respect to $\vec{x}$ and $t$ as if $\vec{C}^*$ is a dependent variable of $\vec{c}$, in oder to preserver the physical meaning of the original BE.
\end{frame}

\begin{frame}
	\frametitle{DDOM}
	e.i.,
	\begin{equation}
	f(\vec{x},\vec{c},t) = f^* (\vec{x},\vec{C}^*(\vec{x},t),t) = f^*
	\end{equation}
	
	Taking the total differenciations of $f^*$ respecto to $x$ and $t$ we get,
			
	\begin{align*}
		\frac{df^*}{dt} &= \frac{\partial{f^*}}{\partial{t}} + 
				\frac{\partial{f^*}}{\partial{C^*}} \frac{\partial{C^*}}{\partial{a}} \frac{\partial{a}}{\partial{t}} + 
				\frac{\partial{f^*}}{\partial{C^*}} \frac{\partial{C^*}}{\partial{u}} \frac{\partial{u}}{\partial{t}}
		\\		
		\\ \frac{df^*}{dx} &= \frac{\partial{f^*}}{\partial{t}} + 
				\frac{\partial{f^*}}{\partial{C^*}} \frac{\partial{C^*}}{\partial{a}} \frac{\partial{a}}{\partial{x}} + 
				\frac{\partial{f^*}}{\partial{C^*}} \frac{\partial{C^*}}{\partial{u}} \frac{\partial{u}}{\partial{x}}
	\end{align*}
\end{frame}

\begin{frame}
	\frametitle{DDOM}
	Substituting know partial factors,
			
	\begin{align*}
		\frac{df^*}{dt} &= \frac{\partial{f^*}}{\partial{t}} + 
				\frac{\partial{f^*}}{\partial{C^*}} \frac{-C^*}{a} \frac{\partial{a}}{\partial{t}} + 
				\frac{\partial{f^*}}{\partial{C^*}} \frac{ -1 }{a} \frac{\partial{u}}{\partial{t}}
		\\		
		\\ \frac{df^*}{dx} &= \frac{\partial{f^*}}{\partial{t}} + 
				\frac{\partial{f^*}}{\partial{C^*}} \frac{-C^*}{a} \frac{\partial{a}}{\partial{x}} + 
				\frac{\partial{f^*}}{\partial{C^*}} \frac{ -1 }{a} \frac{\partial{u}}{\partial{x}}
	\end{align*}
\end{frame}

\begin{frame}
Then Boltzmann - BGK equation using DDOM becomes,
	\begin{equation}
		\begin{split}
			\frac{\partial{f^*_\sigma}}{\partial{t}} &+ (a\vec{C_\sigma}^{*}+\vec{u})\bullet\frac{\partial{f^*_\sigma}}{\partial{\vec{x}}} 
			+\vec{F}\bullet\frac{\partial{f^*_\sigma}}{\partial{\vec{C_\sigma}^{*}}}
			\\ &-\frac{1}{a}\frac{\partial f^*}{\partial \vec{C_\sigma}^*}\vec{C_\sigma}^*\left(\frac{\partial a}{\partial t} + \vec{c} \frac{\partial a}{\partial \vec{x}} \right )
			-\frac{1}{a}\frac{\partial f^*}{\partial \vec{C_\sigma}^*}\left(\frac{\partial u}{\partial t} + \vec{c} \frac{\partial u}{\partial \vec{x}} \right ) 
			\\ &= -\frac{1}{\tau}(f^*_\sigma-f^{*eq}_\sigma)
		\end{split}
	\end{equation}
	Note that $f^*_\sigma = f(\vec{x},\vec{C_\sigma}^{*},t)$ and $f^{*eq}_\sigma=f^eq(\vec{x},\vec{C_\sigma}^{*},t)$
\end{frame}

\begin{frame}
	\frametitle{DDOM}
	Again, we wish to compute the first four moments fo the the distribution function,
	\begin{eqnarray}
	\int f^* d^3 c  &=& \rho \\
	\int \vec{c} f^* d^3 c  &=& \rho \vec{u} \nonumber \\
	\int \frac{\vec{c}^2}{2} f^* d^3 c  &=& \rho E \nonumber \\
	\int \frac{(\vec{c}-\vec{u})^2}{2} f^* d^3 c  &=& \rho e \nonumber
	\end{eqnarray}
	but observe that a change of integration variable is necessary to keep their physical meaning intact.
\end{frame}

\begin{frame}
	\frametitle{DDOM}
	The term $(\frac{\partial C^*}{\partial c} )^3$ is introduced to perform this change of variables inside the moment integrals
	\begin{eqnarray}
		\int f^* \left(\frac{\partial C^*}{\partial c} \right )^3 d^3 c  &=& \rho \\
		\int \vec{c} f^* \left(\frac{\partial C^*}{\partial c} \right )^3 d^3 c &=& \rho \vec{u} \nonumber \\
		\int \frac{\vec{c}^2}{2} f^* \left(\frac{\partial C^*}{\partial c} \right )^3 d^3 c  &=& \rho E \nonumber \\
		\int \frac{(\vec{c}-\vec{u})^2}{2} f^* \left(\frac{\partial C^*}{\partial c} \right )^3 d^3 c &=& \rho e \nonumber
	\end{eqnarray}
	We identify the term $\left(\frac{\partial C^*}{\partial c} \right )$ as the Jacobian, J. 
\end{frame}

\begin{frame}
	\frametitle{DDOM}
	The moment integrals become,
	\begin{eqnarray}
		\sum_\sigma J^3 W_\sigma \exp(c_\sigma^2) f^*(\vec{x},\vec{c_\sigma},t) &=&  \rho, \\
		\sum_\sigma J^3 \vec{c_\sigma} W_\sigma \exp(c_\sigma^2) f^*(\vec{x},\vec{c_\sigma},t)&=& \rho \vec{u}, \nonumber \\
		\sum_\sigma J^3 \frac{\vec{c_\sigma}^2}{2} W_\sigma \exp(c_\sigma^2) f^*(\vec{x},\vec{c_\sigma},t)&=& \rho E, \nonumber \\
		\sum_\sigma J^3 \frac{(\vec{c_\sigma}-\vec{u})^2}{2} W_\sigma \exp(c_\sigma^2) f^*(\vec{x},\vec{c_\sigma},t) &=& \rho e  \nonumber
	\end{eqnarray}
	Where J is the Jacobian. Here $J = \sqrt{\frac{2k_B T(\vec{x},t)}{m}}$ .
\end{frame}


\begin{frame}
	\frametitle{Using DDOM with Semi-classical Formulation}
	\begin{itemize}
	\item as we see the procedure is analogous to the DDOM developed for the classical distribution.
	\item We apply and numerically evalute a case in 1d to test this results.
	\item Our finding and conclusions will follow after.
	\end{itemize}	
\end{frame}

\begin{frame}
	\frametitle{1d Equations}
	Neglecting any external force, the Boltzmann-BKG equation in 1d becomes,
	\begin{equation}
	\frac{\partial f(x,c,t)}{\partial t} + c \frac{\partial f(x,c,t)}{\partial x} = -\frac{f - f^{eq}}{\tau}
	\end{equation}
	and its companion semiclassical equilibrium distribution function becomes,
	\begin{equation}
	f^{eq} (x,c,t) = \frac{1}{(1/z) exp( \frac{ m \left | c - u \right |^2}{\sqrt{2 k_B T}} )+\theta}
	\end{equation}
\end{frame}

\begin{frame}
	\frametitle{Thermal Normalization \& Galilean trasformation of the 1d Equations}
	Our thermal normalized definition of $\vec{C}^*$ goes a follows,
	\begin{align*}
	& C^* = \frac{(c-u)}{\sqrt{\frac{2k_BT_\infty}{m}}} & J = \left(\frac{\partial C^*}{\partial c} \right ) = \sqrt{\frac{2k_BT_\infty}{m}} 
	\end{align*}
	The Semiclassical equilibrium distribution becomes,
	\begin{equation}
	f^{eq} ( x, c, t) = \frac{1}{(1/  z) exp({C^*})+ \theta}
	%\hat f^{eq} (\hat x,\hat c,\hat t) = \frac{1}{(1/ \hat z) exp(\hat {C^*})+\hat \theta}
	\label{eq:normadistf1d}
	\end{equation}
	%We will drop the hat notation in the following slides while working with normalized equations.
\end{frame}

\begin{frame}
	\frametitle{BE with effects of galilean transformation}
	substituting $f(x,c,t)$ by $f^*(x,C^*,t)$, and evaluating the total derivates of $f^*$ we get our dynamics Boltzmann-BGK equation,
	
	\begin{equation}
	\frac{\partial f^*}{\partial t} + c_x \frac{\partial f*}{\partial x} 
		- \frac{1}{a} \frac{\partial f^*}{\partial C^*} C^* \frac{Da}{Dt}
		- \frac{1}{a} \frac{\partial f^*}{\partial C^*} \frac{Du}{Dt}
		= -\frac{f - f^{eq}}{\tau}
	\label{eq:sbbgk1d}
	\end{equation}
	
	Where operator $D/Dt = d/dt + c d/dx$ is identified as the material derivate. Observer that, 
	\begin{equation}
		c(x,t) = a(x,t) C^* + u(x,t)
	\end{equation}	
\end{frame}

\begin{frame}
	\frametitle{Normalization of 1d Equations}
	But before coding it, we first need to normalize our governing equations. We follow Muljadi and Yang \cite{Yang2013} procedure,
	
	\begin{align*}
	V_\infty &= \sqrt{\frac{2k_BT_\infty}{m}}, & t_\infty &= \frac{L}{V_\infty}
	\end{align*}
	
	Here $L$ $(L = V_\infty t_\infty)$ is identified as the characterisitc length.
\end{frame}

\begin{frame}
	Multiplying to equation \ref{eq:sbbgk1d} and it moments by $(\frac{V_\infty}{V_\infty})(\frac{t_\infty}{t_\infty})$, the following dimensionles relations follows,
	\begin{align*}
	(\hat t, \hat \tau) &= \frac{(t,\tau)}{t_\infty}, & (\hat u_x,\hat c_x,) = \frac{(u_x,c_x)}{V_\infty}, \\
	\hat T &= \frac{T}{T_\infty}, & \hat E = E / \left(\frac{m^3V_\infty^4}{h^2} \right ), \\
	\hat f &= f, &\hat {\rho u_x} = \rho u_x / \left(\frac{m^2V_\infty^3}{h^2} \right ), \\
	\hat x &= \frac{x}{L}, &\hat \rho = \rho / \left(\frac{m^2V_\infty^2}{h^2} \right ).
	\end{align*}
\end{frame}

\begin{frame}
	\frametitle{Normalization of 1d Equations}
	Neglecting any external force, the normalized Boltzmann-BKG equation in 1d becomes,
	\begin{equation}
	\frac{\partial \hat {f^*}}{\partial \hat t} + \hat c_x \frac{\partial \hat {f*}}{\partial \hat x} 
		- \frac{1}{\hat a} \frac{\partial \hat {f^*}}{\partial \hat {C^*}} \hat {C^*} \frac{D \hat a}{D \hat t}
		- \frac{1}{\hat a} \frac{\partial \hat {f^*}}{\partial \hat {C^*}} \frac{D \hat u}{D \hat t}
		= -\frac{\hat f - \hat {f^{eq}}}{\hat \tau}
	\end{equation}
	The normalized one-dimensional semiclassical equilibrium distribution function becomes,
	\begin{equation}
	\hat f^{eq} (\hat x,\hat c,\hat t) = \frac{1}{(1/z) exp(\left | \hat c - \hat u \right |^2 / \hat T)+\theta}
\end{equation}
\end{frame}

\begin{frame}
	\frametitle{Thermal Normalization \& Galilean trasformation}
	Normalized definition of $\vec{C}^*$ are,
	\begin{align*}
	&\hat C^* = \frac{(\hat c-\hat u)}{\sqrt{\hat T}} & \hat J = \left(\frac{\partial \hat {C^*}}{\partial \hat c} \right ) = \sqrt{\hat T} 
	\end{align*}
	The Semiclassical equilibrium distribution becomes,
	\begin{equation}
		\hat f^{eq} (\hat x,\hat c,\hat t) = \frac{1}{(1/ \hat z) exp(\hat {C^*})+\hat \theta}
	\label{eq:normadistf1d}
	\end{equation}
	We will drop the hat notation in the following slides while working with normalized equations.
\end{frame}

\begin{frame}
	\frametitle{Apply DOM to modified BE}
	Applying Discrete Ordinate Method, we render the following set of dynamic Boltzmann-BGK equations,
	\begin{equation}
	\frac{\partial {f_\sigma^*}}{\partial t} + c_\sigma \frac{\partial f_\sigma^*}{\partial x} 
		- \frac{1}{a} \frac{\partial f_\sigma^*}{\partial C_\sigma^*} C_\sigma^* \frac{Da}{Dt}
		- \frac{1}{a} \frac{\partial f_\sigma^*}{\partial C_\sigma^*} \frac{Du}{Dt}
		= -\frac{f_\sigma - f_\sigma^{eq}}{\tau}
	\end{equation}
	Note that $c_\sigma(x,t) = a(x,t) C_\sigma^* + u(x,t)$. We also must render their companion semi-classical equilibrium distribution functions,
	\begin{equation}
	f_\sigma^{eq} (x,c_\sigma,t) = \frac{1}{(1/z) exp(C_\sigma^*)^2 + \theta}
	\end{equation}
\end{frame}

\begin{frame}
	\frametitle{Basic Numerical Method}
		\begin{equation}
		\begin{split}
			\frac{f^{*,n+1}_{\sigma,i} - f^{*,n}_{\sigma,i}}{\Delta t} & + 
			(a_i\vec{C_\sigma}^{*}+\vec{u}_i)^{+} \bullet \frac{f^{*,n}_{\sigma,i-1} - f^{*,n}_{\sigma,i}}{\Delta x} 
			\\ & +(a_i\vec{C_\sigma}^{*}+\vec{u}_i)^{-} \bullet \frac{f^{*,n}_{\sigma,i} - f^{*,n}_{\sigma,i+1}}{\Delta x}
			\\ & -\frac{1}{a}\frac{\partial f^*}{\partial \vec{C_\sigma}^*}\vec{C_\sigma}^*\left(\frac{D a}{D t} \right )
			\\ & -\frac{1}{a}\frac{\partial f^*}{\partial \vec{C_\sigma}^*}\left(\frac{D u}{D t} \right)
			\\ & = -\frac{1}{\tau}(f^n_{\sigma,i} - f^{eq,n}_{\sigma,i})
		\end{split}
	\end{equation}
\end{frame}
	
\begin{frame}
	\frametitle{Basic Numerical Method}
		\begin{align*}
			\left(\frac{D a}{D t} \right ) &= \frac{a^{n+1}_{i} - a^{n+1}_{i}}{\Delta t} \\
				& + c^{+} \bullet \frac{a^{n}_{i} - a^{n}_{i-1}}{\Delta x}
				 + c^{-} \bullet \frac{a^{n}_{i+1} - a^{n}_{i}}{\Delta x} \\
			\left(\frac{D a}{D t} \right ) &= \frac{u^{n+1}_{i} - u^{n+1}_{i}}{\Delta t} \\
				& + c^{+} \bullet \frac{u^{n}_{i} - u^{n}_{i-1}}{\Delta x}
				 + c^{-} \bullet \frac{u^{n}_{i+1} - u^{n}_{i}}{\Delta x}
		\end{align*}
\end{frame}

\begin{frame}
	\frametitle{DDOM}
	The moment integrals become,
	\begin{eqnarray}
		\sum_\sigma J W_\sigma \exp(c_\sigma^2) f^*(x,c_\sigma,t) &=&  \rho, \\
		\sum_\sigma J c_\sigma W_\sigma \exp(c_\sigma^2) f^*(x,c_\sigma,t) &=& \rho \vec{u}, \nonumber \\
		\sum_\sigma J \frac{c_\sigma^2}{2} W_\sigma \exp(c_\sigma^2) f^*(x,c_\sigma,t) &=& \rho E, \nonumber \\
		\sum_\sigma J \frac{(c_\sigma-u)^2}{2} W_\sigma \exp(c_\sigma^2) f^*(x,c_\sigma,t) &=& \rho e  \nonumber
	\end{eqnarray}
	Where J is the Jacobian. Here $J = \sqrt{T(x,t)}$ .
\end{frame}

\subsection{Numerical Results with DDOM}

\begin{frame}
	\frametitle{Numerical Results with DDOM}
		
		\begin{figure}
			\centering
				\includegraphics[width=0.90\textwidth]{2RE_upwind_DDOM1}
			\caption{Right Expansion case. Using MB distribution, Upwind method for all space differentials with 100 points in physical space and DDOM with 5 velocity points in velocity space}
			\label{fig:2RE_upwind_DDOM1}
		\end{figure}
		
\end{frame}

\begin{frame}
	\frametitle{Numerical Results with DDOM}
		
		\begin{figure}
			\centering
				\includegraphics[width=0.90\textwidth]{2RE_upwind_DDOM}
			\caption{Right Expansion case. Using MB distribution, Upwind method for all space differentials with 200 points in physical space and DDOM with 5 velocity points in velocity space}
			\label{fig:MB_RE}
		\end{figure}
		
\end{frame}

\begin{frame}
	\frametitle{Numerical Results with DDOM}
		
		\begin{figure}
			\centering
				\includegraphics[width=0.90\textwidth]{2RE_WENO3_DDOM}
			\caption{Right Expansion case. Using MB distribution, WENO3 method for all space differentials with 100 points in physical space and DDOM with 5 velocity points in velocity space}
			\label{fig:MB_RE}
		\end{figure}
		
\end{frame}