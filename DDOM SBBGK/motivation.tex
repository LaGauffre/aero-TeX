\section{Introduction}
	\subsection{Motivation}
	
\begin{frame}
	\frametitle{Using Semi-classical Boltzmann BGK}
	Let us test the following semi-classical input of SB-BGK equation:
	\[
		(Z,U,T) = 
		\begin{cases}
		(0.2821,0.75,4.0), 	&	0.0 \geq x \geq 0.5 \\
		(0.0394,0,3.2),			& 0.5 > x \geq 1.0
		\end{cases}
	\]
	And it's corresponds classical IC's, i.e. using a Maxwell-Boltzmann Distribution, to:
	\[
		(\rho,U,P) = 
		\begin{cases}
		(1.0,0.75,1.0), &	0.0 \geq x \geq 0.5 \\
		(0.125,0,0.1),	& 0.5 > x \geq 1.0
		\end{cases}
	\]
	Which we recognize to Classical Sod's Shocktube problem's ICs.
\end{frame}

\begin{frame}
	Now, if we use a Fermi-Diract Distribution,
	\[
		(\rho,U,P) = 
		\begin{cases}
		(0.8375,0.75,0.9132), &	0.0 \geq x \geq 0.5 \\
		(0.1216,0,0.0986),		& 0.5 > x \geq 1.0
		\end{cases}
	\]
	or a Bose-Einstein Distribution,
	\[
		(\rho,U,P) = 
		\begin{cases}
		(1.2605,0.75,1.1186), &	0.0 \geq x \geq 0.5 \\
		(0.1286,0,0.1014),		& 0.5 > x \geq 1.0
		\end{cases}
	\]
Now, let us plot the choosen semi-classical IC using each statitic's equilibrium distribution function. Let us do this over a \textcolor{red}{space domain with 100 points} by \textcolor{blue}{60 discrete velocity points}. We observe,
\end{frame}
	
\begin{frame}
	\frametitle{Semiclassical Distribution Functions in 1d}
	Using Maxwell-Boltzmann (MB) Distribution:
	\begin{figure}[p1]
	\centering
	\includegraphics[height=5cm]{MB_IC}%
	\caption{Sod's Shocktube IC problem in a classical gas}
	\end{figure}
\end{frame}

\begin{frame}
	\frametitle{Semiclassical Distribution Functions in 1d}
	Using Fermi-Dirac (FD) Distribution:
	\begin{figure}[p2]
	\centering
	\includegraphics[height=5cm]{FD_IC}%
	\caption{Initial condition for FD Quantum gas}
	\end{figure}
\end{frame}

\begin{frame}
	\frametitle{Semiclassical Distribution Functions in 1d}
	Using Bose-Einstein (BE) Distribution:
	\begin{figure}[p3]
	\centering
	\includegraphics[height=5cm]{BE_IC}%
	\caption{Initial condition for BE Quantum gas}
	\end{figure}
\end{frame}

\begin{frame}
	\frametitle{Sod's Euler Problem using a MB distribution}
	\begin{figure}[p4]
	\centering
	\includegraphics[height=5.5cm]{MB_sod's_10ms}%
	\caption{Maxwell-Boltzmann Distribution}
	\end{figure}
\end{frame}

\begin{frame}
	\frametitle{Sod's solution using FD distribution}
	\begin{figure}[p5]
	\centering
	\includegraphics[height=5.5cm]{FD_sod's_10ms}%
	\caption{Fermi-Dirac Distribution}
	\end{figure}
\end{frame}

\begin{frame}
	\frametitle{Sod's solution using BE distribution}
	\begin{figure}[p6]
	\centering
	\includegraphics[height=5.5cm]{BE_sod's_10ms}%
	\caption{Bose-Einstein Distribution}
	\end{figure}
\end{frame}

\begin{frame}
	\frametitle{The Evolution of distribution functions in 1d}
	\begin{figure}[p7]
	\centering
	\includegraphics[height=5cm]{MB_evolution}%
	\caption{Maxwell-Boltzmann Distribution}
	\end{figure}
\end{frame}

\begin{frame}
	\frametitle{The Evolution of distribution functions in 1d}
	\begin{figure}[p8]
	\centering
	\includegraphics[height=5cm]{FD_evolution}%
	\caption{Fermi-Diract Distribution}
	\end{figure}
\end{frame}

\begin{frame}
	\frametitle{The Evolution of distribution functions in 1d}
	\begin{figure}[p9]
	\centering
	\includegraphics[height=5cm]{BE_evolution}%
	\caption{Bose-Einstein Distribution}
	\end{figure}
\end{frame}

\begin{frame}
	\frametitle{Evaluating numerically the semi-classical $f^{eq}$}
	Numerical Observations
	\begin{figure}[small_z]
	\centering
	\includegraphics[height=5cm]{Quad_err_small_z}%
	\caption{Semi-classical function evaluated with small fugacity values}
	\end{figure}
\end{frame}

\begin{frame}
	\frametitle{Evaluating numerically the semi-classical $f^{eq}$}
	Numerical Observations
	\begin{figure}[medium_z]
	\centering
	\includegraphics[height=5cm]{Quad_err_medium_z}%
	\caption{Semi-classical function evaluated with fugacity value of 0.5}
	\end{figure}
\end{frame}

\begin{frame}
	\frametitle{Evaluating numerically the semi-classical $f^{eq}$}
	Numerical Observations
	\begin{figure}[large_z]
	\centering
	\includegraphics[height=5cm]{Quad_err_large_z}%
	\caption{Semi-classical function evaluated with large fugacity values}
	\end{figure}
\end{frame}

\begin{frame}
	\frametitle{Evaluating numerically the semi-classical $f^{eq}$}
	Numerical Observations
	\begin{figure}[disp_a_1]
	\centering
	\includegraphics[height=5cm]{Quad_err_disp_1c}%
	\caption{Semi-classical function evaluated with small Z and displaced one unit from ref.}
	\end{figure}
\end{frame}

\begin{frame}
	\frametitle{Evaluating numerically the semi-classical $f^{eq}$}
	Numerical Observations
	\begin{figure}[disp_a_2]
	\centering
	\includegraphics[height=5cm]{Quad_err_disp_2c}%
	\caption{Semi-classical function evaluated with small Z and displaced two units from ref.}
	\end{figure}
\end{frame}

\begin{frame}
	\frametitle{Conclusions}
	Given the above observations we can conclude
	\begin{itemize}
		\item The accuracy of global results depends on the accuracy of the quadrature used in DOM.
		\item The more discrete velocity points are used, the more stable the evolution of the problem will be. However this doesn't garanties the that the problem will be stable for all time evolution.
		\item The more discrete velocity points are used the more expensive our system becomes.
		\item Therefore a major improvement of the method lies in making the Discrete Ordinate Method (DOM) 'smarter'.
	\end{itemize}
	
\end{frame}