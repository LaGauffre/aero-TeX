\section{Implemting Nodal DG-FEM}
\subsection{The Method}

\begin{frame}
  \frametitle{Nodal DG-FEM}
  \begin{block}{Note}
     We follow Professors J.S. Hesthaven \& T. Warburton implementation \cite{Hesthaven:2010:NDG:1952159} in the following presentation.
  \end{block}
  Let us start with the one-dimensional scalar conservation law for the solution $u(x,t)$,
  \begin{equation}
    \frac{\partial u}{\partial t} + \frac{\partial f}{\partial x} = g, \;\;\; x \in \Omega
  \end{equation}	
  subject to an appropiate set of initial conditions and boundary conditions on the boundary, $\partial \Omega$. Here $f(u)$ is the flux, and $g(x,t)$ is some prescribed forcing function.
\end{frame}

\begin{frame}
  \frametitle{Nodal DG-FEM}
  re-arranging,
  \begin{equation*}
   \frac{\partial u}{\partial t} + \frac{\partial f}{\partial x} - g = 0, \;\; x \in \Omega
  \end{equation*}
  using approximate solution,
  \begin{equation}
   u(x,t) \approx u^k_h(x,t) = \oplus^K_{k=1} u^k_h(x^k,t)
  \end{equation}
  Residual function,
  \begin{equation*}
   R_h(x,t) = \frac{\partial u_h}{\partial t} + \frac{\partial f_h}{\partial x} - g(x,t), \;\; x \in [x^{k-1},x^{k+1}]
  \end{equation*}
\end{frame}

\begin{frame}
  \frametitle{Nodal DG-FEM}
  Residual function for every interval in our domain,
  \begin{equation*}
   R_h(x,t) = \frac{\partial u^k_h}{\partial t} + \frac{\partial f^k_h}{\partial x} - g(x,t), \;\; x \in I^k
  \end{equation*}
  unfinished equation,
  \begin{equation}
  \int_{I^k} \frac{\partial u^k_h}{\partial t} l^k_j -  
  \end{equation}
\end{frame}

\begin{frame}
	\begin{algorithm}[H]
	 \SetAlgoLined
	 \KwData{this text}
	 \KwResult{how to write algorithm with \LaTeX2e }
	 initialization\;
	 \While{not at end of this document}{
		read current\;
		\eIf{understand}{
		 go to next section\;
		 current section becomes this one\;
		 }{
		 go back to the beginning of current section\;
		}
	 }
	 \caption{How to write algorithms}
	\end{algorithm}
\end{frame}
