\section{Alternatives to improve DOM}
\subsection{Conservative DOM}

\begin{frame}
	\frametitle{CDOM}
	Following the recent work on classical model Boltzmann equation by Prof. JC Huang, \cite{Huang2011261} and the article by Prof. Miussens \cite{Mieussens:2000:DMN:351065.351073}, an alternative way to reduce the number of discrete velocities has been observed while keeping the decent aquaracy of DOM strategy with a small computational cost.
\end{frame}
	
\begin{frame}
	To understand how this is done, we start by examing the formulation of the conservative constrain for classical gas. \\ 
	Given the relations between mass, momentum and energy densities with the distribution fucntions are the first three moments,
	\begin{equation}
		\left( \rho, \rho u_{\vec{x}}, E \right)^{T} = \int \phi f d\Theta,
	\label{eq:collision_invariants}
	\end{equation}
	where $\phi=(1,v_{\vec{x}},v^2/2)^{T}$ is identified as the vector of collision invariants. \\
\end{frame}

\begin{frame}
	Let us consider the class of Boltzmann-BGK equation,
	\begin{equation}
		\frac{\partial f}{\partial t} + \vec{c} \bullet \frac{\partial f}{\partial \vec{x}} = -\frac{f - f^{eq}}{\tau}
	\label{eq:kinetic_boltzmann_BGK}
	\end{equation}
	Multiplying both sides by the collision vector (\ref{eq:collision_invariants}) and integrating over the velocity domain, we arrive to:
	\begin{equation}
		\int \phi(f^{eq}-f)d\Theta = 0.
	\label{eq:conservation_principle}
	\end{equation}
	Equation (\ref{eq:conservation_principle}) is a system of equations and here is called the conservation constrain, and will be imposed to ensure conservation of macroscopic quantities.
\end{frame}

\begin{frame}
	\frametitle{Conservative constrain}
	For Maxwellian gases, such as a monoatomic or non-reacting gas; mass, momentum and energy are always conserved. \\ 
	Thus $f$ and $f^{eq}$ satisfy the conservative constrain,
		\begin{equation}
			\int \phi (f^{eq}-f) d\Theta = 0
		\end{equation}
		where $\phi = (1, c_x, c_y, c_z, (c^2_x+c^2_y+c^2_z)/2)^T$ is the vector of collision of invariants and $d\Theta = dc_x dc_y dc_z$.
\end{frame}

\begin{frame}
	\frametitle{Moments of the distribution funcion}
	Recall that the main three moments of the classical distribution function are,
	\begin{align*}
		&\int{
		\begin{bmatrix}
			1 			\\
			c_i 		\\
			\vec{v}
		\end{bmatrix}
		} f(\vec{x},\vec{c},t)d^3c = 
		\begin{bmatrix}
			\rho(\vec{x},t)		\\
			\rho u_i(\vec{x},t)	\\
			\frac{3}{2}\rho RT(\vec{x},t)
		\end{bmatrix}, & i = 1,2,3
	\end{align*}
	where $\vec{v} = \frac{(\vec{c}-\vec{u})^2}{2}$ is the the peculiar velocity of a molecule.
\end{frame}

\begin{frame}
	\frametitle{gas pressure and stress tensor $\tau_{ij}$}
		and the gas pressure and stress tensor are defined by,
		\begin{align*}
			p(\vec{x},t) &= \rho(\vec{x},t) k_B T(\vec{x},t) \\
			\tau_{ij} &= \int c_i c_j f(\vec{x},\vec{c},t) d^3c - p\delta_{ij}
		\end{align*}
		here $k_B$ is the Boltzmann constant and $\delta_{ij}$ is the Kroneker delta.
		The heat vector flux $\vec{q}$ is defined by,
		\begin{align*}
			q_i (\vec{x},t) &= \int \frac{c^2}{2} c_i f(\vec{x},\vec{c},t)d^3c.
		\end{align*}
\end{frame}

\begin{frame}
	\frametitle{Reduced Distributions}
	However Prof. Huang in \cite{Huang2011261} is using the reduced distribution method on his three-dimensional classical model.\\
	His reduced distributions functions are defined as,
	\begin{align*}
			g(x,y,c_x,c_y,t) &= \int^{\infty}_{-\infty} f(x,y,c_x,c_y,t) dc_z \\
			h(x,y,c_x,c_y,t) &= \int^{\infty}_{-\infty} c^2_z f(x,y,c_x,c_y,t) dc_z
	\end{align*}
	and using DOM they further reduce to,
	\begin{align*}
			&g(x,y,c_l,c_m,t) = g_{l,m}(x,y,t),& &h(x,y,c_l,c_m,t) = h_{l,m}(x,y,t)&
	\end{align*}
	here $l = 1, \dots ,N_l$ and $m = 1, \dots N_m$ are discrete values in velocity space.
\end{frame}

\begin{frame}
	\frametitle{2D system}
	A reduced two dimensional system is then derived as follows,
	\begin{align*}
	\frac{\partial g_{l,m}}{\partial t} + c_l \frac{\partial g_{l,m}}{\partial x} + c_m \frac{\partial g_{l,m}}{\partial y} &= \frac{1}{\tau}(G^{eq}_{l,m} - g_{l,m}), \\
	\frac{\partial h_{l,m}}{\partial t} + c_l \frac{\partial h_{l,m}}{\partial x} + c_m \frac{\partial h_{l,m}}{\partial y} &= \frac{1}{\tau}(H^{eq}_{l,m} - h_{l,m}).
	\end{align*}
	Note that we are interested in using the semiclassical formulation eventualy. However we wish to explore first the classical formulation to enlight eventualy our path.
\end{frame}

\begin{frame}
	\frametitle{Conservative constrain}
	from the conservative constrain arises to the following nonlinear system of equations,
	\begin{equation}
		\int 
			\begin{bmatrix}
			f^{eq}-f\\ 
			c_\vec{x}(f^{eq}-f)\\ 
			\vec{v}(f^{eq}-f)
			\end{bmatrix}d\vec{c} 
			=
			\begin{bmatrix}
			\int f^{eq} d\vec{c} - \int f d\vec{c}\\ 
			\int c_\vec{x} f^{eq} d\vec{c} - \int c_\vec{x} f d\vec{c}\\ 
			\int \vec{v} f^{eq} d\vec{c} - \int \vec{v} f d\vec{c}
			\end{bmatrix}
			= \vec{0}
	\end{equation}
	in one-dimension we can write the last expression as, 
	\begin{equation}
			\begin{bmatrix}
			\int \left(\frac{\rho(x,t)}{2 \pi RT}\right)^{1/2} \exp \left(\frac{(c-u)^2}{2RT} \right) dc - \rho(x,t) \\
			\int c \left(\frac{\rho(x,t)}{2 \pi RT}\right)^{1/2} \exp \left(\frac{(c-u)^2}{2RT} \right) dc - \rho u(x,t) \\
			\int v \left(\frac{\rho(x,t)}{2 \pi RT}\right)^{1/2} \exp \left(\frac{(c-u)^2}{2RT} \right) dc - \frac{3}{2} \rho RT(x,t)
			\end{bmatrix} 
			=\vec{0} 
	\label{eq:conservation_constrain_classical}
	\end{equation}
\end{frame}

\begin{frame}
	re-arrange some of the terms in (\ref{eq:conservation_constrain_classical}) and subtituting the integrals by a suitable quadrature rule we obtain:
	\begin{equation}
			\begin{bmatrix}
			\sum w_i \left(\frac{\rho(x,t)}{2 \pi RT(x,t)}\right)^{1/2} \exp \left(\frac{(c_i-u)^2}{2RT(x,t)} \right) - \rho(x,t) \\
			\sum w_i \frac{c}{\rho(x,t)} \left(\frac{\rho(x,t)}{2 \pi RT(x,t)}\right)^{1/2} \exp \left(\frac{(c_i-u)^2}{2RT(x,t)} \right) - u(x,t) \\
			\sum w_i \frac{2v}{3R\rho(x,t)} \left(\frac{\rho(x,t)}{2 \pi RT(x,t)}\right)^{1/2} \exp \left(\frac{(c_i-u)^2}{2RT(x,t)} \right) - T(x,t)
			\end{bmatrix} 
			=\vec{0} 
			\label{eq:conservation_constrain_classical_explicit}
	\end{equation}
	Note that we must solve a non-linear system of equations.
\end{frame}

\begin{frame}
	To solve the system in (\ref{eq:conservation_constrain_classical_explicit}) we can use any root finding method. We chosse Newton method as in \cite{Huang2011261} to solve our nonlinear system becuase it is the fastest method available,
	\begin{align*}
			& \vec{x} \; : \; f(\vec{x}) = \vec{0}
	\end{align*}
	The solution is obtained by iterating,
	\begin{align*}
			& \vec{x}_{n+1} = \vec{x}_{n}-\frac{f(\vec{x}_n)}{f'(\vec{x}_n)},& &\text{for: }n=0,1,\dots&
	\end{align*}
	by recognizing that $\vec{x}_n = [\rho(x,t),u(x,t),T(x,t)]_n^T$ and $\vec{x}_0$ is a vector with initial values computed using traditional DOM.
\end{frame}

\begin{frame}
	Note that for a system of equations $f'(\vec{x})$ is a Jacobian matrix.
	Then our Jacobian is defined as,
	\begin{align*}
		 [J] = f'(\vec{x}) = 
		\begin{bmatrix}
			\frac{\partial f_1}{\partial x_1} & \frac{\partial f_1}{\partial x_2} & \frac{\partial f_1}{\partial x_3} \\[0.3em]
			\frac{\partial f_2}{\partial x_1} & \frac{\partial f_2}{\partial x_2} & \frac{\partial f_2}{\partial x_3} \\[0.3em]
			\frac{\partial f_3}{\partial x_1} & \frac{\partial f_3}{\partial x_2} & \frac{\partial f_3}{\partial x_3} 
		\end{bmatrix}
	\end{align*}
	where $f = [f_1,f_2,f_3]^T$ and $x_n = [x_1,x_2,x_3]_n^T$. \\
	Then we can iterate for the solution of the system by computing,
	\begin{align*}
		& \vec{x}_{n+1} = \vec{x}_{n} - [J]^{-1} f(\vec{x}_n) ,& &\text{for: }n=0,1,\dots&
	\end{align*}
	
\end{frame}


%\subsection{Numerical Results with CDOM}

\begin{frame}
	\frametitle{Numerical Results with CDOM}
		
		\begin{figure}
			\centering
				\includegraphics[width=0.50\textwidth]{DN1n4-61}
			\caption{Lax's Euler Problem. Using MB classical distribution function, using WENO3 with a 100 poinst in physical space and DOM with 61 velocity points.}
			\label{fig:DN1n4-61}
		\end{figure}
		
\end{frame}

\begin{frame}
	\frametitle{Numerical Results with CDOM}
	
		\begin{figure}
			\centering
				\includegraphics[width=0.50\textwidth]{DN1n4-81}
			\caption{Lax's Euler Problem. Using MB classical distribution function, using WENO3 with a 100 poinst in physical space and DOM with 81 velocity points.}
			\label{fig:DN1n4-81}
		\end{figure}
	
\end{frame}

\begin{frame}
	\frametitle{Numerical Results with CDOM}
	
		\begin{figure}
			\centering
				\includegraphics[width=0.50\textwidth]{DN1n4-101}
			\caption{Lax's Euler Problem. Using MB classical distribution function, using WENO3 with a 100 poinst in physical space and DOM with 101 velocity points.}
			\label{fig:DN1n4-101}
		\end{figure}
	
\end{frame}

\begin{frame}
	\frametitle{Numerical Results with CDOM}
	
		\begin{figure}
			\centering
				\includegraphics[width=0.50\textwidth]{DC1n4-41}
			\caption{Lax's Euler Problem. Using MB classical distribution function, using WENO3 with a 100 poinst in physical space and CDOM with 41 velocity points.}
			\label{fig:DC1n4-41}
		\end{figure}
	
\end{frame}

\begin{frame}
	\frametitle{Numerical Results with CDOM}
	
		\begin{figure}
			\centering
				\includegraphics[width=0.50\textwidth]{DC1n4-61}
			\caption{Lax's Euler Problem. Using MB classical distribution function, using WENO3 with a 100 poinst in physical space and CDOM with 61 velocity points.}
			\label{fig:DC1n4-61}
		\end{figure}
	
\end{frame}