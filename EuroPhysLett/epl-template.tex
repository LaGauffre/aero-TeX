\documentclass[doublecol]{epl2} 
% or \documentclass[page-classic]{epl2} for one column style

\title{Semiclassical Lattice Ellipsoidal-Statistical BGK Model for Hydrodynamic Flows of Arbitrary Statistics}
\shorttitle{SLBM-ESBGK Model for Hydrodynamics of Arbitrary Statistics} %Insert here a short version of the title if it exceeds 70 characters

\author{J. Y. Yang\inst{1,2} \and Po-Chen Tsai\inst{1} \and Manuel Diaz\inst{1}}
\shortauthor{F. Author \etal}

\institute{                    
  \inst{1} Institute of Applied Mechanics - National Taiwan
University, Taipei 106, TAIWAN\\
  \inst{2} Center for Advanced Studies in Theoretical Sciences - National Taiwan
University, Taipei 106, TAIWAN
}
\pacs{47.11.-j}{First pacs description}
\pacs{51.10.+y}{Second pacs description}
\pacs{47.45.Ab}{Third pacs description}
\pacs{67.10.Jn}{fourth pacs description}

\abstract{A  semiclassical lattice Boltzmann method is presented based on the Uehling-Uhlenbeck Boltzmann ellipsoidal statistical (ES) kinetic equation which was developed by L. Wu et al. (Proc. R. Soc. A 2012). The method is directly derived by projecting the kinetic governing equation onto the tensor Hermite polynomials and various hydrodynamic approximation orders can be achieved.  The semiclassical lattice Boltzmann-BGK method (Yang and Hung, Phys. Rev. E, 2009) is extended and generalized.  To determine the anisotropic ellipsoidal statistical distribution function, additional information is needed and a corresponding different decoding procedure is required as compared with that for BGK model.  The semiclassical lattice Boltzmann-ES method shares the simplicity of semiclassical lattice Boltzmann-BGK method but has correct Prandtl number.   The semiclassical incompressible Navier-Stokes equations can be recovered via a Chapman-Enskog multi-scale expansion.    Simulations of the lid-driven square cavity flows in semiclassical viscous fluids based on D2Q9 lattice model for several Reynolds numbers and different particle statistics are shown to illustrate the method.    The results indicate distinct characteristics of the effects of particle statistics.}

\begin{document}

\maketitle


\section{Section title}
Insert here the text.
See fig.~\ref{fig.1}, table~\ref{tab.1} and eq.~(\ref{eq.1}).
See also~\cite{b.a,b.b}.
\begin{equation}
\label{eq.1}
0\neq1
\end{equation}


\begin{figure}
\onefigure{epl-template.eps}
\caption{Figure caption.}
\label{fig.1}
\end{figure}


\begin{table}
\caption{Table caption.}
\label{tab.1}
\begin{center}
\begin{tabular}{lcr}
first  & table & row\\
second & table & row
\end{tabular}
\end{center}
\end{table}



\acknowledgments
Insert here the text.

\begin{thebibliography}{0}

\bibitem{b.a}
  \Name{Author F., Author S. \and Author T.}
  \REVIEW{Some Rev. A}{69}{1969}{9691}.

\bibitem{b.b}
  \Name{Author F. \and Author S.}
  \Book{Some Book of Interest}
  \Editor{A. Editor}
  \Vol{9}
  \Publ{Publishing house, City}
  \Year{1939}
  \Page{666}.

\bibitem{b.c}
  \Editor{Editor A.}
  \Book{Some Book of Interest}
  \Vol{9}
  \Publ{Publishing house, City}
  \Year{1939}
  \Section{A}.

\end{thebibliography}

\end{document}

