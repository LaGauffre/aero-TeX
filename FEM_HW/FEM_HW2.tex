%\documentclass[a4paper]{article}
\documentclass[a4paper]{memoir}

% Preamble
\usepackage{calc}
\usepackage{color}
\usepackage{graphicx}
\usepackage{amsmath}
\usepackage{amssymb}

%define title and other basic document info
%the title should reflect the style and give a foretaste of the document
%work on making a stylized title page — or title on a page as in ARTICLE class
\title{\huge \textbf{Homework Problems No.2}}
\author{Manuel A. Diaz \\ f99543083}
\date{March 12th, 2013} % could use \today
%\publisher{Institute of Applied Mechanics, IAM}  %one day I’ll need this  
%\thanks{Special thanks to God for the ability to work}        %produces a footnote to the title

\definecolor{shadecolor}{gray}{0.9}
\definecolor{ared}{rgb}{.647,.129,.149}
\renewcommand\colorchapnum{\color{ared}}
\renewcommand\colorchaptitle{\color{ared}}
\chapterstyle{bringhurst}
%one of a number of chapter styles available…this one doesn’t use the ared color

% Begin Document Here
\begin{document}

% Build costumized Title Page
\thispagestyle{empty}
%\begin{minipage}{300pt}
\begin{center}{
\begin{shaded}
\hrule \vspace{30pt}
\hspace{30pt} \thetitle  \vspace{30pt}
\newline \theauthor \hspace{30pt} \thedate  \vspace{26pt}
\hrule
\end{shaded}
}
\end{center}
%\end{minipage}
\clearpage

%\frontmatter    %use if needed –page numbers as lower case roman numerals i, ii,…

%\mainmatter
%%other declarations
\pagestyle{Ruled}                    %one of a number of possible page styles
\midsloppy                             %to minimize overfull lines

%Layout the page
%%Try this manual golden ratio layout or…           default seems better for now
%\settypeblocksize{*}{\lxvchars}{1.618}
%\setulmargins{50pt}{*}{*}
%\setlrmargins{*}{*}{1.618}
%\setheaderspaces{*}{*}{1.618}
%\semiisopage[12]
%try this predefined layout — others predefined ones are options in MEMOIR…
%this one looked best but did not work

\checkandfixthelayout          %make the layout happen and provide details in log during build

\chapter{Homework Problems No. 2}
\section{Problem 1}
\subsection{Given}
Given the Example 3 in Chapter 1 of the lecture notes. See figure \ref{fig:uniform_cantilever}.

\begin{figure}
	\centering
		\includegraphics[width=0.5\textwidth]{uniform_cantilever.png}
	\caption{Example 3, uniform section cantilever}
	\label{fig:uniform_cantilever}
\end{figure}

\subsection{Find}
\begin{enumerate}
	\item The solution from a cubic polynomial $x(u) = a_0+a_1x+a_2x^2+a_3x^3$ using the Rayleigh-Ritz method.
	\item Compute and tabulate the force boundary conditions at x=L from the trial solution obtained from linear, quadratic and cubic polynomails and give a brief comment on your observation.
\end{enumerate}

\subsection{Plan}
From chapter 1 in lecture notes, we wish to find a smooth function $u(x)$ that satisfy $u = \bar{u}$ on $\Gamma_u$ such that

\begin{align*}
	&\int_\Omega \frac{dw}{dx} AE \frac{du}{dx} dx = (wA\bar{t})|_{\Gamma_t} + \int_\Omega wb dx,  & w = 0 \;\; on \;\; \Gamma_u
\end{align*}

Using Rayleigh-Ritz Method, the total energy for a 1D ideal elastic bar is:
\begin{equation}
	\Pi(u(x)) = \frac{1}{2} \int_\Omega AE \left( \frac{du}{dx} \right)^2 dx-\left( (uA\bar{t})|_{\Gamma_t} + \int_\Omega ub dx \right)
\end{equation}
And that teh variation of the functional must vanish, just as the differentials of a function vanish at a minimun of a function. \\
For the fixed cantilever show in figure \ref{fig:uniform_cantilever}, the total potentail energy is:
\begin{equation}
	\Pi = \frac{1}{2} EA \int^L_0 \left( \frac{du}{dx} \right)^2 dx - c \int^L_0 xudx - Pu(L)
	\label{eq:ubeam_potencial_energy}
\end{equation}
As requested, we assume the solution to be a cubic polynomial
\begin{align*}
	 & u(x) = a_0+a_1x+a_2x^2+a_3x^3,\\ & \Rightarrow \frac{du}{dx} = a_1+2a_2x+3a_3x^2, \\
	 & \Rightarrow \left( \frac{du}{dx} \right)^2 = (a_1^2+4a_1a_2+6a_1a_3x^2+4a_2^2x^2+6a_2a_3x^3+9a_3^2x^4).
\end{align*}
however due to our fixed BC.
\begin{align*}
	 u(0) &= 0, \Rightarrow  a_0 = 0
\end{align*}

\subsection{Calculations}
Substituting our assumed cubic solution in to equation (\ref{eq:ubeam_potencial_energy}), we 
\begin{align*}
	&
	\begin{split}
	\Pi &= \frac{1}{2} EA \int^L_0 (a_1^2+4a_1a_2+6a_1a_3x^2+4a_2^2x^2+6a_2a_3x^3+9a_3^2x^4)dx \\ 
			&- c \int^L_0 x(a_1x+a_2x^2+a_3x^3)dx - P(a_1L+a_2L^2+a_3L^3)
	\end{split} \\
	&\text{Integrating,} \\
	&
	\begin{split}
	\Pi &= \frac{1}{2} EA (a_1^2L+2a_1a_2L^2+2a_1a_3L^3+4/3a_2^2L^3+3a_2a_3L^4+9/5a_3^2L^5) \\
			&- c(\frac{1}{2}a_1L^2+\frac{2}{3}a_2L^3+\frac{3}{4}a_3L^5) - P(a_1L+a_2L^2+a_3L^3)
	\end{split} 
	\end{align*} 
	Taking the derivates respect to $a_i$ where $i = 1,2,3$, we get
	\begin{align*}
	\frac{\partial \Pi}{\partial a_1} &=	\frac{1}{2} EA (2a_1L+2a_2L^2+2a_3L^3) - \frac{1}{2}CL^2 - PL = 0 \\
	\frac{\partial \Pi}{\partial a_2} &=	\frac{1}{2} EA (2a_1L^2+8/3a_2L^3+3a_3L^4) - \frac{2}{3}CL^3 - 2PL^2 = 0 \\
	\frac{\partial \Pi}{\partial a_3} &=	\frac{1}{2} EA (2a_1L^3+3a_2L^4+18/5a_3L^5) - \frac{3}{4}CL^4 - 3PL^3 = 0 
	\end{align*}
	Solving this equations,
	
	Substituting into the assumed cubic polynomail, we get the following solution for the problem:

\subsection{Solution}
\begin{minipage}{300pt}
	\begin{center}{
		\begin{shaded}
			\hrule
			\vspace{20pt}
			The mass of water in the pool is $1.44 \times 10^5 kg$  %nicely written sentence solution goes here
			\vspace{16pt}
			\hrule
		\end{shaded}
	}
	\end{center}
\end{minipage}

Tabulating the force boundary condition for $x=L$

\begin{table*}[htbp]
	\centering
		\begin{tabular}
			
		\end{tabular}
	\caption{Force Boundary Conditions at $x=L$}
	\label{tab:ForceBoundaryConditionsAtXL}
\end{table*}

We observe: 

\section{Problem 2}
\subsection{Given}
The tapered bar fixed at one end and subject to a static point load at the other end as show in figure ref{fig:tappered cantilever}.
The bar is subject to a linearly varying axial load $q=cx$, where $c$ is a given constant. The area varies linearly from $A_0$ to $A_L$ where,
\begin{align*}
	&A(x) = \frac{(l+(-1+r)x)A_0}{L}	& r = A_l/A_0
\end{align*}

\begin{figure}
	\centering
		\includegraphics[width=0.5\textwidth]{tappered_cantilever.png}
	\caption{Tapered cantilever}
	\label{fig:tappered_cantilever}
\end{figure}

\subsection{Find}
Obtain the solution from a linear polynimal $u(x) = a_0+a_1x$ using Rayleight-Ritz method 

\subsection{Plan}
Follwing Problem 1, we know ...

\subsection{Calculations}
\begin{eqnarray*}
V &=& \frac{3+1}{2} \times 12 \times 6
\end{eqnarray*}

\subsection{Solution}
\begin{minipage}{300pt}
	\begin{center}{
		\begin{shaded}
			\hrule
			\vspace{20pt}
			The mass of water in the pool is $1.44 \times 10^5 kg$  %nicely written sentence solution goes here
			\vspace{16pt}
			\hrule
		\end{shaded}
	}
	\end{center}
\end{minipage}

\section{Bonus Problem}
\subsection{Given}
Consider again the tapered bar fixed at one end and subject to a static point load in problem 2.

\subsection{Find}
Continue the derivation from problem 2 and obtain the solution from a quadratic polynomial $x(u) = a_0+a_1x+a_2x^2$ using Rayleigh-Ritz method.

\subsection{Plan}
From problem 2, we know ...

\subsection{Calculations}
\begin{eqnarray*}
V &=& \frac{3+1}{2} \times 12 \times 6
\end{eqnarray*}

\subsection{Solution}
\begin{minipage}{300pt}
	\begin{center}{
		\begin{shaded}
			\hrule
			\vspace{20pt}
			The mass of water in the pool is $1.44 \times 10^5 kg$  %nicely written sentence solution goes here
			\vspace{16pt}
			\hrule
		\end{shaded}
	}
	\end{center}
\end{minipage}

\end{document}