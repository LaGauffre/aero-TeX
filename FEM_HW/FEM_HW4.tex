\documentclass[a4paper]{memoir}

% Preamble
\usepackage{calc}
\usepackage{color}
\usepackage{graphicx}
\usepackage{amsmath}
\usepackage{amssymb}

%define title and other basic document info
%the title should reflect the style and give a foretaste of the document
%work on making a stylized title page — or title on a page as in ARTICLE class
\title{\huge \textbf{Homework Problems No.4}}
\author{Manuel A. Diaz \\ f99543083}
\date{March 26th, 2013} % could use \today
%\publisher{Institute of Applied Mechanics, IAM}  %one day I’ll need this  
%\thanks{Special thanks to God for the ability to work}        %produces a footnote to the title

\definecolor{shadecolor}{gray}{0.9}
\definecolor{ared}{rgb}{.647,.129,.149}
\renewcommand\colorchapnum{\color{ared}}
\renewcommand\colorchaptitle{\color{ared}}
\chapterstyle{bringhurst}
%one of a number of chapter styles available…this one doesn’t use the ared color

% Begin Document Here
\begin{document}

% Build costumized Title Page
\thispagestyle{empty}
%\begin{minipage}{300pt}
\begin{center}{
\begin{shaded}
\hrule \vspace{30pt}
\hspace{30pt} \thetitle  \vspace{30pt}
\newline \theauthor \hspace{30pt} \thedate  \vspace{26pt}
\hrule
\end{shaded}
}
\end{center}
%\end{minipage}
\clearpage

%\frontmatter    %use if needed –page numbers as lower case roman numerals i, ii,…

%\mainmatter
%%other declarations
\pagestyle{Ruled}                    %one of a number of possible page styles
\midsloppy                             %to minimize overfull lines

%Layout the page
%%Try this manual golden ratio layout or…           default seems better for now
%\settypeblocksize{*}{\lxvchars}{1.618}
%\setulmargins{50pt}{*}{*}
%\setlrmargins{*}{*}{1.618}
%\setheaderspaces{*}{*}{1.618}
%\semiisopage[12]
%try this predefined layout — others predefined ones are options in MEMOIR…
%this one looked best but did not work

\checkandfixthelayout          %make the layout happen and provide details in log during build

\chapter{Homework Problems No. 4}
\section{Problem 1}
\subsection{Given}
Given the Example of the Tapered Elastic Bar in Chapter 1 of the lecture notes. See figure \ref{fig:tappered_beam}.

\begin{figure}
	\centering
		\includegraphics[width=0.6\textwidth]{tappered_beam.jpg}
	\caption{Example 1, Tappered Elastic Bar using three-point quadratic element}
	\label{fig:tappered_beam}
\end{figure}

\subsection{Find}
Derive $K_{22}$ using one-point, two-point and three-point Guass Quadrature

\subsection{Plan}
Recall the shape functions and $\vec{B}$ matrix for a three-done quadratic element:
\begin{align*}
	N_1^{(1)} &= \frac{(x-x_2^{(1)})(x-x_3^{(1)})}{(x_1^{(1)}-x_2^{(1)})(x_1^{(1)}-x_3^{(1)})} = \frac{(x-4)(x-6)}{(-2)(-4)} = 1/8 (x^2-10x+24) \\
	N_2^{(1)} &= \frac{(x-x_1^{(1)})(x-x_3^{(1)})}{(x_2^{(1)}-x_1^{(1)})(x_2^{(1)}-x_3^{(1)})} = \frac{(x-2)(x-6)}{(2)(-2)} = -1/4 (x^2-8x+12) \\
	N_3^{(1)} &= \frac{(x-x_1^{(1)})(x-x_2^{(1)})}{(x_3^{(1)}-x_1^{(1)})(x_3^{(1)}-x_2^{(1)})} = \frac{(x-2)(x-4)}{(4)(2)} = 1/8 (x^2-6x+8) 
\end{align*}

and 
\begin{align*}
	B_1^{(1)} &= \frac{\partial N_1^{(1)}}{\partial x} = 1/4 (x-5) \\
	B_2^{(1)} &= \frac{\partial N_2^{(1)}}{\partial x} = -1/2 (x-4) \\
	B_3^{(1)} &= \frac{\partial N_3^{(1)}}{\partial x} = 1/4 (x-3)
\end{align*}

Thus,
\begin{align*}
	\vec{N}^{(1)}(x) &= [1/8 (x^2-10x+24)& &,-1/4 (x^2-8x+12)& &,1/8 (x^2-6x+8) &] \\
	\vec{B}^{(1)}(x) &= [1/4 (x-5)& &,-1/2 (x-4)& &,1/4 (x-3) &]
\end{align*}

The Stiffness Matrix '$K$' for our three-point element as,

\begin{equation}
	k = K^{(1)} = \int_{x_1^{(1)}}^{x_3^{(1)}} \vec{B^{(1)}}^T A^{(1)} B^{(1)} \vec{B^{(1)}} dx
	\label{eq:stiffness_matrix}
\end{equation}

\subsection{Calculations}
Substitute our definition of vector $\vec{B^{(1)}}$ into equation (\ref{eq:stiffness_matrix}), we obtain:

\begin{align*}
	k &= \int_2^6 \frac{1}{4} {
	\begin{bmatrix}
		(x-5) \\
		(8-2x) \\
		(x-3)
	\end{bmatrix}
	} (2x) (8) \frac{1}{4} [(x-5),(8-2x),(x-3)]dx \\
	  &= \int_2^6 \frac{1}{4} {
	\begin{bmatrix}
				x(x-5)^2 & x(x-5)(8-2x) & x(x-5)(x-3) \\
				x(8-2x)(x-5) & x(8-2x)^2 & x(8-2x)(x-3) \\
				x(x-3)(x-5) & x(x-3)(8-2x) & x(x-3)^2
	\end{bmatrix}
	} dx
\end{align*}

Therefore Element $K_{22}$ is,
\begin{equation}
	K_{22} = \int_2^6 x(x-3)^2 dx
	\label{eq:k22}
\end{equation} 

Using 1 gauss quadrature point we have, \\

\begin{center}
	\begin{tabular}{l | c | r}
		\hline
		$i$ & 	$\xi_i$ 	&	$w_i$ \\ \hline
		1 	& 	0 				&	2		
	\end{tabular}
\end{center} 

then, equation (\ref{eq:k22}) becomes, 
\begin{align*}
	K_{22} &= \frac{b-a}{2} \sum_i^1 w_i f(\frac{b+a}{2}+\frac{b-a}{2}\xi_i)
\end{align*}
in out case $a=2$ and $b=6$ therefore, 
\begin{align*}
	\xi_1 &= \frac{6+2}{2}+\frac{6-4}{2}(0) = 4 \\
	f(\xi_1) &= 4(4-3)^2 = 4 \\
	K_{22} &= \frac{6-4}{2} (2) (4) = 16
\end{align*}

Using 2 gauss quadrature points we have, \\

\begin{center}
	\begin{tabular}{l | c | r}
		\hline
		$i$ & 	$\xi_i$ 		&	$w_i$ \\ \hline
		1 	& $-1/\sqrt{3}$	&	1			\\ \hline
		2 	& $1/\sqrt{3}$	&	1
	\end{tabular}
\end{center} 

then, equation (\ref{eq:k22}) becomes, 
\begin{align*}
	K_{22} &= \frac{b-a}{2} \sum_i^2 w_i f(\frac{b+a}{2}+\frac{b-a}{2}\xi_i)
\end{align*}
in out case $a=2$ and $b=6$ therefore, 
\begin{align*}
	\xi_1 &= \frac{6+2}{2}+\frac{6-4}{2}(-1/ \sqrt{3}) = 4-2/\sqrt{3} \\
	\xi_2 &= \frac{6+2}{2}+\frac{6-4}{2}(1/ \sqrt{3}) = 4+2/\sqrt{3} \\
	f(\xi_1) &= 4-2/\sqrt{3}(4-2/\sqrt{3}-3)^2 = 23.9319 \\
	f(\xi_2) &= 4+2/\sqrt{3}(4+2/\sqrt{3}-3)^2 = 0.0681 \\
	K_{22} &= \frac{6-4}{2} \left[ (1)(23.9319) + (1)(0.0681) \right] = 48
\end{align*}

Using 3 gauss quadrature points we have, \\

\begin{center}
	\begin{tabular}{l | c | r}
		\hline
		$i$ & 	$\xi_i$ 		&	$w_i$ \\ \hline
		1 	& $-\sqrt{3/5}$	&	5/9		\\ \hline
		2 	& 0							&	8/9		\\ \hline
		3 	& $\sqrt{3/5}$	&	5/9
	\end{tabular}
\end{center} 

then, equation (\ref{eq:k22}) becomes, 
\begin{align*}
	K_{22} &= \frac{b-a}{2} \sum_i^3 w_i f(\frac{b+a}{2}+\frac{b-a}{2}\xi_i)
\end{align*}
in out case $a=2$ and $b=6$ therefore, 
\begin{align*}
	\xi_1 &= \frac{6+2}{2}+\frac{6-4}{2}(-\sqrt{3/5}) = 4-2\sqrt{3/5} \\
	\xi_2 &= \frac{6+2}{2}+\frac{6-4}{2}(0) = 4 \\
	\xi_3 &= \frac{6+2}{2}+\frac{6-4}{2}(\sqrt{3/5}) = 4+2\sqrt{3/5} \\
	f(\xi_1) &= 4-2/\sqrt{3}(4-2\sqrt{3/5}-3)^2 = 36.0608 \\
	f(\xi_2) &= 4(4-3)^2 = 4 \\
	f(\xi_3) &= 4+2/\sqrt{3}(4+2\sqrt{3/5}-3)^2 = 0.0681 \\
	K_{22} &= \frac{6-4}{2} \left[ (5/9)(2.8453) + (8/9)(4) + (5/9)(5.1547) \right] = 48
\end{align*}

\subsection{Solution}
\begin{minipage}{300pt}
	\begin{center}{
		\begin{shaded}
			\hrule
			\vspace{20pt}
			Using 2 or more gauss points we get, $K_{22} = 48.00$ %nicely written sentence solution goes here
			\vspace{16pt}
			\hrule
		\end{shaded}
	}
	\end{center}
\end{minipage}

\section{Problem 2}
\subsection{Given}
Given the following Differential Equation
\begin{align*}
	&-\frac{d^2 u}{dx^2} = cos(\pi x), &\text{for } 0<x<1, \\
	& u(0) = 0, \:\: u(1) = 0
\end{align*}

\begin{enumerate}
	\item Derivate element ''stiffness'' matrix $K^{(e)}$ and element external ''force'' vector $\vec{f}^{(e)}$ using linear shape functions.
	\item Use the uniform mesh of three linear elements to solve the problem. Plot your solution $u(x)$ and its derivate $\frac{du(x)}{dx}$ against the exact solution $u_{exact}(x) = \frac{1}{\pi^2}(cos(\pi x)+2x-1)$ and its derivate.
\end{enumerate}

\subsection{Plan}
Compute the corresponding weakformulation of the problem by re-arranging and integrating equation our problem ODE into,
\begin{equation}
	\int_0^1 w\left[ \frac{\partial}{\partial x} \left( \frac{\partial u}{\partial x} \right) + cos(\pi x) \right] dx = 0
	\label{eq:galerkin}
\end{equation}

by using the relation;
\begin{equation}
	\int_0^1 f \frac{\partial g}{\partial x}dx = fg|_{x=1} - fg|_{x=0} - \int_0^1 g df
	\label{eq:diverge_theorem}
\end{equation}

by substituing equation (\ref{eq:diverge_theorem}) into equation (\ref{eq:galerkin}) we obtain,
\begin{align*}
		\left( w\frac{du}{dx}\right)|_{x=1} &-\left( w\frac{du}{dx}\right)|_{x=0} - \int_0^1 \frac{du}{dx}\frac{dw}{dx}dx +\int_0^1 w cos(\pi x) dx = 0 \\
		& \text{with} \\
		& u(0) = 0, \;\; u(1)=0 \\
		& w(0) = 0, \;\; w(1)=0 
\end{align*}

Therefore our weak formulation is,
\begin{align*}
	&\int_0^1 \frac{du}{dx}\frac{dw}{dx}dx + \int_0^1 w cos(\pi x) dx = 0, \\
	& \text{with} \\
	& u(0) = 0 \:\: \text{and} \:\: u(1) = 0.
\end{align*}

Using 3 symetric linear elements the equation in our weak formulation can be written as,
\begin{equation}
	\sum_{e=1}^3 \int_{x_1^{(e)}}^{x_3^{(e)}} \frac{\partial w^e}{\partial x} \frac{\partial u^e}{\partial x} dx - 
		\int_{x_1^{(e)}}^{x_3^{(e)}} (w^e)^T cos(\pi x) dx = 0
\end{equation}

Define:
\begin{align*}
	& u^e(x) = \vec{N}^e(x) u^e = [N_1^{(1)}(x),N_2^{(1)}(x)]
	\begin{bmatrix}
		u_1^e \\
		u_2^e
	\end{bmatrix} \\
	& \frac{du^e(x)}{dx} = \vec{B}^e(x) u^e = [B_1^{(1)}(x),B_2^{(1)}(x)] 
	\begin{bmatrix}
		u_1^e \\
		u_2^e
	\end{bmatrix}
\end{align*}

Then our Weak formulation for the three elements becomes,
\begin{equation}
	\sum_{e=1}^3 (\vec{w}^{e})^T \left[ \int_{x_1^{(e)}}^{x_3^{(e)}} (\vec{B}^e)^T (\vec{B}^e) dx - \int_{x_1^{(e)}}^{x_3^{(e)}} (\vec{N}^e)^T Cos(\pi x) dx \right] = 0
\end{equation}

Where:
\begin{align*}
	& K^e = \int_{x_1^{(e)}}^{x_3^{(e)}} (\vec{B}^e)^T (\vec{B}^e) dx, & \vec{f}^e = \vec{f}_\omega^e = \int_{x_1^{(e)}}^{x_3^{(e)}} (\vec{N}^e)^T Cos(\pi x) dx
\end{align*}
are the ''Stiffness'' matrix and ''Body'' force vector respectivelly.

\subsection{Calculations}
Define element nodes,
\begin{align*}
	&x_1^{(1)} = 0, &x_1^{(2)} = 1/3 &x_1^{(3)} = 2/3 \\
	&x_2^{(1)} = 1/3, &x_2^{(2)} = 1/3 &x_2^{(3)} = 1 
\end{align*}

Using the linear elements we build the following shape function,
\begin{align*}
	N_1^{(1)} &= \frac{x-x_2^{(1)}}{x_1^{(1)}-x_2^{(1)}} = \frac{x-1/3}{0-1/3} = 1-3x \\
	N_2^{(1)} &= \frac{x-x_1^{(1)}}{x_2^{(1)}-x_1^{(1)}} = \frac{x-0}{1/3-0} = 3x \\
	N_1^{(2)} &= \frac{x-x_2^{(1)}}{x_1^{(1)}-x_2^{(1)}} = \frac{x-2/3}{1/3-2/3} = 2-3x \\
	N_2^{(2)} &= \frac{x-x_1^{(1)}}{x_2^{(1)}-x_1^{(1)}} = \frac{x-1/3}{2/3-1/3} = 3x-1 \\ 
	N_1^{(3)} &= \frac{x-x_2^{(1)}}{x_1^{(1)}-x_2^{(1)}} = \frac{x-1}{2/3-1} = 3-3x \\
	N_2^{(3)} &= \frac{x-x_1^{(1)}}{x_2^{(1)}-x_1^{(1)}} = \frac{x-2/3}{1-2/3} = 3x-2
\end{align*}

and its corresponding derivates,
\begin{align*}
	&B_1^{(1)} = -3, 	&B_1^{(2)} = -3,&	&B_1^{(3)} = -3, \\
	&B_2^{(1)} = 3,		&B_2^{(2)} = 3,&	&B_2^{(3)} = 3.
\end{align*}

Then, the ''Stiffness'' matrix for each element is,
\begin{align*}
	& K^{(1)} = \int_0^{1/3} 
	\begin{bmatrix}
		-3 \\
		3
	\end{bmatrix}
	\begin{bmatrix}
		-3 & 3 
	\end{bmatrix}
	dx = \int_0^{1/3} 
	\begin{bmatrix}
		 9 & -9 \\
		-9 &  9
	\end{bmatrix}
	dx = 
	\begin{bmatrix}
		 3 & -3 \\
		-3 &  3
	\end{bmatrix}	
\end{align*}

Similarly
\begin{align*}
	& K^{(2)} = \int_{1/3}^{2/3} 
	\begin{bmatrix}
		-3 \\
		3
	\end{bmatrix}
	\begin{bmatrix}
		-3 & 3 
	\end{bmatrix}
	dx = 
	\begin{bmatrix}
		 3 & -3 \\
		-3 &  3
	\end{bmatrix}	\\
	& K^{(3)} = \int_{2/3}^{1} 
	\begin{bmatrix}
		-3 \\
		3
	\end{bmatrix}
	\begin{bmatrix}
		-3 & 3 
	\end{bmatrix}
	dx = 
	\begin{bmatrix}
		 3 & -3 \\
		-3 &  3
	\end{bmatrix}		
\end{align*}

The ''Body'' force in vector for every element is,
\begin{align*}
	\vec{f}_\Omega^{(1)} = \int_{0}^{1/3} 
	\begin{bmatrix}
		1-3x \\
		3x
	\end{bmatrix}
	cos(\pi x)dx = 
	\begin{bmatrix}
		\frac{3}{2 \pi^2} \\
		\frac{\sqrt{2}\pi-3}{2 \pi^2}
	\end{bmatrix}\\
	\vec{f}_\Omega^{(2)} = \int_{1/3}^{2/3} 
	\begin{bmatrix}
		2-3x \\
		3x-1
	\end{bmatrix}
	cos(\pi x)dx = 
	\begin{bmatrix}
		\frac{6-\sqrt{3}\pi}{2 \pi^2} \\
		\frac{\sqrt{2}\pi-6}{2 \pi^2}
	\end{bmatrix} \\
	\vec{f}_\Omega^{(3)} = \int_{2/3}^{1} 
	\begin{bmatrix}
		3-3x \\
		3x-2
	\end{bmatrix}
	cos(\pi x)dx = 
	\begin{bmatrix}
		\frac{3-\sqrt{3}\pi}{2 \pi^2} \\
		\frac{\sqrt{2}\pi-3}{2 \pi^2}
	\end{bmatrix}
\end{align*}

Suming the individual contibution of each element the total nodal ''Stiffness'' matrix, total vector ''Body ''force are, displacemente vector are, 
\begin{align*}
	&K = 
	\begin{bmatrix}
	 3 & -3 &  0 &  0 \\
	-3 &  6 & -3 &  0 \\
	 0 & -3 &  6 & -3 \\
	 0 &  0 & -3 &  3 
	\end{bmatrix}, 
	&\vec{f} = 
	\begin{bmatrix}
	\frac{ 3}{2\pi^2} \\
	\frac{ 3}{2\pi^2} \\
	\frac{-3}{2\pi^2} \\
	\frac{-3}{2\pi^2}
	\end{bmatrix},&
	&\vec{u} = 
	\begin{bmatrix}
	u_1 \\
	u_2 \\
	u_3 \\
	u_4 
	\end{bmatrix}
\end{align*}

However we know that for our case $u_1 = 0$ and $u_2 = 0$, therefore the only active Degrees of freedom are ${2,3}$ (i.e. $activeDOFs = TotalDOFs \cap PrescribedDOFs$). We then only need to solve the following system: 
\begin{align*}
	&
	\begin{bmatrix}
		 6 & -3 \\
		-3 &  6
	\end{bmatrix}
	\begin{bmatrix}
		u_2 \\
		u_3
	\end{bmatrix}
	 = 
	\begin{bmatrix}
		\frac{ 3}{2\pi^2}
		\frac{-3}{2\pi^2}
	\end{bmatrix}
	\rightarrow
	\begin{bmatrix}
		u_2 \\
		u_3
	\end{bmatrix}
	=
	\begin{bmatrix}
		 2/9 & -1/9 \\
		-1/9 &  2/9
	\end{bmatrix}
	\begin{bmatrix}
		\frac{ 3}{2\pi^2}
		\frac{-3}{2\pi^2}
	\end{bmatrix}
\end{align*}

We found that $u_2 = \frac{1}{6\pi^2}$ and $u_3 = \frac{-1}{6\pi^2}$. We can know compute the global function that represent every element using our original definitions of shape function $N^e(x)$ and its derivate $B^e(x)$. Therefore, 
\begin{align*}
	u^{(1)}(x) &= (1-3x) u_1^{(1)} + (3x) u_1^{(1)} = \frac{x}{2\pi^2}, x \in [0,1/3] \\
	u^{(2)}(x) &= (2-3x) u_1^{(2)} + (3x-1) u_1^{(2)} = \frac{1-2x}{2\pi^2}, x \in [1/3,2/3] \\
	u^{(3)}(x) &= (3-3x) u_1^{(3)} + (3x-2) u_1^{(3)} = \frac{x-1}{2\pi^2}, x \in [2/3,1]
\end{align*}

\begin{align*}
	\frac{\partial u^{(1)}(x)}{\partial x} &= -3 u_1^{(1)} + 3 u_1^{(1)} = \frac{1}{2\pi^2}, x \in [0,1/3] \\
	\frac{\partial u^{(2)}(x)}{\partial x} &= -3 u_1^{(2)} + 3 u_1^{(2)} = \frac{-1}{\pi^2}, x \in [1/3,2/3] \\
	\frac{\partial u^{(3)}(x)}{\partial x} &= -3 u_1^{(3)} + 3 u_1^{(3)} = \frac{1}{2\pi^2}, x \in [2/3,1]
\end{align*}

The Exact solution and its derivate are defined,
\begin{eqnarray}
u_exact(x) &=& \frac{1}{\pi^2}(cos(\pi x)+2x-1) \\
\frac{\partial u_exact(x)}{\partial x} &=& \frac{1}{\pi^2}(2-\pi sin(\pi x))
\end{eqnarray}

Now we proceed to plot the linear elements and compare them with the plot of the exact solution and its derivate using Matlab.

\section{Ploting Result}

\begin{figure}
	\centering
		\includegraphics[width=0.8\textwidth]{ODE_solution.jpg}
	\caption{Comparison of the approximate solution with 3 linear elements and the ODE's exact solution}
	\label{fig:ODE_solution}
\end{figure}

As we can observe in figure \ref{fig:ODE_solution} the approximation with 3 elements matchess faily the behavior of the function $u(x)$ and its derivate. Probably a second order solution would do a much better job in this problem.

\section{Problem 3}
\subsection{Given}
A flat aluminum bar has constant thickness of 10 mm and a variable profile as shown in the figure \ref{fig:weird_bar}. Use Lagrange interpolation to determine an expresion for its area of cross-section, and graph the interpolation with a Matlab Plot. The interpolation function should pass through all the seven data points.

\begin{figure}
	\centering
		\includegraphics[width=0.5\textwidth]{weird_bar.png}
	\caption{Problem 3, Simetrical non-unifrom bar with a uniform thickness of 10 mm.}
	\label{fig:weird_bar}
\end{figure}

\section{Plan}
Using a lagrange interpolation of the form:
\begin{align*}
		& L(x) = sum_{i=0}^k y_i l_i(x) \\
		& \text{where} \\
		& l_i(x)  = \prod_{0\leq m < k, k\neq i} \frac{x-x_m}{x_i-x_m}
\end{align*}

Tabulating the information of the bar, \\
\begin{center}
	\begin{tabular}{c | c | c | c}
	\hline
		i	&	$x_i$		& $h_i$ & $y_i=\frac{h_i}{2}$ \\ \hline
		0	& 0		&	8		&	4 	\\
		1	& 15	&	10	&	5		\\
		2 &	30	&	9		&	4.5	\\
		3	&	45	&	12	&	6		\\
		4 & 60	&	12	&	6		\\
		5	& 75	&	10	&	5		\\
		6	& 90	&	5		&	2.5	
	\end{tabular}
\end{center}

Due to the symmetry of the bar we proceed to use $y_i$ values to formulate our lagrange polynomial,
\begin{equation}
	\begin{split}
		y(x) &= y_0 \frac{(x-x_1)(x-x_2)(x-x_3)(x-x_4)(x-x_5)(x-x_6)}{(x_0-x_1)(x_0-x_2)(x_0-x_3)(x_0-x_4)(x_0-x_5)(x_0-x_6)} \\
					&+ y_1 \frac{(x-x_0)(x-x_2)(x-x_3)(x-x_4)(x-x_5)(x-x_6)}{(x_1-x_0)(x_1-x_2)(x_1-x_3)(x_1-x_4)(x_1-x_5)(x_1-x_6)} \\
					&+ y_2 \frac{(x-x_0)(x-x_1)(x-x_3)(x-x_4)(x-x_5)(x-x_6)}{(x_2-x_0)(x_2-x_1)(x_2-x_3)(x_2-x_4)(x_2-x_5)(x_2-x_6)} \\
					&+ y_3 \frac{(x-x_0)(x-x_1)(x-x_2)(x-x_4)(x-x_5)(x-x_6)}{(x_3-x_0)(x_3-x_1)(x_3-x_2)(x_3-x_4)(x_3-x_5)(x_3-x_6)} \\
					&+ y_4 \frac{(x-x_0)(x-x_1)(x-x_2)(x-x_3)(x-x_5)(x-x_6)}{(x_4-x_0)(x_4-x_1)(x_4-x_2)(x_4-x_3)(x_4-x_5)(x_4-x_6)} \\
					&+ y_5 \frac{(x-x_0)(x-x_1)(x-x_2)(x-x_3)(x-x_4)(x-x_6)}{(x_5-x_0)(x_5-x_1)(x_5-x_2)(x_5-x_3)(x_5-x_4)(x_5-x_6)} \\
					&+ y_6 \frac{(x-x_0)(x-x_1)(x-x_2)(x-x_3)(x-x_4)(x-x_5)}{(x_6-x_0)(x_6-x_1)(x_6-x_2)(x_6-x_3)(x_6-x_4)(x_6-x_5)} 
	\end{split}
\end{equation}

Subtitution values the tabulated values, the resulting polynomial A(x) is:
\begin{equation}
	\begin{split}
A(x) &= 2 [ y_0 \frac{(x-5)(x-4.5)(x-6)(x-6)(x-5)(x-2.5)}{(4-5)(4-4.5)(4-6)(4-6)(4-5)(4-2.5)} \\
					&+ y_1 \frac{(x-4)(x-4.5)(x-6)(x-6)(x-5)(x-2.5)}{(5-4)(5-4.5)(5-6)(5-6)(5-5)(5-2.5)} \\
					&+ y_2 \frac{(x-4)(x-5)(x-6)(x-6)(x-5)(x-2.5)}{(4.5-4)(4.5-5)(4.5-6)(4.5-6)(4.5-5)(4.5-2.5)} \\
					&+ y_3 \frac{(x-4)(x-5)(x-4.5)(x-6)(x-5)(x-2.5)}{(6-4)(6-5)(6-4.5)(6-6)(6-5)(6-2.5)} \\
					&+ y_4 \frac{(x-4)(x-5)(x-4.5)(x-6)(x-5)(x-2.5)}{(6-4)(6-5)(6-4.5)(6-6)(6-5)(6-2.5)} \\
					&+ y_5 \frac{(x-4)(x-5)(x-4.5)(x-6)(x-6)(x-2.5)}{(5-4)(5-5)(5-4.5)(5-6)(5-6)(5-2.5)} \\
					&+ y_6 \frac{(x-4)(x-5)(x-4.5)(x-6)(x-6)(x-5)}{(2.5-4)(2.5-5)(2.5-4.5)(2.5-6)(2.5-6)(2.5-5)} ]
	\end{split}
\end{equation}

\section{Ploting Result}
Ploting the resulting function A(x) in Matlab. The result can be observed in figure \ref{fig:lagrange}.

\begin{figure}
	\centering
		\includegraphics[width=0.8\textwidth]{Lagrange_interpolation.jpg}
	\caption{Lagrange Interpolation results for bar in problem 3.}
	\label{fig:lagrange}
\end{figure}

The interpolated x-y data we provied is matched perfectly by the Lagrange polynomial. However the interpolating profile may not provide the true profile of our bar when compared to the figure \ref{fig:weird_bar}, due to high oscilations in our resulting polynomial.

\end{document}
