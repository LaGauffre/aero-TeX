\documentclass[a4paper]{memoir}

% Preamble
\usepackage{calc}
\usepackage{color}
\usepackage{graphicx}
\usepackage{amsmath}
\usepackage{amssymb}

% load package with ``framed'' and ``numbered'' option.
\usepackage[framed,numbered,autolinebreaks,useliterate]{mcode}

%define title and other basic document info
%the title should reflect the style and give a foretaste of the document
%work on making a stylized title page — or title on a page as in ARTICLE class
\title{\huge \textbf{Homework Problems No.6}}
\author{Manuel A. Diaz \\ f99543083}
\date{April 23rd, 2013} % could use \today
%\publisher{Institute of Applied Mechanics, IAM}  %one day I’ll need this  
%\thanks{Special thanks to God for the ability to work}        %produces a footnote to the title

\definecolor{shadecolor}{gray}{0.9}
\definecolor{ared}{rgb}{.647,.129,.149}
\renewcommand\colorchapnum{\color{ared}}
\renewcommand\colorchaptitle{\color{ared}}
\chapterstyle{bringhurst}
%one of a number of chapter styles available…this one doesn’t use the ared color

% Begin Document Here
\begin{document}

% Build costumized Title Page
\thispagestyle{empty}
%\begin{minipage}{300pt}
\begin{center}{
\begin{shaded}
\hrule \vspace{30pt}
\hspace{30pt} \thetitle  \vspace{30pt}
\newline \theauthor \hspace{30pt} \thedate  \vspace{26pt}
\hrule
\end{shaded}
}
\end{center}
%\end{minipage}
\clearpage

%\frontmatter    %use if needed –page numbers as lower case roman numerals i, ii,…

%\mainmatter
%%other declarations
\pagestyle{Ruled}                    %one of a number of possible page styles
\midsloppy                             %to minimize overfull lines

%Layout the page
%%Try this manual golden ratio layout or…           default seems better for now
%\settypeblocksize{*}{\lxvchars}{1.618}
%\setulmargins{50pt}{*}{*}
%\setlrmargins{*}{*}{1.618}
%\setheaderspaces{*}{*}{1.618}
%\semiisopage[12]
%try this predefined layout — others predefined ones are options in MEMOIR…
%this one looked best but did not work

\checkandfixthelayout          %make the layout happen and provide details in log during build

\chapter{Homework Problems No. 6}
\section{Problem 1}
\subsection{Given}
With the same four mesh models given in the Lab Exercise 5, change the loading condition to moment loading as illustrated in figure \ref{fig:loaded_plate}.

\begin{figure}
	\centering
		\includegraphics[width=0.6\textwidth]{Plate_with_moment.png}
	\caption{Problem 1, Elastic Plate subject to a moment load at one end}
	\label{fig:loaded_plate}
\end{figure}

\subsection{Goal}
Perform a convergence study with the same four models. The four models used are depicted in figure \ref{fig:models},

\begin{figure}
	\centering
		\includegraphics[width=1.0\textwidth]{plate_models.png}
	\caption{Problem 1, Model used to evaluate new traction distribution}
	\label{fig:models}
\end{figure}

\subsection{Plan}
The force vector due to tractions $\textbf{f}^e_\Gamma$ is defined over the boundary of the a single element as,
\begin{equation}
	\textbf{f}^e_\Gamma = \int_{\Gamma^e_t} (N^{e})^{T} \bar{t} d\Gamma
\end{equation}
Note that $\Gamma^e_t$ is indicating us that we are integrating over a line for 2D elements.
We know from the Kroneker delta property of shape functions, $N^e_i$, vanish at other nodes that aren't node $i$. We know also that this functions are linear along the edge and can be written in terms of the edge parameters $\xi$. e.g.
\begin{align*}
	& N_2 = 1-\xi \\
	& N_3 = \xi
\end{align*}
We define $\xi$ also as our local parameter in the adimensional range of $0<\xi<1$, also observe that only the traction is acting in the x-direction and it's contribution to the force vector is only felt by nodes at the edge where the traction is located. \\
Four our case all contribution comes from the edge 2-3 of ever element at the free end of our plate models. (also our node numeration convention used in our code is counter clockwise). Given the above assumptions we can examine a single element force vector and conclude:
\begin{equation}
f_{\Gamma}^{(e)} = \int_{\Gamma^e_t} (N^{e})^{T} \bar{t} d\Gamma = 
	\begin{Bmatrix}
		\int N_1\bar{t}_x dl \\
		\int N_1\bar{t}_y dl \\
		\int N_2\bar{t}_x dl \\
		\int N_2\bar{t}_y dl \\
		\int N_3\bar{t}_x dl \\
		\int N_3\bar{t}_y dl \\
	\end{Bmatrix}
	=
	\begin{Bmatrix}
		0 \\
		0 \\
		\int N_2\bar{t}_x dl \\
		0 \\
		\int N_3\bar{t}_x dl \\
		0 \\
	\end{Bmatrix}
\end{equation}
\label{eq:traction}
where $dl = l_{23}d\xi$ is our differential line element. To define $\bar{t}_x(l)$ and $\bar{t}_y(l)$, we use the element shape function to interpolate the traction values to the nodes,
\begin{align*}
		& \bar{t}_x(l) = t_{x2}N_2 + t_{x3}N_3 \\
		& \bar{t}_y(l) = t_{y2}N_2 + t_{y3}N_3
\end{align*}
Then subtituting in to surviving terms of \ref{eq:traction} we get,
\begin{align}
	& \int N_2\bar{t}_x dl = \int_{-1}^{1} N_2(\xi)[t_{x2}N_2(\xi)+t_{x3}N_3(\xi)]l_{23}d\xi = \frac{l_{23}}{6}(2t_{x2}+t_{x3}) \\
	& \int N_3\bar{t}_x dl = \int_{-1}^{1} N_3(\xi)[t_{x2}N_2(\xi)+t_{x3}N_3(\xi)]l_{23}d\xi = \frac{l_{23}}{6}(t_{x2}+2t_{x3})
\end{align}

\subsection{Code}
To implement equation 1.3 and 1.4, we define the follow strategy:
\begin{itemize}
	\item Find the elements at the end of the plate,
	\item Identify and list for each element its corresponding the nodes \#2 and \#3,
	\item The nodes at the edge of every element, we locate its correspondent degrees of freedom in x (x-Dofs),
	\item Define a function for distribution of pressure at the egde,
	\item Evaluate for every element equations 1.3 and 1.4 with using the above information.
\end{itemize} 
We can now proceed to examing the this strategy in our MATLAB implementation,
\lstinputlisting[firstline=73, lastline=88]{problem2dTensileT3_HW6_problem_1.m}

\subsection{Solution}
The results and comparison with respect to the partition size \textit{h} are shown in figure \ref{fig:h_comparison}. More details on the code can be found in the Anexes.

\begin{figure}
	\centering
		\includegraphics[width=1.0\textwidth]{h_comparison.jpg}
	\caption{Problem 1, Results from the Plate with moment pressure load}
	\label{fig:h_comparison}
\end{figure}


\section{Problem 2}
\subsection{Given}
\begin{enumerate}
	\item  a triangular element with a linearly varing load, figure \ref{fig:t3_linear_load},
	\item  a sinusoidal load, figure \ref{fig:t3_sinusoidal_load}.
\end{enumerate}

\begin{figure}
	\centering
		\includegraphics[width=0.6\textwidth]{T3LinearLoad.png}
	\caption{Problem 2a, Elastic T3 element subject to linear varing load}
	\label{fig:t3_linear_load}
\end{figure}

\begin{figure}
	\centering
		\includegraphics[width=0.6\textwidth]{T3SineLoad.png}
	\caption{Problem 3b, Elastic T3 element subject to sinusoidal varing load}
	\label{fig:t3_sinusoidal_load}
\end{figure}

\subsection{Find}
Determine the element external force matrix, (e.i. the equivalent nodal forces) on the edge of the triangular elements shown below.

\subsection{Plan}
We start from our element formulation of vector forces for each of it's degress of freedom, 
\begin{align*}
& f_{\Gamma}^{(e)} = \int_{\Gamma^e_t} (N^{e})^{T} \bar{t} d\Gamma = 
	\begin{Bmatrix}
		\int N_1\bar{t}_x dl \\
		\int N_1\bar{t}_y dl \\
		\int N_2\bar{t}_x dl \\
		\int N_2\bar{t}_y dl \\
		\int N_3\bar{t}_x dl \\
		\int N_3\bar{t}_y dl \\
	\end{Bmatrix}
\end{align*}
For case 1, we observe that the distributed traction is acting only in x-direction of the edge 2-3, thus our problem simplifies to,
\begin{align*}
& f_{\Gamma}^{(e)} = \int_{\Gamma^e_t} (N^{e})^{T} \bar{t} d\Gamma = 
	\begin{Bmatrix}
		0 \\
		0 \\
		\int N_2\bar{t}_x dl \\
		0 \\
		\int N_3\bar{t}_x dl \\
		0 \\
	\end{Bmatrix}
\end{align*}

For case 2, we observe that the distributed traction is acting only in y-direction of the edge 1-3, and for this case problem simplifies to,
\begin{align*}
& f_{\Gamma}^{(e)} = \int_{\Gamma^e_t} (N^{e})^{T} \bar{t} d\Gamma = 
	\begin{Bmatrix}
		0 \\
		\int N_1\bar{t}_y dl \\
		0 \\
		0 \\
		0 \\
		\int N_3\bar{t}_y dl \\
	\end{Bmatrix}
\end{align*}
We also assume for both cases that we are working on a normalized section using $0<\xi<1$ and our shape functions are,
\begin{align*}
	& N^{(e)}_{2x} = 1-\xi \\
	& N^{(e)}_{3x} = \xi
\end{align*}
Thus we can still use the result from problem (1) to evaluate the integrals,
\begin{align}
	& \int N_i\bar{t}_x dl = \int_{-1}^{1} N_i(\xi)[t_{xi}N_i(\xi)+t_{xj}N_j(\xi)]l_{ij}d\xi = \frac{l_{ij}}{6}(2t_{xi}+t_{xj}) \\
	& \int N_j\bar{t}_x dl = \int_{-1}^{1} N_j(\xi)[t_{xi}N_i(\xi)+t_{xj}N_i(\xi)]l_{ij}d\xi = \frac{l_{ij}}{6}(t_{xi}+2t_{xj})
\end{align}

\subsection{Calculations}
To evaluate equations 1.4 and 1.5 in both cases we must first define the traction forces for each case,
For Case 1,
\begin{align*}
	& \bar{t} = 
	\begin{Bmatrix}
		\bar{t}_x1 \\
		\bar{t}_y1 \\
		\bar{t}_x2 \\
		\bar{t}_y2 \\
		\bar{t}_x3 \\
		\bar{t}_y3 \\
	\end{Bmatrix}
	= 
	\begin{Bmatrix}
		0 \\
		0 \\
		0 \\
		0 \\
		P_0 \\
		0 \\
	\end{Bmatrix}
\end{align*}

Substituting $\bar{t}_x2 = 0$ and $\bar{t}_x3 = P_0$ into equations 1.3 and 1.4, we then find
\begin{align} 
	& \int N_i\bar{t}_x dl = \frac{l_{23}}{6}(t_{x3}) = P_0\\
	& \int N_j\bar{t}_x dl = \frac{l_{23}}{6}(2t_{x3}) = \frac{Lt}{3}P_0
\end{align}

Therefore
\begin{align*}
& f_{\Gamma}^{(e)} = \int_{\Gamma^e_t} (N^{e})^{T} \bar{t} d\Gamma = 
	\begin{Bmatrix}
		0 \\
		P_0 \\
		0 \\
		0 \\
		0 \\
		2P_0 \\
	\end{Bmatrix} \frac{Lt}{6}
\end{align*}

For Case 2,
\begin{align*}
	& \bar{t} = 
	\begin{Bmatrix}
		\bar{t}_x1 \\
		\bar{t}_y1 \\
		\bar{t}_x2 \\
		\bar{t}_y2 \\
		\bar{t}_x3 \\
		\bar{t}_y3 \\
	\end{Bmatrix}
	= 
	\begin{Bmatrix}
		0 \\
		0 \\
		0 \\
		0 \\
		0 \\
		0 \\
	\end{Bmatrix}
\end{align*}
This implies:
\begin{align*}
	& f_{\Gamma}^{(e)} = \int_{\Gamma^e_t} (N^{e})^{T} \bar{t} d\Gamma = \vec{0}
\end{align*}
However this cannot be right. We then assume we are not goint to use the approximation using $\bar{t}$, then
\begin{align*}
	& N_1(x) = \frac{x_1-x}{L}
	& N_2(x) = \frac{x-x_2}{L}
\end{align*}
and using the given pressure distribution function $\bar{t}_y = P_0 sin(\frac{\pi*x}{L})$ we can compute,
\begin{align*}
	& \int N_1\bar{t}_y dl = \frac{L}{\pi}P_0 \\
	& \int N_3\bar{t}_y dl = \frac{L}{\pi}P_0
\end{align*}
Therefore
\begin{align*}
& f_{\Gamma}^{(e)} = \int_{\Gamma^e_t} (N^{e})^{T} \bar{t} d\Gamma = 
	\begin{Bmatrix}
		0 \\
		\frac{L}{\pi}P_0 \\
		0 \\
		0 \\
		0 \\
		\frac{L}{\pi}P_0 \\
	\end{Bmatrix}
\end{align*}

\subsection{Solution}

\section{Problem 3}
\subsection{Given}
Consider a 3-node triangular element as shown in the figure (unit in meter). Let $E = 30\times106$ $kN/m^2$ , and $\nu = 0.3$ . Imagine this is a long thickness ($t$) triangular concrete prism so we can idealize it as a plane strain problem. The only loading for the prism is the dead load from its self-weight with the gravity $g = 9.8$ $m/s^2$ pointing downward along the y-direction. Assume that the density of concrete is 2400 $kg/m^3$. See figure \ref{fig:T3underGravity}.

\begin{figure}
	\centering
		\includegraphics[width=0.6\textwidth]{T3.png}
	\caption{Problem 3, Elastic T3 element subject to gravity load.}
	\label{fig:T3underGravity}
\end{figure}

\subsection{Find}
\begin{enumerate}
 \item Compute the shape functions $\textbf{N}^e$,
 \item Compute the strain-displacement matrix $\textbf{B}^e$
 \item Find the equivalent nodal forces of the element $\vec{f}^e$
 \item Evaluate the element stiffness matrix $\textbf{K}^e$
\end{enumerate}

\subsection{Calculations and Results}
1. Using the coordinates of the element in figure \ref{fig:T3underGravity} we can compute the element's shape functions,
Compute first, compute polynomials arrays,
\begin{align*}
	\textbf{M} = \left(
\begin{array}{ccc}
 1 & x_1 & y_1 \\
 1 & x_2 & y_2 \\
 1 & x_3 & y_3 
\end{array}
\right)
=
\left(
\begin{array}{ccc}
 1 & 2 & 3 \\
 1 & 3 & 1 \\
 1 & 9/4 & 13/4
\end{array}
\right)
\end{align*}

Compute Element Area,
\begin{align*}
	& A_{section} = \frac{1}{2}Det[\textbf{M}] = (x_2y_3-x_3y_2)+(x_3y_1-x_1y_3)+(x_1y_2-x_2y_1)
\end{align*}

Using the following relations
\begin{align*}
	& N_1(x,y) = \frac{1}{2A_{section}} ((x_2y_3-x_3y_2)+(y_2-y_3)+(x_3-x_2)) \\
	& N_2(x,y) = \frac{1}{2A_{section}} ((x_3y_1-x_1y_3)+(y_3-y_1)+(x_1-x_3)) \\
	& N_3(x,y) = \frac{1}{2A_{section}} ((x_1y_2-x_2y_1)+(y_1-y_2)+(x_2-x_1))
\end{align*}

\begin{align*}
	& N_1(x,y) = 10 - 3 x - y \\
	& N_2(x,y) = \frac{1}{3}+\frac{x}{3}-\frac{y}{3} \\
	& N_3(x,y) = -(28/3) + (8 x)/3 + (4 y)/3
\end{align*}

And formulate the shape functions array,
\begin{align*}
	&\textbf{N}(x,y) = 
	\left(
	\begin{array}{cccccc}
	 N_1(x,y) & 0 			& N_2(x,y) & 0 			& N_3(x,y) 	& 0 		 \\
	 0 			& N_1(x,y) 	& 0 		 	& N_2(xy) & 0 			& N_3(x,y) \\
	\end{array}
	\right)
\end{align*}

2. We now proceed to compute the derivate of the shape function arrays,
\begin{align*}
	&\textbf{B}(x,y) = 
	\left(
	\begin{array}{cccccc}
	 B_1(x,y) & 0 			& B_2(x,y) & 0 			& B_3(x,y) 	& 0 		 \\
	 0 			& B_1(x,y) 	& 0 		 	& B_2(xy) & 0 			& B_3(x,y) \\
	\end{array}
	\right)
\end{align*}

Writting it explicitly, it turns out to be:
\begin{align*}
	&\textbf{B}(x,y) = 
	\left(
\begin{array}{cccccc}
 -3 & 0 & \frac{1}{3} & 0 & \frac{8}{3} & 0 \\
 0 & -1 & 0 & -\frac{1}{3} & 0 & \frac{4}{3} \\
 -1 & -3 & -\frac{1}{3} & \frac{1}{3} & \frac{4}{3} & \frac{8}{3}
\end{array}
\right)
\end{align*}

3. Define the Strain-displacement relation matrix for our case,
\begin{align*}
	&\textbf{D} = \frac{E}{(1+\nu)(1-2\nu)}
	\left(
	\begin{array}{ccc}
	 0.7 & 0.3 & 0 \\
	 0.3 & 0.7 & 0 \\
	 0 & 0 & 0.2
	\end{array}
	\right)
	=
	\left(
	\begin{array}{ccc}
	 4.03846\times 10^7 & 1.73077\times 10^7 & 0. \\
	 1.73077\times 10^7 & 4.03846\times 10^7 & 0. \\
	 0. & 0. & 1.15385\times 10^7
	\end{array}
	\right)
\end{align*}

For our case a section of along concrete beam is assumed to be under the influence of gravity force and we wish to model it as a plate, to do this we need to know the force that the weigt of this section will be given as an avergare of traction.\\
Assume:
\begin{align*}
	& g = 9.80 \;\; m/s^2 \\
	& \rho_{concrete} = 2400 \;\; kg/m^3 \\
	& thickness = 1 \;\; m \\
	& A_{section} = 3/8 \;\; m^2 \\
	& V_{section} = thickness*A_{section} \\
	& W_{section} = V_{section}*\rho_{concrete}*g \\
	& \bar{b}_y = W_{section}/Area = \;\; 23520 N/m^2
\end{align*}

Subtituting in our force vector and assuming that it will be distributed evenly among the nodes we get,
\begin{equation}
f_{\Gamma}^{(e)} = 
	\frac{Area}{3}
	\begin{Bmatrix}
		0 \\
		\bar{b}_y \\
		0 \\
		\bar{b}_y \\
		0 \\
		\bar{b}_y \\
	\end{Bmatrix} 
	= 
	\begin{Bmatrix}
		0 \\
		2940 \\
		0 \\
		2940 \\
		0 \\
		2940 \\
	\end{Bmatrix} \;\; [N]
\end{equation}

4. Finally we can know proceed to compute the stiffness matrix for the actual element
\begin{align*}
	K = t A_{section}[\textbf{B}]^{T} [\textbf{D}] [\textbf{B}]
\end{align*}

The matrix is too big, to be posted here. This result will be anexed in our calculations performed in Mathematica.

\end{document}


