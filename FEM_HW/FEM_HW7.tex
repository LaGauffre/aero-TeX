\documentclass[a4paper]{memoir}

% Preamble
\usepackage{calc}
\usepackage{color}
\usepackage{graphicx}
\usepackage{amsmath}
\usepackage{amssymb}

% load package with ``framed'' and ``numbered'' option.
\usepackage[framed,numbered,autolinebreaks,useliterate]{mcode}

%define title and other basic document info
%the title should reflect the style and give a foretaste of the document
%work on making a stylized title page — or title on a page as in ARTICLE class
\title{\huge \textbf{Homework Problems No.7}}
\author{Manuel A. Diaz \\ f99543083}
\date{May 7th, 2013} % could use \today
%\publisher{Institute of Applied Mechanics, IAM}  %one day I’ll need this  
%\thanks{Special thanks to God for the ability to work}        %produces a footnote to the title

\definecolor{shadecolor}{gray}{0.9}
\definecolor{ared}{rgb}{.647,.129,.149}
\renewcommand\colorchapnum{\color{ared}}
\renewcommand\colorchaptitle{\color{ared}}
\chapterstyle{bringhurst}
%one of a number of chapter styles available…this one doesn’t use the ared color

% Begin Document Here
\begin{document}

% Build costumized Title Page
\thispagestyle{empty}
%\begin{minipage}{300pt}
\begin{center}{
\begin{shaded}
\hrule \vspace{30pt}
\hspace{30pt} \thetitle  \vspace{30pt}
\newline \theauthor \hspace{30pt} \thedate  \vspace{26pt}
\hrule
\end{shaded}
}
\end{center}
%\end{minipage}
\clearpage

%\frontmatter    %use if needed –page numbers as lower case roman numerals i, ii,…

%\mainmatter
%%other declarations
\pagestyle{Ruled}                    %one of a number of possible page styles
\midsloppy                             %to minimize overfull lines

%Layout the page
%%Try this manual golden ratio layout or…           default seems better for now
%\settypeblocksize{*}{\lxvchars}{1.618}
%\setulmargins{50pt}{*}{*}
%\setlrmargins{*}{*}{1.618}
%\setheaderspaces{*}{*}{1.618}
%\semiisopage[12]
%try this predefined layout — others predefined ones are options in MEMOIR…
%this one looked best but did not work

\checkandfixthelayout          %make the layout happen and provide details in log during build

\chapter{Homework Problems No. 6}
\section{Problem 1}
\subsection{Given}
Consider the differential equation discussed in the lecture,

\begin{align}
  &-\frac{\partial^2 u}{\partial x^2} = 2 & & \text{for } 0<x<1 	& \text{with } u(0)=u(1)=0
\end{align}

\subsection{Find}
Evaluate and verify finite element solution, the errors of $L_2$ norm and energy norm for $N=3$.

\subsection{Plan}
The $L_2$ norm error in a finite element solution is defined as,

\begin{align}
		& \left\|e \right\|_{L_2} = \left\|u(x)-u_h(x) \right\|_{L_2} = \sqrt{ \int_{x_1}^{x_2} (u(x)-u_h(x))^2 dx }&
\end{align}
in which $u(x)$ is the exact solution and $u_h(x)$ is finite element approximation. \\
And the error in Energy is similarly defined as,
\begin{align}
		&\left\|e \right\|_{en} = \left\|\epsilon(x)-\epsilon_h(x) \right\|_{en} = \sqrt{ \int_{x_1}^{x_2} (\epsilon(x)-\epsilon_h(x))^2 dx }&
\end{align}
in which $\epsilon(x) = \frac{\partial u}{\partial x} $ is the exact solution of strain and $\epsilon_h$ is finite approximation. \\

From lecture notes we know: Exact solution of displacements, $u(x)$, and the strains, $\epsilon(x)$, defined as
\begin{align*}
	& u(x) = x(1-x)& \\
	& \epsilon(x) = \frac{du}{dx} = 1-2x& 
\end{align*}
And the approximate displacement functions for $N=3$ elements,
\begin{align*}
	& u^{(1)}_h(x) = 2h^2 (x/h) \\
	& u^{(2)}_h(x) = 2h^2 (2 - x/h) + 2h^2 (x/h - 1) \\
	& u^{(3)}_h(x) = 2h^2 (3 - x/h)
\end{align*}

\subsection{Calculations}
With the all the above information, and assuming $h=1/3$ we proceed to evalute the normed errors in a element by element strategy, \\
For $L_2$ norm error we have,
\begin{align*}
	& \left\|e \right\|_{L_2}^{(1)} = \int_{0}^{\frac{h}{3}} (-2 h x+(1-x) x) dx \\
	& \left\|e \right\|_{L_2}^{(2)} = \int_{0}^{\frac{h}{3}} (1-x) x-2 h^2 \left(2-\frac{x}{h}\right)-2 h^2 \left(-1+\frac{x}{h}\right) dx \\
	& \left\|e \right\|_{L_2}^{(3)} = \int_{0}^{\frac{h}{3}} (1 - x) x - 2 h^2 (3 - x/h) dx \\
	& \left\|e \right\|_{L_2} = \left\|e \right\|_{L_2}^{(1)} + \left\|e \right\|_{L_2}^{(2)} + \left\|e \right\|_{L_2}^{(3)} = \frac{1}{9\sqrt{30}} = 0.02028
\end{align*}

For error  in \textit{Energy} we have,
\begin{align*}
	& \left\|e \right\|_{en}^{(1)} = \int_{0}^{\frac{h}{3}} (1-2 h-2 x) dx \\
	& \left\|e \right\|_{en}^{(2)} = \int_{0}^{\frac{h}{3}} (1-2 x) dx \\
	& \left\|e \right\|_{en}^{(3)} = \int_{0}^{\frac{h}{3}} (1+2 h-2 x) dx \\
	& \left\|e \right\|_{en} = \left\|e \right\|_{en}^{(1)} + \left\|e \right\|_{en}^{(2)} + \left\|e \right\|_{en}^{(3)} = \frac{1}{3\sqrt{3}} = 0.1925
\end{align*} 

\subsection{Solution}
Tabulating our results as presented in the lecture notes we get:
\begin{center}
	\begin{tabular}{l|l||l|l||l|l}
		\hline
		\textbf{h} & $log_{10}\textbf{h}$ &	$\left\|\textbf{e} \right\|_{L_2}$ & $log_{10} \left\|\textbf{e} \right\|_{L_2}$ & $\left\|\textbf{e} \right\|_{en}$ & $log_{10} \left\|\textbf{e} \right\|_{en}$  \\ 
		\hline \hline
		1/3 & -0.477 & 0.0208 & -1.693 & 0.1925 & -0.7157 \\
		\hline
	\end{tabular}
\end{center}

\section{Problem 2}
\subsection{Given}
Consider a bar of length $2l$ , cross-sectional area $A$ and Young’s modulus $E$. The bar is fixed at $x=0$, subject to a linear body force $cx$ and an applied traction $\bar{t} = cl^2/A$ at $x=2l$ as shown in the figure \ref{fig:Loaded_beam}. \\

Let $E = 104$ $Nm^{−2}$ , $A = 1$ $m^2$ , $c = 1$ $Nm^{−2}$ and $l = 1 m$ . \\

The strong form is given as,
\begin{align}
  & \frac{\partial }{\partial x}\left( AE \frac{\partial u}{\partial x} \right) +cx =0,& &\text{for } 0<x<2l& \\ 
  & u|_{x=0} = u(0) = 0 \nonumber \\
  & \bar{t} = \left( E \frac{\partial u}{\partial x}n \right)_{x=2l} = \frac{cl^2}{A} \nonumber
\end{align}

\begin{figure}
	\centering
		\includegraphics[width=0.6\textwidth]{LinearBeam.jpg}
	\caption{Problem 2, Elastic 1D cantilever subject to a distributed load along its length and point load at it's free end}
	\label{fig:Loaded_beam}
\end{figure}

\subsection{Find}
Discretize the domain and solve uniformly with 1-, 2- and 4- of \textit{linear} and \textit{quadratic} elements by extending our MATLAB program. Tabulate your results as indicated in lecture notes.

\subsection{Plan}
We extend the MATLAB code for linear and quadratic isoparametric implementations to solve the beam in figure \ref{fig:Loaded_beam} using the given conditions. \\

To compute the normed errors we compute our exact displacement and strain equations, e.i.
\begin{align*}
	& u(x) = \frac{c}{AE} \left( xl^2 -\frac{x^3}{6} \right) \\
	& \epsilon(x) = \frac{du}{dx} = \frac{c}{AE} \left( l^2 -\frac{x^2}{2} \right) 
\end{align*}

Now we proceed to compute the aproximate solutions in every element of our fomulation by usign the following relations,
\begin{align*}
	& u_h^{(e)}(x) =  N^{(e)}(x) d^{(e)}\\
	& \epsilon_h^{(e)}(x) = B^{(e)}(x) d^{(e)}
\end{align*}
Form here, our fomulation of the $L_2$ norm displacement error and Energy error are as follows,
\begin{align*}
	& \left\|e \right\|_{L_2} = \sqrt{\sum_{e=1}^{n} \int_{x_1^{(e)}}^{x_2^{(e)}}\left(u^{(e)}(x)-u_h^{(e)}(x)\right)^2 dx} \\
	& \left\|e \right\|_{en} = \sqrt{\sum_{e=1}^{n} \int_{x_1^{(e)}}^{x_2^{(e)}}\left(\epsilon^{(e)}(x)-\epsilon_h^{(e)}(x)\right)^2 dx}
\end{align*}
We wish to perform those integrations using gauss integration. We can take advante of the polynomial degree of the exact solution, 3rd order, to choose the requeried points and make sure we can integrate with enough accuaracy our polynomial of 6th order. \\
Notice that if we choose 3 gauss poinst, $ngp = 2(3)-1 = 5$, our quadrature won't be enought, therefore taking 4 gauss points, $ngp = 2(4)-1 = 9$ will ensure enough accuracy on our method. \\

We now proceed to change integration range using the relation $x=\frac{b+a}{2}+\frac{b-a}{2}\xi$, to a normalized range of $-1<\xi<1$,
\begin{align*}
	& \left\|e \right\|_{L_2} = \sqrt{\sum_{e=1}^{n} \int_{-1}^{1}\left(u^{(e)}(\xi)-u_h^{(e)}(\xi)\right)^2 J d\xi} \\
	& \left\|e \right\|_{en} = \sqrt{\sum_{e=1}^{n} \int_{-1}^{1}\left(\epsilon^{(e)}(\xi)-\epsilon_h^{(e)}(\xi)\right)^2 J d\xi}
\end{align*}
here $u^{(e)}(\xi)$ and $\epsilon^{(e)}(\xi)$ are the exact solutions evaluated in the natural range for every element and J is the Jacobian defined as $J = \frac{x_2^{(e)}-x_1^{(e)}}{2} = \frac{l^{(e)}}{2}$. \\

We also recognize that with the change of integration range, our definitions of shape and Natural derivates function will change sligthly,
\begin{align*}
	& u_h^{(e)}(\xi) =  N^{(e)}(\xi) d^{(e)}\\
	& \epsilon_h^{(e)}(\xi) = \frac{1}{J} \frac{\partial N^{(e)}(\xi)}{\partial \xi} d^{(e)}
\end{align*}

Our discrete normed errors then become,
\begin{align}
	& \left\|e \right\|_{L_2} = \sqrt{\sum_{e=1}^{n} \frac{l^{(e)}}{2}\sum_i^4 W_i\left(u^{(e)}(\xi_i)-N^{(e)}(\xi) d^{(e)}\right)^2 } \\
	& \left\|e \right\|_{en} = \sqrt{\sum_{e=1}^{n} \frac{l^{(e)}}{2}\sum_i^4 W_i\left(\epsilon^{(e)}(\xi_i)-\frac{1}{J} \frac{\partial N^{(e)}(\xi)}{\partial \xi} d^{(e)}\right)^2 }
\end{align}

We use these results to compute the normed errors for different values of \textit{h}.

\subsection{Code}
In the Linear Isoparametric formulation, the implementation of equations 1.5 and 1.6 goes as follows,
\lstinputlisting[firstline=124, lastline=148]{linear_disp.m}

Similarly, in the Quadratic Isoparametric formulation, the implementation of equations 1.5 and 1.6 is,
\lstinputlisting[firstline=128, lastline=152]{quad_disp.m}

\subsection{Solution}
Results for the linear and quadratic isoparametric formulations can be observed in figures \ref{fig:LineaIso} and \ref{fig:QuadIso}, repectively. More Details of these implementation will be provied in the anexus. \\

Using this models we compute and evaluate the errors in the displacements and computation of stress. Tabulating our results as presented in the lecture notes we get: \\

Linear Case
\begin{center}
	\begin{tabular}{l|l||l|l||l|l}
		\hline
		\textbf{h} & $log_{10}\textbf{h}$ &	$\left\|\textbf{e} \right\|_{L_2}$ & $log_{10} \left\|\textbf{e} \right\|_{L_2}$ 
		& $\left\|\textbf{e} \right\|_{en}$ & $log_{10} \left\|\textbf{e} \right\|_{en}$  \\ 
		\hline \hline
		2   & 0.3010 	& 5.2048e-5 & -4.2836 & 8.4327e-5 & -4.0740 \\
		1   & 0 			& 1.4457e-5 & -4.8399 & 4.5947e-5 & -4.3377 \\
		0.5 & -0.3013 & 3.6989e-5 & -5.4319 & 2.3422e-5 & -4.6304 \\
		\hline
	\end{tabular}
\end{center}

Quadratic Case
\begin{center}
	\begin{tabular}{l|l||l|l||l|l}
		\hline
		\textbf{h} & $log_{10}\textbf{h}$ &	$\left\|\textbf{e} \right\|_{L_2}$ & $log_{10} \left\|\textbf{e} \right\|_{L_2}$ 
		& $\left\|\textbf{e} \right\|_{en}$ & $log_{10} \left\|\textbf{e} \right\|_{en}$  \\ 
		\hline \hline
		2		& 0.3010 	& 6.5060e-6 & -5.1867 & 2.1082e-5 & -4.6760 \\
		1		& 0 			& 8.1325e-7 & -6.0898 & 5.2705e-6 & -5.2781 \\
		0.5 & -0.3010 & 1.0166e-7 & -6.9928 & 1.3176e-6 & -5.8802	\\
		\hline
	\end{tabular}
\end{center}

Our results can be appreciated graphically in figure \ref{fig:convergence}
\begin{figure}
	\centering
		\includegraphics[width=0.8\textwidth]{Convergence.jpg}
	\caption{Problem 2, Convergence fo Plots of Linear and Quadratic Isoparametric formulations}
	\label{fig:convergence}
\end{figure}

\begin{figure}
	\centering
		\includegraphics[width=0.8\textwidth]{Linear.jpg}
	\caption{Problem 2a, Solution of the Linear Formulation}
	\label{fig:LineaIso}
\end{figure}

\begin{figure}
	\centering
		\includegraphics[width=0.8\textwidth]{Quadratic.jpg}
	\caption{Problem 2b, Solution fo the Quadratic Formulation}
	\label{fig:QuadIso}
\end{figure}

\end{document}