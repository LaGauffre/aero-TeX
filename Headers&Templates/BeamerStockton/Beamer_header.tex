% ***********************************************************
% ********* PHYSICS HEADER FOR BEAMER PRESENTATIONS *********
% ***********************************************************
% Version 2.1 
% Orignal author: Chris Clark, UCLA, http://www.dfcd.net/articles/latex
% Modified by: Manuel Diaz, NTU 2013, https://github.com/wme7

\documentclass[xcolor=svgnames, 9pt]{beamer} 
\usetheme{Stockton}
\usepackage{amsmath} % AMS Math Package
\usepackage{amsthm} % Theorem Formatting
\usepackage{amssymb}	% Math symbols such as \mathbb
\usepackage{graphicx} % Allows for eps images
\usepackage{epsfig} %for figures
\usepackage{xcolor} %for color
\usepackage{subcaption} %for figures arrays % Sets margins and page size
% Add water mark in your presentation
\definecolor{hughesblue}{rgb}{.9,.9,1} %A blue I like to use for highlighting, matches Hughes Hallet's book
%\logo{\includegraphics[height=2cm]{NTU_logo_watermark.jpg}} % comment out this line if you do not have the pacific-seal file}

% \DeclareMathOperator{\Sample}{Sample}
\let\vaccent=\v % rename builtin command \v{} to \vaccent{}
\renewcommand{\v}[1]{\ensuremath{\mathbf{#1}}} % for vectors
\newcommand{\gv}[1]{\ensuremath{\mbox{\boldmath$ #1 $}}} 

% for vectors of Greek letters
\newcommand{\uv}[1]{\ensuremath{\mathbf{\hat{#1}}}} % for unit vector
\newcommand{\abs}[1]{\left| #1 \right|} % for absolute value
\newcommand{\avg}[1]{\left< #1 \right>} % for average
\let\underdot=\d % rename builtin command \d{} to \underdot{}
\renewcommand{\d}[2]{\frac{d #1}{d #2}} % for derivatives
\newcommand{\dd}[2]{\frac{d^2 #1}{d #2^2}} % for double derivatives
\newcommand{\pd}[2]{\frac{\partial #1}{\partial #2}} 

% for partial derivatives
\newcommand{\pdd}[2]{\frac{\partial^2 #1}{\partial #2^2}} 

% for double partial derivatives
\newcommand{\pdc}[3]{\left( \frac{\partial #1}{\partial #2}
 \right)_{#3}} % for thermodynamic partial derivatives
\newcommand{\ket}[1]{\left| #1 \right>} % for Dirac bras
\newcommand{\bra}[1]{\left< #1 \right|} % for Dirac kets
\newcommand{\braket}[2]{\left< #1 \vphantom{#2} \right|
 \left. #2 \vphantom{#1} \right>} % for Dirac brackets
\newcommand{\matrixel}[3]{\left< #1 \vphantom{#2#3} \right|
 #2 \left| #3 \vphantom{#1#2} \right>} % for Dirac matrix elements
\newcommand{\grad}[1]{\gv{\nabla} #1} % for gradient
\let\divsymb=\div % rename builtin command \div to \divsymb
\renewcommand{\div}[1]{\gv{\nabla} \cdot #1} % for divergence
\newcommand{\curl}[1]{\gv{\nabla} \times #1} % for curl
\let\baraccent=\= % rename builtin command \= to \baraccent
\renewcommand{\=}[1]{\stackrel{#1}{=}} % for putting numbers above =
\newtheorem{prop}{Proposition}
\newtheorem{thm}{Theorem}[section]
\newtheorem{lem}[thm]{Lemma}
\theoremstyle{definition}
\newtheorem{dfn}{Definition}
\theoremstyle{remark}
\newtheorem*{rmk}{Remark}

% Modify '\vec{}' command to generate bold vector characters
\let\oldhat\hat
\renewcommand{\vec}[1]{\mathbf{#1}}

% At the Beginning of every Section
\AtBeginSection[]
{
  \begin{frame} \frametitle{Outline}
    \tableofcontents[currentsection]
  \end{frame}
}

% ***********************************************************
% ********************** END HEADER **************** MD2013 *
% ***********************************************************

% General Guide lines:
% Add \input{header.tex} to the first line of your document, e.g.:

% ***********************************************************
%\input{header.tex}
%\title{}
%\author{}
%
%\begin{document}
%\maketitle
%
%\end{document}
% ***********************************************************

% The LaTeX for Physicists Header has the following features:

% Sets font size to 9pt
% Includes commonly needed packages for beamer presentation 
% \v{ } makes bold vectors (\v is redefined to \vaccent)
% \uv{ } makes bold unit vectors with hats
% \gv{ } makes bold vectors of greek letters
% \abs{ } makes the absolute value symbol
% \avg{ } makes the angled average symbol
% \d{ }{ } makes derivatives (\d is redefined to \underdot)
% \dd{ }{ } makes double derivatives
% \pd{ }{ } makes partial derivatives
% \pdd{ }{ } makes double partial derivatives
% \pdc{ }{ }{ } makes thermodynamics partial derivatives
% \ket{ } makes Dirac kets
% \bra{ } makes Dirac bras
% \braket{ }{ } makes Dirac brackets
% \matrixel{ }{ }{ } makes Dirac matrix elements
% \grad{ } makes a gradient operator
% \div{ } makes a divergence operator (\div is redefined to \divsymb)
% \curl{ } makes a curl operator
% \={ } makes numbers appear over equal signs (\= is redefined to \baraccent)

% General LaTeX tips:

% Use "$ ... $" for inline equations
% Use "\[ ... \]" for equations on their own line
% Use "\begin{center} ... \end{center}" to center something
% Use "\includegraphics[width=?cm]{filename.eps}" for images - must compile to dvi then use dvipdfm from a batch file
% Use "\begin{multicols}{2} ... \end{multicols}" for two columns
% Use "\begin{enumerate} \item ... \end{enumerate}" for parts of physics exercises
% Use "\section*{ }" for sections without numbering
% Use "\begin{cases} ... \end{cases}" for piecewise functions
% Use "\mathcal{ }" for a caligraphic font
% Use "\mathbb{ }" for a blackboard bold font