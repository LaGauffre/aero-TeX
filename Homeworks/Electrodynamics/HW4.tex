% ***********************************************************
% ******************* PHYSICS BEAMER HEADER *****************
% ***********************************************************
% Version 2.2 
% Author: Chris Clark, UCLA, http://www.dfcd.net/
% Modifications by Manuel Diaz, NTU 2013, https://github.com/wme7

\documentclass[11pt]{article} 
\usepackage{amsmath} % AMS Math Package
\usepackage{amsthm} % Theorem Formatting
\usepackage{amssymb}	% Math symbols such as \mathbb
\usepackage{graphicx} % Allows for eps images
\usepackage{multicol} % Allows for multiple columns
\usepackage[dvips,letterpaper,margin=0.75in,bottom=0.5in]{geometry}
 % Sets margins and page size
\pagestyle{empty} % Removes page numbers
\makeatletter % Need for anything that contains an @ command 
\renewcommand{\maketitle} % Redefine maketitle to conserve space
{ \begingroup \vskip 10pt \begin{center} \large {\bf \@title}
	\vskip 10pt \large \@author \hskip 20pt \@date \end{center}
  \vskip 10pt \endgroup \setcounter{footnote}{0} }
\makeatother % End of region containing @ commands
\renewcommand{\labelenumi}{(\alph{enumi})} % Use letters for enumerate
% \DeclareMathOperator{\Sample}{Sample}
\let\vaccent=\v % rename builtin command \v{} to \vaccent{}
\renewcommand{\v}[1]{\ensuremath{\mathbf{#1}}} % for vectors
\newcommand{\gv}[1]{\ensuremath{\mbox{\boldmath$ #1 $}}} 
% for vectors of Greek letters
\newcommand{\uv}[1]{\ensuremath{\mathbf{\hat{#1}}}} % for unit vector
\newcommand{\abs}[1]{\left| #1 \right|} % for absolute value
\newcommand{\avg}[1]{\left< #1 \right>} % for average
\let\underdot=\d % rename builtin command \d{} to \underdot{}
\renewcommand{\d}[2]{\frac{d #1}{d #2}} % for derivatives
\newcommand{\dd}[2]{\frac{d^2 #1}{d #2^2}} % for double derivatives
\newcommand{\pd}[2]{\frac{\partial #1}{\partial #2}} 
% for partial derivatives
\newcommand{\pdd}[2]{\frac{\partial^2 #1}{\partial #2^2}} 
% for double partial derivatives
\newcommand{\pdc}[3]{\left( \frac{\partial #1}{\partial #2}
 \right)_{#3}} % for thermodynamic partial derivatives
\newcommand{\ket}[1]{\left| #1 \right>} % for Dirac bras
\newcommand{\bra}[1]{\left< #1 \right|} % for Dirac kets
\newcommand{\braket}[2]{\left< #1 \vphantom{#2} \right|
 \left. #2 \vphantom{#1} \right>} % for Dirac brackets
\newcommand{\matrixel}[3]{\left< #1 \vphantom{#2#3} \right|
 #2 \left| #3 \vphantom{#1#2} \right>} % for Dirac matrix elements
\newcommand{\grad}[1]{\gv{\nabla} #1} % for gradient
\let\divsymb=\div % rename builtin command \div to \divsymb
\renewcommand{\div}[1]{\gv{\nabla} \cdot #1} % for divergence
\newcommand{\curl}[1]{\gv{\nabla} \times #1} % for curl
\let\baraccent=\= % rename builtin command \= to \baraccent
\renewcommand{\=}[1]{\stackrel{#1}{=}} % for putting numbers above =
\newtheorem{prop}{Proposition}
\newtheorem{thm}{Theorem}[section]
\newtheorem{lem}[thm]{Lemma}
\theoremstyle{definition}
\newtheorem{dfn}{Definition}
\theoremstyle{remark}
\newtheorem*{rmk}{Remark}

% ***********************************************************
% ********************** END HEADER *************************
% ***********************************************************

% Document Begin's
%\begin{document}

\title{Electrodynamics, Spring 2014 \\ Solution of Homework 4}
\author{by Manuel Diaz\\
			f99543083@ntu.edu.tw\\}
\date{\today}

\begin{document}
\maketitle

\section{Basic Definitions}
1. Let us recall from lectures notes the following figure,
\begin{figure}[h]
	\centering
	\includegraphics[width=0.4\textwidth]{images/electrostatics_summary.png} 
	\caption{Electrostacis Summary}
	\label{fig:electrostatics_summary}
\end{figure}

Observe that by starting form the Potential definition of any problem it is always easier to know it's charge density and the electrical field.\\
2. From Chapter 3, the dipole potential field, $V_{dip}$, and the dipole moment, $\vec{p}$, are defined as
\begin{align}
	& V_{dip}(\vec{r}) = \frac{1}{4\pi \epsilon_0}\frac{\vec{p}\cdot\hat{r}}{r^2},& &\vec{p} = \int_{\tau}\vec{P}d\tau&
\end{align}
Where $\vec{P}$ is the polarization per unit volume.

% Problem 1
\section{Problem 4.2}
Find the electric field outside a uniformaly polarized sphere or radious $R$.\\
\begin{figure}
	\centering
	\includegraphics[width=0.2\textwidth]{images/example4_2.png} 
	\caption{Example 4.2}
	\label{fig:example_4.2}
\end{figure}

Solution: We choose the z axis to coincide with the direction of Polarization (Fig. \ref{fig:example_4.2}). The volume bound charge density $\rho_b$ is zero, since $\vec{P}$ is uniform, but
\begin{equation}
	\sigma_b = \vec{P}\cdot\hat{n} = P\cos\theta
\end{equation}
where theta is the usual spherical coordinate. From example 4.2 the potential field produced by this density, $P\cos\theta$, over the surface of a sphere is given by
\begin{equation}
	V(r,\theta) = \begin{cases}
	\frac{P}{3\epsilon_0}r\cos\theta, & 0\leq r\leq R \\
	\frac{P}{3\epsilon_0}\frac{R^3}{r^2}\cos\theta, & r \geq R
	\end{cases}
\end{equation}
To compute the electric field, recall in figure \ref{fig:electrostatics_summary} and let us use the gradient relation, $E = -\grad{V}$. 
The gradient of $V$ in spherical coordinates is given by:
\begin{equation}
	\grad{t} = \pd{t}{r}\hat{r} + \frac{1}{r}\pd{t}{\theta}\hat{\theta}+\frac{1}{r\sin\theta} \pd{t}{\phi}\hat{\phi}
\end{equation}
Evaluating our potential function, $V$, into (4) will yield,
\begin{equation}
\vec{E} = \begin{cases}
	\left[ -\frac{P\cos\theta}{3\epsilon_0}\hat{r},\frac{P\sin\theta}{3\epsilon_0}\hat{\theta},0\hat{\phi} \right] , & 0\leq r\leq R \\
	\left[ \frac{2P\,R^3\cos\theta}{3r^3\epsilon_0}\hat{r},\frac{P\,R^3\sin\theta}{3r^3\epsilon_0}\hat{\theta},0\hat{\phi} \right] & r \geq R
\end{cases}
\end{equation}
Assuming $R=1$, $P=1$ and $\epsilon_0=1$. The electric field computed and ploted in Mathematica. The result is shown in figure \ref{fig:Efield_4.2}
\begin{figure}
	\centering
	\includegraphics[width=0.33\textwidth]{images/plotExample4_2.png} 
	\caption{Electric Field for example 4.2}
	\label{fig:Efield_4.2}
\end{figure}

% Problem 2
\section{Problem 4.7a}
Find the electric field outside a homogenours linear dielectric sphere place in an uniform electric field $\vec{E}_0$, similarly as shown in figure \ref{fig:example_4.7}.\\
\begin{figure}[h]
	\centering
	\includegraphics[width=0.3\textwidth]{images/example4_7.png} 
	\caption{Example 4.7}
	\label{fig:example_4.7}
\end{figure}

Solution: From Example 4.7 we know that the potential inside and outside the sphere is given by,
\begin{equation}
	V(r,\theta) = \begin{cases}
		-\frac{3E_0}{2+\epsilon_r}r\cos\theta, & 0\leq r\leq R \\
		-E_0 r\cos\theta+\frac{E_0R^3}{r^2}\frac{\epsilon_r-1}{\epsilon_r+2}\cos\theta, & r\geq R
	\end{cases}
\end{equation}
Once again, let us use the relation $E = -\grad{V}$ and expresion (4). The Electric field are,
\begin{equation}
	\vec{E} = \begin{cases}
		\left[\frac{3E_0\cos\theta}{2+\epsilon_0}\hat{r},
					-\frac{3E_0\sin\theta}{2+\epsilon_0}\hat{\theta},
					0\hat{\phi} \right], & 0\leq r\leq R \\
		\left[(E_0\cos\theta+2\frac{\epsilon_r-1}{\epsilon_r+2}\frac{R^3}{r^3}E_0\cos\theta)\hat{r},
					(-E_0\sin\theta+\frac{\epsilon_r-1}{\epsilon_r+2}\frac{R^3}{r^3}E_0\sin\theta)\hat{\theta}, 
					0\hat{\phi} \right], & r\geq R
	\end{cases}
\end{equation}
We can further simply (6) and (7), however I choose to leave it this way to always recall that the first term is due to the external field, while the second one is due to the polarized sphere.
As before, assuming $E_0=1$, $R=1$ and $\epsilon_r=1.5$. The electric field ploted in Mathematica and shown in figure \ref{fig:Efield_4.7}
\begin{figure}
	\centering
	\includegraphics[width=0.33\textwidth]{images/plotExample4_7.png} 
	\caption{Electric Field for example 4.7}
	\label{fig:Efield_4.7}
\end{figure}

% Problem 3
\section{Problem 4.7b}
Find the dipole for problem 4.7a\\

Solution: Let us compare the potential outside the sphere (without the contribution of the external field $\vec{E}_0$) to the potential of a dipole, 
\begin{equation}
	V_{dip}(\vec{r}) = \frac{1}{4\pi \epsilon_0}\frac{\vec{p}\cdot\hat{r}}{r^2} 
									= \frac{1}{4\pi \epsilon_0}\frac{p\cos\theta}{r^2} 
									=\frac{E_0R^3}{r^2}\frac{\epsilon_r-1}{\epsilon_r+2}\cos\theta
\end{equation}
And solve for $p$ we have,
\begin{align}
	p = 4\pi R^3\epsilon_0 \frac{\epsilon_r-1}{\epsilon_r+2} E_0\cos\theta \\
	\vec{p} = 4\pi R^3\epsilon_0 \frac{\epsilon_r-1}{\epsilon_r+2} E_0 \hat{z}
\end{align}
By reviewing the last homework I noted that many have also submited the Polarization (dipole moment per unit volume) of example 4.7. 
Using the second relation in (1). We can define
\begin{equation}
	\vec{p} = \int_{\tau}\vec{P}d\tau = \frac{4\pi R^3}{3}\vec{P}
\end{equation}
Equating (10) with (11) and solving for $\vec{P}$ yields
\begin{equation}
	\vec{P} = 3E_0\frac{\epsilon_r-1}{\epsilon_r+2}\epsilon_0 \hat{z}
\end{equation}
End.
\end{document}
