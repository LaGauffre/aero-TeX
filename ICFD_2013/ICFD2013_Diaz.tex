%*************************************************************
% Intented for ICFD2013
% Format modifications by Manuel Diaz, NTU, 2013.07.28
%*************************************************************
\documentclass[twoside,twocolumn,prc,floats,amsmath,amssymb]{revtex4} %for publication
\usepackage[paperwidth=210mm,paperheight=297mm,centering,hmargin=2cm,vmargin=2.5cm]{geometry}
\usepackage{graphicx}		% Include figure files
\usepackage{bm}					% bold math
\RequirePackage{fix-cm}	%
\usepackage{subcaption}	% Subfigures package
\captionsetup{compatibility=false}

% Compacting texfile! 
\setlength{\parskip}{0cm}				% Remove space between paragrahps
\setlength{\parindent}{1em}			
\usepackage{mathptmx}						% Times Roman Font
\usepackage[compact]{titlesec}	% Reduce space around sections headings
\titlespacing{\section}{0pt}{2ex}{1ex}
\linespread{0.90}								% Use linespread [the last silver bullet :( ]

% Modifications to the style
\renewcommand\thesection{\arabic{section}}
\renewcommand\thesubsection{\thesection.\arabic{subsection}}

% Begin document
\begin{document}\fontsize{11.5}{10}
\title{Towards a General Purpose Algorithm for Applications on Rarefied Gas Flows Using Semi-classical Boltzmann-BGK Equation}
\author{ Jaw-Yen Yang$^{1,2}$ } 
\author{ \underline{Manuel Diaz$^{1}$} }
\author{ Ming-Hung Chen$^{3}$ }
\date{\today}

\begin{abstract}
We report the current progress on the development of a general purpose algorithm for dealing with applications for rarefied gas dynamics that follows classical and quantum statisticas by using a parallel treatment of the semiclassical Boltzmann ellipsoidad statisitical (ES) BGK equation. Computation of various degrees of rarefaction and Hydrodynamic limits are presented using a numerical method that combines the discrete velocity (or momentum) ordinate method in momentum space and Discontinuous Galerkin-Finite Element Method (DG-FEM) as high resolution methods for shock capturing in physical space.
\end{abstract}

\affiliation{$^{1}$Institute of Applied Mechanics, National Taiwan University, Taipei 106, TAIWAN}
\affiliation{$^{2}$Center of Advanced Study in Theoretical Science, National Taiwan University, Taipei 106, TAIWAN}
\affiliation{$^{3}$Department of Mathematics, National Cheng Kung University, Tainan 70101, TAIWAN}

%\keywords{Suggested keywords}%Use showkeys class option if keyword
                              %display desired
\maketitle

\section{Introduction}

\label{sec:1}   
Based on the generalization of Uehling and Uhlenbeck \cite{Ueh1933} for the classical Boltzmann Equation to quantum gases, and the BGK approximation \cite{BGK1954}; new classes of classical and semiclassical Boltzmann-BGK solvers have been derived and succesfully tested for dealing with applications near equilibrium \cite{Yang2007,Yang2009,Wu2012,Yang2013}. In this regard, we further study the implementation in \cite{Yang2013} with a class of high resolution shock-capturing scheemes based on dicontinuous Galerkin-finite element method due to have combined advantages of finite element and finite volume methods. In this method one may assume, for different element, shape functions of differente degress. On the other hand, this method have the property of element wise conservation. In the classical rarefied gas flow computation, the implementation of discrete ordinate method to nonlinear model Boltzmann equations has been developed by Yang and Huang \cite{Yang1995}.  Extension to semiclassical Boltzmann-BGK equation has been reported \cite{Yang2013} Also, if the classical limit situations of the same flow problem are considered, then one expects to obtain similar or identical flow structures for the three statistics. Computations of several 2-D Riemann problems \cite{Rinne1993, Laxliu1995} have been performed and flows over a \emph{Forward Facing Step}, extensively studied in \cite{Cockburn1998,Woodward1984}, are used to illustrate the complex rarefied gas dynamics as governed by the semiclassical Boltzmann-ES-BGK equation.

\section{Governing Equations in Two Space Dimensions}
\label{sec:3}
The semiclassical Boltzmann-UU-BGK equation in two space dimensions can be expressed as
%%
\begin{align}
\begin{split}
&\frac{\partial f({\upsilon}_x,{\upsilon}_y, x, y, t)}{\partial t} + {\upsilon}_x\,\frac{\partial f({\upsilon}_x,{\upsilon}_y, x, y, t)}{\partial x } \\
&+{\upsilon}_y\,\frac{\partial f({\upsilon}_x,{\upsilon}_y, x, y, t)} {\partial y} =-\ \frac{f-f^{ES}_{2d}}{\tau },
\end{split}
\label{eq:normalized_B_ES_BGK}
\end{align}
where ${\upsilon}_x$ and ${\upsilon}_y$ as particle velocity components and the two-dimensional ES equilibrium distribution, $f^{ES}_{2d}$, is
\begin{equation}
\begin{split}
f^{UU}_{2d}\left({\upsilon}_x,{\upsilon}_y, x, y, t\right) &= \\
&\frac{1}{z^{-1} \, exp\left\{ \frac{(v_{x}-u_{x})^2 + (v_{y}-u_{y})^2}{T}   \right\} + \theta }
\end{split}
\label{eq:normalized_ESBGK_PDF}
\end{equation}
where $\theta = -1, 0$, and $+1$ denote the Bose-Einstein (BE), Maxwell-Boltzmann (MB), and the Fermi-Dirac (FD) statistics, respectively. 

Once the distribution function is known, the macroscopic quantities, the number density $n$, number density flux $n \vec u$, and energy density $\epsilon$, the pressure tensor $P_{ij}$ and the heat flux vector $Q_{i}$ are defined, respectively, by
\begin{align}
\Phi (\vec x, t) = \int  f(\vec p, \vec x, t) \phi(\vec p)\frac{d \vec p }{ h^3},
\end{align}
where $\Phi =(n, n\vec u, \epsilon)^T$ and $\phi=(1, \vec \xi, \frac{m}{2} c^2)^T$.  Here, $\vec \xi =\vec p/m$ is the particle velocity and $\vec c= \vec \xi - \vec u$ is the thermal velocity.  The gas pressure is defined by $P(\vec x, t) = [\epsilon(x,y,t) -\frac{1}{2}n(x,y,t)(u_x^2-u_y^2)](\gamma-1)$. To obtain the new $f^{UU}$, one needs $z$ this value can and these can be determined through solving,
\begin{equation}
	\Psi(z) = 2\epsilon - \frac{\mathcal{Q}_{2}(z)}{\pi}\frac{n}{\mathcal{Q}_{1}(z)} - n(u_x^2 + u_y^2)
\end{equation}
Using a suitable root finding algorithm. Multiplying Eq. (1) by $1, \vec p$, or $\vec p^2/2m$, and integrating the resulting equations over all $\vec p$, then one obtains the general hydrodynamical equations
\begin{align}
\frac{ \partial n}{\partial t} &+ \nabla_{\vec x} \cdot (n \vec u) = 0, \\
n ( \frac{ \partial }{\partial t} &+ \vec u \cdot \nabla_{\vec x}) u_{i} + \frac{\partial P_{ij} }{\partial x_{j} } = 0, \\ 
\frac{\partial \epsilon}{\partial t} &+ \nabla_{\vec x} \cdot (\epsilon \vec u) + \nabla_{\vec x} \cdot \vec Q + S_{ij} P_{ij} = 0.
\end{align}
where $S_{ij}=(\partial u_{i}/\partial x_{j} +
\partial u_{j}/\partial x_{i})/2$ is the rate of strain tensor.

\section{Solution Methods}

We first apply the discrete ordinate method to discretize the velocity space and render a set of hyperbolic conservation equation with source term in physical space.  Then we implement a class of high resolution shock capturing scheme. This direct solution methods in phase space for the Boltzmann-BGK type equations have been proven to be very accurate and efficient and can simulate wide range of flow parameters such as Reynolds number, Mach number and Knudsen numbers \cite{Yang1995, Yang2013}. 

\section{Results and Discussion}

We numerically study the two-dimensional Riemann problems and flows over a facing step for rarefied quantum gas dynamics for several relaxation times using the present direct solver for the semiclassical Boltzmann-BGK equation.   In the $\tau \approx 0$, i.e.,  $Kn \approx 0$ limit, then $f \approx f^{UU}$, we can recover the ideal gas dynamics governed by the semiclassical Euler solution.  Following the works of Lax and Liu \cite{Laxliu1995} and Schultz-Rinne et al. \cite{Rinne1993}, we selected several configurations to be tested among those 19 configurations classified. Each configuration was tested using several relaxation times, and a mesh grid refiment was performed ensure the convergence of the method. Results are consisten with observations in \cite{Rinne1993, Laxliu1995}. In Fig. \ref{fig:FFD_DG-FEM}, an example of the mesh employed, the number density, pressure and fugacity countours are shown for gas following BE statistics in is hydrodynamic limit are presented. The output time 0.2. No analytical solution is known for this case, but direct comparison shows they are consistent with calculations in \cite{Cockburn1998,Woodward1984}. In both case, there are detectable differences among the three statistics although the overall wave patterns are similar. Quantitatively, comparing the same contours for the three statistics, the numerical values of number density and pressure for the Bose-Einstein statistics are the largest among the three statistics and the Maxwell-Boltzmann statistics always lie between the other two as dictated by the $\theta$ values.

\begin{figure}
        \centering
        \begin{subfigure}[b]{0.45\textwidth}
                \centering
                \includegraphics[trim = 20mm 18mm 20mm 136mm,clip,width=\textwidth]{ForwardFacingStep_Mesh}
                \caption{Mesh}
                \label{FFS_mesh}
        \end{subfigure}%
				
				%~ %add desired spacing between images, e. g. ~, \quad, \qquad etc.
          %(or a blank line to force the subfigure onto a new line)
        \begin{subfigure}[b]{0.45\textwidth}
                \centering
                \includegraphics[trim = 20mm 18mm 20mm 140mm,clip,width=\textwidth]{ForwardFacingStep_Density}
                \caption{Density}
                \label{fig:FFS_Density}
        \end{subfigure}
				
        \begin{subfigure}[b]{0.45\textwidth}
								\centering
                \includegraphics[trim = 20mm 18mm 20mm 140mm,clip,width=\textwidth]{ForwardFacingStep_Pressure}
                \caption{Pressure}
                \label{fig:FFS_Pressure}
        \end{subfigure}
				
        %\begin{subfigure}[b]{0.45\textwidth}
				%				\centering
        %        \includegraphics[trim = 20mm 18mm 20mm 140mm,clip,width=\textwidth]{ForwardFacingStep_Temperature}
        %        \caption{Temperature}
        %        \label{fig:FFS_temperaute}
        %\end{subfigure}
				%
				\begin{subfigure}[b]{0.45\textwidth}
								\centering
                \includegraphics[trim = 20mm 18mm 20mm 140mm,clip,width=\textwidth]{ForwardFacingStep_Fugacity}
                \caption{Fugacity}
                \label{fig:FFS_Fugacity}
        \end{subfigure}
				\caption{Hidrodynamic Limit for BE gas over a \emph{Forward Facing step} using DG-FEM method.}
				\label{fig:FFD_DG-FEM}
\end{figure}

\section{Concluding Remarks}
Computations of 2-D rarefied gas flows based on the semiclassical Boltzmann-BGK equation as proposed in \cite{Yang2013} have been presented. The computational method treats the governing equation in phase space and employs the discrete ordinate method and high resolution shock capturing schemes. Specifically, we describe the solution method in details for the equation in two space dimensions.   A decoding procedure is devised for the semiclassical distribution which is different from that for standard Bose-Einstein or Fermi-Dirac distribution.  Computations of two dimensional problems for rarefied gas flows of arbitrary particle statistics are where performed for several order of relaxation times which corresponding to various range of Knudsen numbers.  Mesh refinement test for solution convergence has been checked and our results for small Knudsen number (Euler limit) are in good agreement with the calculations in \cite{Laxliu1995}\cite{Rinne1993} and with \cite{Cockburn1998,Woodward1984}.    These computational examples serve the purpose of exploring the nonlinear manifestation of shock wave, contact line and rarefaction wave and testing the robustness of the present method. All the expected flow profiles comprising shock, rarefaction wave and contact discontinuities of semiclassical ideal gases and their nonlinear interactions can be observed with considerably good detail and are in good agreement with available results. The present work emphasizes on building the unified and parallel framework for treating semiclassical gas dynamics of three statistics. 

This work is supported by grants NSC 99-2221-E002-084-MY3 and CASTS Subproject 10R80909-4.

%\#597R0066-69
\vspace*{-.25cm}
\begin{thebibliography}{1}
\section{References}
\bibitem{Ueh1933} E. A. Uehling and G. E. Uhlenbeck, Phys. Rev. {\bf 43}, (1933) 552. 
\bibitem{BGK1954} P. L. Bhatnagar, E. P. Gross and M. Krook, Phys. Rev. {\bf 94}, 511 (1954).
\bibitem{Wu2012} W. Lei, J. P Meng, and Yonghao Zhang,Proc. Roy. Soc. A, {\bf 468},(2012), p. 1799.
\bibitem{Yang1995} J. Y. Yang and J. C. Huang, J. Comput. Phys., {\bf 120}, (1995) 323-339.
\bibitem{Yang2007} J. Y. Yang,T. Y. Hsieh, Y. H. Shi, Sci. Comput. {\bf 29},(2007) 221-244.
\bibitem{Yang2009} J. Y. Yang and L. H. Hung, Phys Rev. E {\bf 79},(2009), 056708.
\bibitem{Yang2013} J. Y. Yang,B. P. Muljadi, Z. H. Li and H. X. Zhang, Commun. Comput. Phys. {\bf 14}, (2013) p. 242-264 .
\bibitem{Rinne1993} C. W. Schultz-Rinne, , J. P. Collinsand H. M. Glaz, SIAM J. Sci. Comput. {\bf 14}, (1993) 1394-1414.
\bibitem{Laxliu1995} P. D. Lax and X. D. Liu, SIAM J. Sci. Comput {\bf 19}, (1995) 319-340.
\bibitem{Cockburn1998} B. Cockburn and C. W. Shu, J Comput. Phys. {\bf 141}, (1998) p. 199-224.
\bibitem{Woodward1984} P. Woodward and P. Colella, J Comput. Phys. {\bf 54}, (1984) p. 115-173.

\end{thebibliography}
%\bibliographystyle{Thesisstyle}

\end{document}
%
% ****** End of file ICFD2013.tex ******
