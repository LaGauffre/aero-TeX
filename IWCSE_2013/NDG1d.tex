\subsection{High Resolution Scheme}
\begin{frame}
  \frametitle{Nodal Discontinuous Galerkin Method}
  %\begin{block}{Note}
  %   We follow Professors J.S. Hesthaven \& T. Warburton implementation \cite{hesthaven2008nodal} in the following presentation.
  %\end{block}
  Let us start with the one-dimensional non-homogenous conservation law for the solution $u(t,x)$,
  \begin{align}
    \frac{\partial u}{\partial t} + \frac{\partial f(u)}{\partial x} = s(u), \;\;\; x \in [a,b]
		\label{eq:non_homogeneous_consevationlaw}
  \end{align}	
  Where $f(u)$ is the flux, and $s(u)$ is a source term. \\
	Togueter with initial condition,
	\begin{equation}
		u(0,x) = u_0(x),
	\end{equation}
	and suitable boundary conditions provided as
	\begin{align}
	&u(t,R)=g_1(t), &u(t,L)=g_2(t) 
	\end{align}
\end{frame}

\begin{frame}
	Define for our global coordinate, $x \in [a,b]$,
	
	\begin{itemize}
		\item $x_{j+\frac{1}{2}}, \ (j=1,\cdots,N+1)$, as the grid partition of our linear domain,
		\item $E_j = [x_{j-\frac{1}{2}},x_{j+\frac{1}{2}}], \ (j=1,\cdots,N)$, non-overlaping cell/elements,
		\item $\Delta x_j = x_{j-\frac{1}{2}} - x_{j+\frac{1}{2}}$, as the local grid size,
		\item $xc_j = \frac{1}{2}(x_{j-\frac{1}{2}} + x_{j+\frac{1}{2}})$, as the cell/elements center,
		\item $\xi_k \in [-1,1], \ (k=1,\cdots,K)$, a local coordinate for a standard element,
		\item $K-1$, as the order of the polynomail used to approximate our solution in every local element.
	\end{itemize}
	A simple way to map the information from the local, $\xi$, to global, $x$, coordinates is
	\begin{equation}
		x(\xi) = xc_j + J\xi,
	\end{equation}
	where
	\begin{equation}
		J=\frac{\Delta x_j}{2}.
	\end{equation}
\end{frame}
	
\begin{frame}	
	In DG methods, the solution is represented by a global solution as
	\begin{equation}
   u(t,x) \approx u^h(t,x) = \oplus^N_{j=1} u_j^h(t,x_j)
  \end{equation}
	where the local solutions can be assumed either,
	\begin{equation}
		u_j^h(t,x_j) = \sum_{n=1}^{K} \hat{u}_n^k(t) \phi_{n}(x)
								 = \sum_{j=1}^{K} u_j^h(t,x_{j,k}) \psi_{j}^k(x), 
								\text{ for } x \in [x_{j-\frac{1}{2}},x_{j+\frac{1}{2}}]
		%u_j^h(t,x_j) = \sum_{n=1}^{K} \hat{u}_n^k(t) \mathcal{P}_{n-1}(x)
		%						 = \sum_{j=1}^{K} u_j^h(t,x_{j,k})\mathcal{L}_{j}^k(x), 
		%						\text{ for } x \in [x_{j-\frac{1}{2}},x_{j+\frac{1}{2}}]
	\end{equation}
	Where the first approximation corresponds is by modal expansion while the second one is by nodal expansion.
	Here $\phi_{n}(x)$ and $\psi_{j}^k(x)$ are base functions that corresponds to $U_n$ and $U_j^k$ function space respectively.
	
\end{frame}

\begin{frame}
	By using the $u(t,x) \approx u_h(t,x)$ into (\ref{eq:non_homogeneous_consevationlaw}), the residual function can be defined as
	\begin{align}
	 R_h = \frac{\partial u_h}{\partial t} + \frac{\partial f(u_h)}{\partial x} - s(t,x),
	\label{eq:residual}
	\end{align}
	Here let us require that the local residual be orthogonal to all test functions $\Phi_j^k \in V_j^k$.
	\begin{align}
	\int_{D^k} R_h(t,x) \Phi_j(x)dx = 0, \;\; 1 \geq j \geq N_p
	\end{align}
	One way to ensure this condition is to make the space of the base functions $U$ and the test functions $V$ coincide.
\end{frame}

\begin{frame}
	By making coincide the function spaces of base and test functions, we are using Galerkin method.
	Integrate by parts and use Gauss' theorem on (\ref{eq:residual}). We then the weak formulation for (\ref{eq:non_homogeneous_consevationlaw}) is given by,
	\begin{equation}
	\int_{D^k} \left( \frac{\partial u_h^k}{\partial t} \psi_h^k - f_h^k(u_h^k) \frac{d\psi_h^k}{dx} -s \ \psi_h^k\right) dx = 
	- \int_{\partial D^k} \hat{\vec{n}}\cdot f^{*} \psi_h^k dx,
	\label{eq:weakform}
	\end{equation}
	and by integrating by part again, the strong formulation for (\ref{eq:non_homogeneous_consevationlaw}) is,
	\begin{equation}
	\int_{D^k} \left( \frac{\partial u_h^k}{\partial t} + \frac{\partial f_h^k(u_h^k)}{\partial x} -s\right) \psi_h^k dx = 
	-\int_{\partial D^k} \hat{\vec{n}}\cdot ( f_h^k(u_h^k) - f^{*} ) \psi_h^k dx.
	\label{eq:strongfrom}
	\end{equation}
	Note that the flux is allowed to be discontinuous across the boundaries of each element. Here the term $f^{*}$ is introduced as our numerical flux at the elements interfaces. Here we will use Lax-Friedrich in all our numerical experiments,
	\begin{align}
		f^{*} = f^{LF}(a,b) = \frac{1}{2}[f(a)+f(b)-\alpha(b-a)], \;\; \alpha=max|f'(u)|
	\end{align}
\end{frame}

\begin{frame}
	Let us concider the local approximations of the form,
	\begin{equation}
		u^h(t,\xi) = \sum_{j=1}^{K} \hat{u}_j^k(t) \mathcal{P}_{j-1}(\xi)
							 = \sum_{j=1}^{K} u_j^h(t,\xi_k)\mathcal{L}_{j}^k(\xi), 
							\text{ for } \xi \in [-1,1]
	\end{equation}
	Here $\xi_k$ are choosen as Legendre-Gauss-Lobatto (LGL) points, $\mathcal{P}_{n}(x)$ is the n-th order Legendre Polynomial and therefore $\hat{u}_j^k(t)$ are called Legendre coeficients. Lastly $\mathcal{L}_{j}^k(x)$ is defined as,
	\begin{equation}
		\mathcal{L}_{j}^k(\xi) = \prod_{\substack{j = 1 \\ j\neq i}}^K \frac{\xi-\xi_j}{\xi_i-\xi_j},
	\end{equation}
	Notice that in FEM context $\mathcal{L}_{j}(x)$ can be also considered the shape function of the $j$th-node .
\end{frame}

\begin{frame}
	The LGL points can be found as the roots of Lobatto Polynomials. The lobato polynomials can be defined in terms of Legendre Polynomials as,
	\begin{equation}
		Lo_k = \mathcal{P}_{k}-\mathcal{P}_{k-2}
	\end{equation}
	The first 4 lobato polynomials are,
	\begin{align*}
		\left[\frac{3}{2} \left(x^2-1\right), 
		\frac{5}{2} x \left(x^2-1\right), 
		\frac{7}{8} \left(5 x^4-6 x^2+1\right), 
		\frac{9}{8} x \left(7 x^4-10 x^2+3\right)\right]
	\end{align*}
	\begin{figure}
		\centering
		\includegraphics[trim = 0mm 0mm 0mm 0mm,clip,width=0.50\textwidth]{../IWCSE_2013/Mathematica_pics/Lobatto}
		\caption{Plot of initial four Lobatto polynomials}
    \label{fig:Lobatto_polynomials}
	\end{figure}
\end{frame}

\begin{frame}
	A relation between the nodal information and the Legendre coeficients can be defined as,
	\begin{equation}
		\mathcal{V}\hat{\vec{u}} = \vec{u},
	\end{equation}
	Where 
	\begin{align}
		&\mathcal{V}_{ij} = \mathcal{P}_{j-1}(\xi_i),& 
		&\hat{\vec{u}}=\hat{u}_i,& 
		&\vec{u}_i=u(\xi_i).&
	\end{align}
	The matrix, $\mathcal{V}_{ij}$, is recognized as the generalized Vandermonde matrix, and will ensure that our noda data, $\vec{u}_i$, is well conditioned \cite{hesthaven2008nodal}. The nodal base function is the defined as,
	\begin{equation}
		\mathcal{L}_{j}(\xi) = \sum_{n=1}^K (\mathcal{V}^T_{jn})^{-1}\mathcal{P}_{n-1}(\xi)
	\end{equation}
	Therefore, the mass matrix, the stiffness matrix and legendre derivates matrix, are
	\begin{align}
		\mathcal{M}_{ij} = \int_{-1}^{1} \mathcal{L}_{i}(\xi)\mathcal{L}_{j}(\xi) d\xi =& (\mathcal{V}\mathcal{V}^T)^{-1}, 
		\:\:\:
		\mathcal{S}_{ij} = \int_{-1}^{1} \mathcal{L}_{i}(\xi)\pd{\mathcal{L}_{j}(\xi)}{\xi} d\xi, \\
		\mathcal{D}r =& (\mathcal{M})^{-1}\mathcal{S} = \pd{\mathcal{P}_{j}}{\xi}|_{\xi_k}.
	\end{align}
\end{frame}


\begin{frame}
	Using the strong form in (\ref{eq:strongfrom}), We obtain the following scheme,
	\begin{equation}
		(F_j)_{\xi}(\xi_k) = -\mathcal{D}r^{j} \ f(u_{j,k}) + 
		(\mathcal{M}^j)^{-1} \left[\left[\mathcal{L}^{j}(f(u_{j,k})-f^{*})\right]_{-1}^{+1} 
		+ s(u_{j,k}) \right]
	\end{equation}
	finally we can update the numerical solution using the followind ODE,
	\begin{equation}
		\pd{u_{j,k}}{t} + \frac{1}{J} (F_j)_{\xi}(\xi_k);
	\end{equation}
	Which can be solved using the choosen RK method in (\ref{eq:RK33}).
\end{frame}