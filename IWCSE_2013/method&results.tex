\section{Method \& Results}

\subsection{Discrete Ordinate Method (DOM)}
\begin{frame} \frametitle{Conventional Discrete Ordinate Method}
	Following ideas in \cite{Yang1995323}, when applying conventional DOM to either classical or semi-classical Boltzmann-BGK formulation, we render it as set of linear PDE's with fixed/constant advection velocity $v_\sigma$,
	\begin{align}
	\frac{\partial{f_\sigma}}{\partial{t}} +
	v_\sigma\cdot\frac{\partial{f_\sigma}}{\partial x} = 
	-\frac{1}{\tau}(f_\sigma-f^{eq}_\sigma)
	\end{align}
Here $\sigma$ $(\sigma = 1,\dots,N_v)$ stands as the index the number of discrete velocities. Each equation has a corresponding equilibrium distribution funcion, namely
	\begin{align}
	f_\sigma^{M}(t,x,v_\sigma)= \frac{\rho}{(\pi T)^{\frac{1}{2}}} \exp\left({-\frac{(v_\sigma-u)^2}{T}} \right)
	\end{align}
\end{frame}

\begin{frame}
Graphical Perspective of the evolution of the Boltzmann-BGK Equation
\begin{figure}
	\centering
	\includegraphics[trim = 20mm 10mm 30mm 20mm,clip,width=\textwidth]{../IWCSE_2013/BasicConcepts_pics/MB_IC}
	\caption{Initial Condition}
	\label{fig:f_IC}
\end{figure}
\end{frame}

\begin{frame}
Graphical Perspective of the evolution of the Boltzmann-BGK Equation
\begin{figure}
	\centering
	\includegraphics[trim = 20mm 10mm 30mm 20mm,clip,width=\textwidth]{../IWCSE_2013/BasicConcepts_pics/MB_evolution}
	\caption{Evolution by BBGK}
	\label{fig:f_evolution}
\end{figure}
\end{frame}

\begin{frame}
	\frametitle{Moments of the classical distribution function}
	We use Gauss-Hermite quadrature rule to re-express the integral moments of our distribution function,
	\begin{subequations}
	\begin{align}
	\rho = \int f_\sigma dv &= \sum_\sigma W_\sigma \exp{(v_\sigma^2)} f(t,x,v_\sigma)\\
	\rho u = \int v_\sigma f_\sigma dv &= \sum_\sigma v_\sigma W_\sigma \exp{(v_\sigma^2)} f(t,x,v_\sigma)\\
	E = \int \frac{v_\sigma^2}{2} f_\sigma dv &= \sum_\sigma \frac{v_\sigma^2}{2} W_\sigma \exp{(v_\sigma^2)} f(t,x,v_\sigma)\\
	\rho \epsilon = \int \frac{C_\sigma^2}{2} f_\sigma dv &= \sum_\sigma \frac{C_\sigma^2}{2} W_\sigma \exp{(v_\sigma^2)} f(t,x,v_\sigma)
	\end{align}
	\end{subequations}
	Here we let $v_\sigma$ be the abscissas of our quadrature and $W_\sigma$ their corresponding weigthing values. 
\end{frame} 

\subsection{Time Integration}
\begin{frame} \frametitle{Dealing with time}
	We focus on a explicit multi-stage strong stability preserving (SSP) Runge-Kutta (RK) schemes as presented in \cite{Gottlieb2005} for the integration in the temporal dimension. The semidiscrete form of our model is 
	\begin{equation}
		\frac{du_h}{dt} = \mathcal{L}_h(u_h,t)
	\label{eq:semidiscrete_problem}
	\end{equation}
	Where $u_h$ is the variable we want to solve. The choosen Runge-Kutta scheme is expressed as 
	\begin{equation}
		\begin{split}
		u_h^{(1)}   &= u_h^{n} + \Delta t R(u_h^{n}) \\
		u_h^{(2)}   &= \frac{3}{4} u_h^{n} + \frac{1}{4}[u_h^{(1)}+\Delta t R(u_h^{(1)})] \\
		u_h^{(n+1)} &= \frac{1}{3} u_h^{n} + \frac{2}{3}[u_h^{(2)}+\Delta t R(u_h^{(2)})]
	\end{split}
	\end{equation}
\end{frame}

\begin{frame}
  \frametitle{Nodal DG-FEM}
  \begin{block}{Note}
     We follow Professors J.S. Hesthaven \& T. Warburton implementation \cite{hesthaven2008nodal} in the following presentation.
  \end{block}
  Let us start with the one-dimensional scalar conservation law for the solution $u(t,x)$,
  \begin{align}
    \frac{\partial u}{\partial t} + \frac{\partial f(u)}{\partial x} = s(u), \;\;\; x \in [L,R]
		\label{eq:non_homogeneous_consevationlaw}
  \end{align}	
  Where $f(u)$ is the flux, and $s(u)$ is a source term. \\
	Togueter with initial condition $u(0,x) = u_0(x),$ and suitable boundary conditions provided as
	\begin{align}
	&u(t,R)=g_1(t), &u(t,L)=g_2(t) 
	\end{align}
\end{frame}

\begin{frame}
	Define:
	\begin{itemize}
	\item $x^k, \ k=1,\cdots,K$ as the grid partition of our linear domain,
	\item $D^k = [x^k,x^{k+1}]$ as our non-overlaping cells or elements,
	\item $h^k = x^{k+1}-x^k$ as the local grid size.
	\end{itemize}
	We start by assuming that we can represent the global solution as
	\begin{equation}
   u(t,x) \approx u_h(t,x) = \oplus^K_{k=1} u^k_h(t,x^k)
  \end{equation}
	where the local solutions are assumed to be of the form
	\begin{equation}
	x \in D^k = [x^k,x^{k+1}]: u_h^k(t,x) = \sum_{i=1}^{N_p} u_h^k(x_i^k,t)\phi_i^k(x)
	\end{equation}
	Where $\phi_i^k(x) \in U_h^k$ is a base function and $N_p$ is the number of nodes inside every $k$ element.
\end{frame}

\begin{frame}
	By using the $u(t,x) \approx u_h(t,x)$ into (\ref{eq:non_homogeneous_consevationlaw}), the residual is defined as
	\begin{align}
	 R_h = \frac{\partial u_h}{\partial t} + \frac{\partial f(u_h)}{\partial x} - s(t,x),
	\label{eq:residual}
	\end{align}
	We now require that the local residual be orthogonal to all test functions $\psi_h^k \in V_h^k$.
	\begin{align}
	\int_{D^k} R_h(t,x) \psi_j(x)dx = 0, \;\; 1 \geq j \geq N_p
	\end{align}
	This is acomplish by using Galerking method, i.e., making the space of the base functions $U_h^k$ and the test functions $V_h^k$ coincide.
\end{frame}

\begin{frame}
	For our test function $\phi_i^k(x)$ we will use Legendre-Gauss-Lobatto (LGL) polynomials. The first few Lobatto polynomials are
	\begin{align*}
		\left[\frac{3}{2} \left(x^2-1\right), 
		\frac{5}{2} x \left(x^2-1\right), 
		\frac{7}{8} \left(5 x^4-6 x^2+1\right), 
		\frac{9}{8} x \left(7 x^4-10 x^2+3\right)\right]
	\end{align*}
	\begin{figure}
		\centering
		\includegraphics[trim = 0mm 0mm 0mm 0mm,clip,width=0.50\textwidth]{../IWCSE_2013/Mathematica_pics/Lobatto}
		\caption{Plot of initial four Lobatto polynomials}
    \label{fig:Lobatto_polynomials}
	\end{figure}
\end{frame}

\begin{frame}
	Using gauss' theorem on (\ref{eq:residual}) yields,
	\begin{equation}
	\int_{D^k} \left( \frac{\partial u_h^k}{\partial t} \psi_h^k - f_h^k(u_h^k) \frac{d\psi_h^k}{dx} -g \ \psi_h^k\right) dx = 
	- \int_{\partial D^k} \hat{\vec{n}}\cdot f_h^{k} \psi_h^k dx
	\end{equation}
	introduce the numerical flux, $f^{*}$, as the unique value to be used at the interface and obtained by using the information from both elements. Then, we recover the weak formulation
	\begin{equation}
	\int_{D^k} \left( \frac{\partial u_h^k}{\partial t} \psi_h^k - f_h^k(u_h^k) \frac{d\psi_h^k}{dx} -g \ \psi_h^k\right) dx = 
	- \int_{\partial D^k} \hat{\vec{n}}\cdot f^{*} \psi_h^k dx
	\end{equation}
	an the strong form:
	\begin{equation}
	\int_{D^k} \left( \frac{\partial u_h^k}{\partial t} + \frac{\partial f_h^k(u_h^k)}{\partial x} -g\right) \psi_h^k dx = 
	-\int_{\partial D^k} \hat{\vec{n}}\cdot ( f_h^k(u_h^k) - f^{*} ) \psi_h^k dx
	\end{equation}
\end{frame}

\begin{frame} \frametitle{Numerical Flux}
	$f^{*}$ is a monotone numerical flux between the boundaries of every cell/element. 
	Here we choose to use the Lax-Friedrichs,
	\begin{align}
		f^{*} = f^{LF}(a,b) = \frac{1}{2}[f(a)+f(b)-\alpha(b-a)], \;\; \alpha=max|f'(u)|
	\end{align}
\end{frame}

\begin{frame}
	by using Galerkin Approach, we discover the local semidiscrete for the strong form,
	\begin{equation}
		\frac{du_h^k}{dt} + (\mathcal{M}^k)^{-1} (\mathcal{S} \ f(u_h^k) - g ) = 
		(\mathcal{M}^k)^{-1} \left[\psi^k(f_h^k(u_h^k) - f^{*})\right]_{x_l^k}^{x_r^k}
	\end{equation}
	where
	\begin{align}
		&\mathcal{M}_{ij}^{k} = (\psi_i^k,\psi_j^k)_{D^k}, 
		&\mathcal{S}_{ij}^{k} = \left( \psi_i^k,\frac{d\psi_j^k}{dx} \right)_{D^k}
	\end{align}
	are the local mass and stiffness matrices, respectively.
\end{frame}

\subsection{1D \& 2D cases}
\begin{frame} \frametitle{Numerical Results}
	In the following, test our algorithm in some classical shocktube cases in the range of $x \in [0,1]$ for the following initial condition problems:
	\begin{enumerate}
	\item Sod's Shocktube Problem
		\begin{itemize}
		\item IC: 
		\[
		[\rho,u,p] =
		\begin{cases}
		[1.0,0.75,1.0] & \text{if } x \geq 0.5 \\
		[0.125,0,0.1] & \text{if } x < 0.5
		\end{cases}
		\]
		\end{itemize}
	\item Lax's Shocktube Problem
	\begin{itemize}
		\item IC: 
		\[
		[\rho,u,p] =
		\begin{cases}
		[0.445,0.698,3.528] & \text{if } x \geq 0.5 \\
		[0.5,0,0.571] & \text{if } x < 0.5
		\end{cases}
		\]
		\end{itemize}
	\end{enumerate}
\end{frame}

\begin{frame}
		\begin{figure}
			\centering
				\includegraphics[width=0.95\textwidth]{../IWCSE_2013/result1d/SODsDG100N3}
			\caption{Sod's  Socktube Problem. Advecting information with Nodal DG method with 100 elements (3rd order LGL polynomials) in physical space, RK3 in time, $\tau = 1/10^4$ and DOM with 81 velocity points.}
			\label{fig:SODsDG100N3}
		\end{figure}
\end{frame}

\begin{frame}
		\begin{figure}
			\centering
				\includegraphics[width=0.95\textwidth]{../IWCSE_2013/result1d/SODsDG100N4}
			\caption{Sod's  Socktube Problem. Advecting information with Nodal DG method with 100 elements (4th order LGL polynomials) in physical space, RK3 in time, $\tau = 1/10^4$ and DOM with 81 velocity points.}
			\label{fig:SODsDG100N4}
		\end{figure}
\end{frame}

\begin{frame}
		\begin{figure}
			\centering
				\includegraphics[width=0.95\textwidth]{../IWCSE_2013/result1d/SODsDG200N3}
			\caption{Sod's  Socktube Problem. Advecting information with Nodal DG method with 200 elements (3rd order LGL polynomials) in physical space, RK3 in time, $\tau = 1/10^4$ and DOM with 81 velocity points.}
			\label{fig:SODsDG200N3}
		\end{figure}
\end{frame}

\begin{frame}
		\begin{figure}
			\centering
				\includegraphics[width=0.95\textwidth]{../IWCSE_2013/result1d/LAXsDG100N3}
			\caption{Sod's Shocktube Problem. Advecting information with Nodal DG method with 100 elements (3rd order LGL polynomials) in physical space, RK3 in time, $\tau = 1/10^4$ and DOM with 200 velocity points.}
			\label{fig:LAXsDG100N3}
		\end{figure}
\end{frame}

\begin{frame}
		\begin{figure}
			\centering
				\includegraphics[width=0.95\textwidth]{../IWCSE_2013/result1d/LAXsDG100N4}
			\caption{Sod's Shocktube Problem. Advecting information with Nodal DG method with 100 elements (4th order LGL polynomials) in physical space, RK3 in time, $\tau = 1/10^4$ and DOM with 200 velocity points.}
			\label{fig:LAXsDG100N4}
		\end{figure}
\end{frame}

\begin{frame}
		\begin{figure}
			\centering
				\includegraphics[width=0.95\textwidth]{../IWCSE_2013/result1d/LAXsDG150N4}
			\caption{Sod's Shocktube Problem. Advecting information with Nodal DG method with 150 elements (4th order LGL polynomials) in physical space, RK3 in time, $\tau = 1/10^4$ and DOM with 200 velocity points.}
			\label{fig:LAXsDG150N4}
		\end{figure}
\end{frame}

\begin{frame}
	\begin{figure}
        \centering
        \begin{subfigure}[b]{0.45\textwidth}
                \centering
                \includegraphics[trim = 20mm 18mm 20mm 136mm,clip,width=\textwidth]{../IWCSE_2013/result2d/ForwardFacingStep_Mesh}
                \caption{Mesh}
                \label{FFS_mesh}
        \end{subfigure}%
				~
				%~ %add desired spacing between images, e. g. ~, \quad, \qquad etc.
          %(or a blank line to force the subfigure onto a new line)
        \begin{subfigure}[b]{0.45\textwidth}
                \centering
                \includegraphics[trim = 20mm 18mm 20mm 140mm,clip,width=\textwidth]{../IWCSE_2013/result2d/ForwardFacingStep_Density}
                \caption{Density}
                \label{fig:FFS_Density}
        \end{subfigure}
				
        \begin{subfigure}[b]{0.45\textwidth}
								\centering
                \includegraphics[trim = 20mm 18mm 20mm 140mm,clip,width=\textwidth]{../IWCSE_2013/result2d/ForwardFacingStep_Pressure}
                \caption{Pressure}
                \label{fig:FFS_Pressure}
        \end{subfigure}
				~
        \begin{subfigure}[b]{0.45\textwidth}
								\centering
                \includegraphics[trim = 20mm 18mm 20mm 140mm,clip,width=\textwidth]{../IWCSE_2013/result2d/ForwardFacingStep_Temperature}
                \caption{Temperature}
                \label{fig:FFS_temperaute}
        \end{subfigure}
				%
				%\begin{subfigure}[b]{0.45\textwidth}
				%				\centering
        %        \includegraphics[trim = 20mm 18mm 20mm 140mm,clip,width=\textwidth]{../IWCSE_2013/result2d/ForwardFacingStep_Fugacity}
        %        \caption{Fugacity}
        %        \label{fig:FFS_Fugacity}
        %\end{subfigure}
				\caption{Hydrodynamic Limit classical gas over a \emph{sharp corner} using DG-FEM method and triangular elements. We compare our solution with the one presented in \cite{Qiu:2005:RDG:1046640.1046665}}
				\label{fig:FFD_DG-FEM}
\end{figure}
\end{frame}

\section[]{Acknowlegments}
\begin{frame} \frametitle{Acknowlegment}
The Author wishes to thanks professors
\begin{align*}
&\text{Min-Hung Chen,}& &\text{Jaw-Yen Yang,}& &\text{\&}& &\text{Juan-Chen Huang}&
\end{align*}
for many usefull discussions and their guide in the development of the current work.
\end{frame}