\subsection{1D \& 2D cases}
\begin{frame} \frametitle{Numerical Results}
	In the following, test our algorithm in some classical shocktube cases in the range of $x \in [0,1]$ for the following initial condition problems:
	\begin{enumerate}
	\item Sod's Shocktube Problem
		\begin{itemize}
		\item IC: 
		\[
		[\rho,u,p] =
		\begin{cases}
		[1.0,0.75,1.0] & \text{if } x \geq 0.5 \\
		[0.125,0,0.1] & \text{if } x < 0.5
		\end{cases}
		\]
		\end{itemize}
	\item Lax's Shocktube Problem
	\begin{itemize}
		\item IC: 
		\[
		[\rho,u,p] =
		\begin{cases}
		[0.445,0.698,3.528] & \text{if } x \geq 0.5 \\
		[0.5,0,0.571] & \text{if } x < 0.5
		\end{cases}
		\]
		\end{itemize}
	\end{enumerate}
\end{frame}

\begin{frame}
		\begin{figure}
			\centering
				\includegraphics[width=0.95\textwidth]{../IWCSE_2013/result1d/SODsDG100N3}
			\caption{Sod's  Socktube Problem. Advecting information with Nodal DG method with 100 elements (3rd order LGL polynomials) in physical space, RK3 in time, $\tau = 1/10^4$ and DOM with 81 velocity points.}
			\label{fig:SODsDG100N3}
		\end{figure}
\end{frame}

\begin{frame}
		\begin{figure}
			\centering
				\includegraphics[width=0.95\textwidth]{../IWCSE_2013/result1d/SODsDG100N4}
			\caption{Sod's  Socktube Problem. Advecting information with Nodal DG method with 100 elements (4th order LGL polynomials) in physical space, RK3 in time, $\tau = 1/10^4$ and DOM with 81 velocity points.}
			\label{fig:SODsDG100N4}
		\end{figure}
\end{frame}

\begin{frame}
		\begin{figure}
			\centering
				\includegraphics[width=0.95\textwidth]{../IWCSE_2013/result1d/SODsDG200N3}
			\caption{Sod's  Socktube Problem. Advecting information with Nodal DG method with 200 elements (3rd order LGL polynomials) in physical space, RK3 in time, $\tau = 1/10^4$ and DOM with 81 velocity points.}
			\label{fig:SODsDG200N3}
		\end{figure}
\end{frame}

\begin{frame}
		\begin{figure}
			\centering
				\includegraphics[width=0.95\textwidth]{../IWCSE_2013/result1d/LAXsDG100N3}
			\caption{Sod's Shocktube Problem. Advecting information with Nodal DG method with 100 elements (3rd order LGL polynomials) in physical space, RK3 in time, $\tau = 1/10^4$ and DOM with 200 velocity points.}
			\label{fig:LAXsDG100N3}
		\end{figure}
\end{frame}

\begin{frame}
		\begin{figure}
			\centering
				\includegraphics[width=0.95\textwidth]{../IWCSE_2013/result1d/LAXsDG100N4}
			\caption{Sod's Shocktube Problem. Advecting information with Nodal DG method with 100 elements (4th order LGL polynomials) in physical space, RK3 in time, $\tau = 1/10^4$ and DOM with 200 velocity points.}
			\label{fig:LAXsDG100N4}
		\end{figure}
\end{frame}

\begin{frame}
		\begin{figure}
			\centering
				\includegraphics[width=0.95\textwidth]{../IWCSE_2013/result1d/LAXsDG150N4}
			\caption{Sod's Shocktube Problem. Advecting information with Nodal DG method with 150 elements (4th order LGL polynomials) in physical space, RK3 in time, $\tau = 1/10^4$ and DOM with 200 velocity points.}
			\label{fig:LAXsDG150N4}
		\end{figure}
\end{frame}

\begin{frame}
	\begin{figure}
        \centering
        \begin{subfigure}[b]{0.45\textwidth}
                \centering
                \includegraphics[trim = 20mm 18mm 20mm 136mm,clip,width=\textwidth]{../IWCSE_2013/result2d/ForwardFacingStep_Mesh}
                \caption{Mesh}
                \label{FFS_mesh}
        \end{subfigure}%
				~
				%~ %add desired spacing between images, e. g. ~, \quad, \qquad etc.
          %(or a blank line to force the subfigure onto a new line)
        \begin{subfigure}[b]{0.45\textwidth}
                \centering
                \includegraphics[trim = 20mm 18mm 20mm 140mm,clip,width=\textwidth]{../IWCSE_2013/result2d/ForwardFacingStep_Density}
                \caption{Density}
                \label{fig:FFS_Density}
        \end{subfigure}
				
        \begin{subfigure}[b]{0.45\textwidth}
								\centering
                \includegraphics[trim = 20mm 18mm 20mm 140mm,clip,width=\textwidth]{../IWCSE_2013/result2d/ForwardFacingStep_Pressure}
                \caption{Pressure}
                \label{fig:FFS_Pressure}
        \end{subfigure}
				~
        \begin{subfigure}[b]{0.45\textwidth}
								\centering
                \includegraphics[trim = 20mm 18mm 20mm 140mm,clip,width=\textwidth]{../IWCSE_2013/result2d/ForwardFacingStep_Temperature}
                \caption{Temperature}
                \label{fig:FFS_temperaute}
        \end{subfigure}
				%
				%\begin{subfigure}[b]{0.45\textwidth}
				%				\centering
        %        \includegraphics[trim = 20mm 18mm 20mm 140mm,clip,width=\textwidth]{../IWCSE_2013/result2d/ForwardFacingStep_Fugacity}
        %        \caption{Fugacity}
        %        \label{fig:FFS_Fugacity}
        %\end{subfigure}
				\caption{Hydrodynamic Limit classical gas over a \emph{sharp corner} using DG-FEM method and triangular elements. We compare our solution with the one presented in \cite{Qiu:2005:RDG:1046640.1046665}}
				\label{fig:FFD_DG-FEM}
\end{figure}
\end{frame}