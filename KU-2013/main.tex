% ***********************************************************
% ********* PHYSICS HEADER FOR BEAMER PRESENTATIONS *********
% ***********************************************************
% Version 2.1 
% Orignal author: Chris Clark, UCLA, http://www.dfcd.net/articles/latex
% Modified by: Manuel Diaz, NTU 2013, https://github.com/wme7

\documentclass[xcolor=svgnames, 9pt]{beamer} 
\usetheme{Stockton}
\usepackage{amsmath} % AMS Math Package
\usepackage{amsthm} % Theorem Formatting
\usepackage{amssymb}	% Math symbols such as \mathbb
\usepackage{graphicx} % Allows for eps images
\usepackage{epsfig} %for figures
\usepackage{xcolor} %for color
\usepackage{subcaption} %for figures arrays % Sets margins and page size
% Add water mark in your presentation
\definecolor{hughesblue}{rgb}{.9,.9,1} %A blue I like to use for highlighting, matches Hughes Hallet's book
%\logo{\includegraphics[height=2cm]{NTU_logo_watermark.jpg}} % comment out this line if you do not have the pacific-seal file}

% \DeclareMathOperator{\Sample}{Sample}
\let\vaccent=\v % rename builtin command \v{} to \vaccent{}
\renewcommand{\v}[1]{\ensuremath{\mathbf{#1}}} % for vectors
\newcommand{\gv}[1]{\ensuremath{\mbox{\boldmath$ #1 $}}} 

% for vectors of Greek letters
\newcommand{\uv}[1]{\ensuremath{\mathbf{\hat{#1}}}} % for unit vector
\newcommand{\abs}[1]{\left| #1 \right|} % for absolute value
\newcommand{\avg}[1]{\left< #1 \right>} % for average
\let\underdot=\d % rename builtin command \d{} to \underdot{}
\renewcommand{\d}[2]{\frac{d #1}{d #2}} % for derivatives
\newcommand{\dd}[2]{\frac{d^2 #1}{d #2^2}} % for double derivatives
\newcommand{\pd}[2]{\frac{\partial #1}{\partial #2}} 

% for partial derivatives
\newcommand{\pdd}[2]{\frac{\partial^2 #1}{\partial #2^2}} 

% for double partial derivatives
\newcommand{\pdc}[3]{\left( \frac{\partial #1}{\partial #2}
 \right)_{#3}} % for thermodynamic partial derivatives
\newcommand{\ket}[1]{\left| #1 \right>} % for Dirac bras
\newcommand{\bra}[1]{\left< #1 \right|} % for Dirac kets
\newcommand{\braket}[2]{\left< #1 \vphantom{#2} \right|
 \left. #2 \vphantom{#1} \right>} % for Dirac brackets
\newcommand{\matrixel}[3]{\left< #1 \vphantom{#2#3} \right|
 #2 \left| #3 \vphantom{#1#2} \right>} % for Dirac matrix elements
\newcommand{\grad}[1]{\gv{\nabla} #1} % for gradient
\let\divsymb=\div % rename builtin command \div to \divsymb
\renewcommand{\div}[1]{\gv{\nabla} \cdot #1} % for divergence
\newcommand{\curl}[1]{\gv{\nabla} \times #1} % for curl
\let\baraccent=\= % rename builtin command \= to \baraccent
\renewcommand{\=}[1]{\stackrel{#1}{=}} % for putting numbers above =
\newtheorem{prop}{Proposition}
\newtheorem{thm}{Theorem}[section]
\newtheorem{lem}[thm]{Lemma}
\theoremstyle{definition}
\newtheorem{dfn}{Definition}
\theoremstyle{remark}
\newtheorem*{rmk}{Remark}

% Modify '\vec{}' command to generate bold vector characters
\let\oldhat\hat
\renewcommand{\vec}[1]{\mathbf{#1}}

% At the Beginning of every Section
\AtBeginSection[]
{
  \begin{frame} \frametitle{Outline}
    \tableofcontents[currentsection]
  \end{frame}
}

% ***********************************************************
% ********************** END HEADER **************** MD2013 *
% ***********************************************************

% General Guide lines:
% Add \input{header.tex} to the first line of your document, e.g.:

% ***********************************************************
%\input{header.tex}
%\title{}
%\author{}
%
%\begin{document}
%\maketitle
%
%\end{document}
% ***********************************************************

% The LaTeX for Physicists Header has the following features:

% Sets font size to 9pt
% Includes commonly needed packages for beamer presentation 
% \v{ } makes bold vectors (\v is redefined to \vaccent)
% \uv{ } makes bold unit vectors with hats
% \gv{ } makes bold vectors of greek letters
% \abs{ } makes the absolute value symbol
% \avg{ } makes the angled average symbol
% \d{ }{ } makes derivatives (\d is redefined to \underdot)
% \dd{ }{ } makes double derivatives
% \pd{ }{ } makes partial derivatives
% \pdd{ }{ } makes double partial derivatives
% \pdc{ }{ }{ } makes thermodynamics partial derivatives
% \ket{ } makes Dirac kets
% \bra{ } makes Dirac bras
% \braket{ }{ } makes Dirac brackets
% \matrixel{ }{ }{ } makes Dirac matrix elements
% \grad{ } makes a gradient operator
% \div{ } makes a divergence operator (\div is redefined to \divsymb)
% \curl{ } makes a curl operator
% \={ } makes numbers appear over equal signs (\= is redefined to \baraccent)

% General LaTeX tips:

% Use "$ ... $" for inline equations
% Use "\[ ... \]" for equations on their own line
% Use "\begin{center} ... \end{center}" to center something
% Use "\includegraphics[width=?cm]{filename.eps}" for images - must compile to dvi then use dvipdfm from a batch file
% Use "\begin{multicols}{2} ... \end{multicols}" for two columns
% Use "\begin{enumerate} \item ... \end{enumerate}" for parts of physics exercises
% Use "\section*{ }" for sections without numbering
% Use "\begin{cases} ... \end{cases}" for piecewise functions
% Use "\mathcal{ }" for a caligraphic font
% Use "\mathbb{ }" for a blackboard bold font

\title[Short Title \hspace{4em}\insertframenumber/
\inserttotalframenumber]{~ \\ Semiclassical Botlzmann Formulation \\ in one dimensional phase space \\~} %the [whatever] appears in the footer on the right
\author[M. A. Diaz]{ Manuel A. Diaz, \\ National Taiwan Universty } %the [whatever] appears in the footer on the left
\date{\today}

\begin{document}

\begin{frame}
\maketitle
\end{frame}

\section*{About NTU}

\begin{frame}
\begin{figure}
 \centering
 \includegraphics[height = 40 mm,width = 80 mm,trim=1pt 1pt 1pt 1pt]{../KU-2013/NTU_pics/NTU_library.jpg}
 \caption{NTU's heart: the Main Library}
 \label{fig:NTU_library}
\end{figure}
\end{frame}

\begin{frame}
\begin{figure}
 \centering
 \includegraphics[height = 50 mm,width = 80 mm,trim=1pt 1pt 1pt 1pt]{../KU-2013/NTU_pics/NTU_main_gate.jpg}
 \caption{NTU's main entrance}
 \label{fig:NTU_main_gate}
\end{figure}
\end{frame}

\begin{frame}
\begin{figure}
 \centering
 \includegraphics[height = 50 mm,width = 80 mm,trim=1pt 1pt 1pt 1pt]{../KU-2013/NTU_pics/NTU_Taipei.jpg}
 \caption{NTU's main campus, located in the heart of Taipei City}
 \label{fig:NTU_Taipei}
\end{figure}
\end{frame}

\begin{frame} \frametitle{NTU Statistics}
 \begin{orangeitemize}
  \item Stablished
    \begin{itemize}
     \item Founded 1928
     \item Reorganized 1945
    \end{itemize}
  \item Academic Staff
    \begin{itemize}
     \item 1,793 (full time),
     \item 1,188 (joint and adjunct)
    \end{itemize}
  \item Student Population
    \begin{itemize}
     \item 17,106 (Undergraduates)
     \item 15,710 (Post Graduates)
    \end{itemize}
  \item Campus
    \begin{itemize}
     \item 1,600,000 $m^2 \approx$ 395.4 acres.
    \end{itemize}
 \end{orangeitemize}
 *according to 2012 statistics.
\end{frame}

\begin{frame}
\begin{figure}
 \centering
 \includegraphics[height = 50 mm,width = 80 mm,trim=1pt 1pt 1pt 1pt]{../KU-2013/NTU_pics/National_Palace_museum.jpg}
 \caption{Taiwan's National Palace Museum}
 \label{fig:National_Palace_museum}
\end{figure}
\end{frame}

\begin{frame}
\begin{figure}
 \centering
 \includegraphics[height = 50 mm,width = 80 mm,trim=1pt 1pt 1pt 1pt]{../KU-2013/NTU_pics/Taipei_city_view.jpg}
 \caption{A View of the Taipei's commercial District}
 \label{fig:Taipei_city_view}
\end{figure}
\end{frame}

\section*{Outline} 

\begin{frame} \frametitle{Outline}
  \begin{orangeitemize}
  % Today I want to pre-introduce three important concept to approach boltzmann equation.
  \item Three important concepst before starting solving Botlzmann Equation, 
  \begin{itemize}
   \item Phase-Space diagramas
   \item Probability functions and some general properties
   \item The kinetic Boltzmann Equation
  \end{itemize}
  % Of course for this small presentation I have no intention of been as complete possible, however for more background the following we can refer to excelent text on classical Mechanics, Statisticas and Derivation of the Boltzmann Equation. \\
  These concepts will be breafly discused here, more details can be pursued in \cite{morin2008introduction,flury1997first} and \cite{kremer2010introduction}.
  % Secondly, I'll introduce a,
  \item 1D formulation for classical Botlzmann Equation-BGK.
  % I'm assuming that Professor Yang, has already introduced the classical formulation.
  \item Nodal Discontinuous Galerkin Formulation of BBGK.
  \item Numerical Examples.
  \end{orangeitemize}
\end{frame}

\section{Motivation}

\begin{frame} \frametitle{Regimes of continuous and rarefied gases}
  Why we want to solve Boltzmann Equation?
  \begin{figure}
  \centering
  \includegraphics[height = 40 mm,width = 80 mm,trim=1pt 1pt 1pt 1pt]{../KU-2013/Intro_pics/Knudsen_number_ranges}
  \caption{Ranges of applicability for methods with respect to Kundsen number / Degree of rarefaction}
  \label{fig:Knudsen_number_ranges}
  \end{figure}
\end{frame}

\begin{frame}
 \begin{figure}
        \centering
        \begin{subfigure}[b]{0.30\textwidth}
                \centering
                \includegraphics[trim = 1mm 1mm 1mm 1mm,clip,width=\textwidth]{../KU-2013/Intro_pics/micro_gas_flows}
                \caption{Micro-gas Flows}
                \label{fig:micro_gas_flows}
        \end{subfigure}%
        ~ %add desired spacing between images, e. g. ~, \quad, \qquad etc.
          %(or a blank line to force the subfigure onto a new line)
        \begin{subfigure}[b]{0.30\textwidth}
                \centering
                \includegraphics[trim = 1mm 1mm 1mm 1mm,clip,width=\textwidth]{../KU-2013/Intro_pics/space_reentry}
                \caption{Space Reentry}
                \label{fig:space_reentry}
        \end{subfigure}
        \caption{Typical Application for the Classical Boltzmann Equation}
	\label{fig:Classical_Applications}
 \end{figure}
\end{frame}

\begin{frame}
 Why we want to solve Semiclassical Boltzmann Equation?
  \begin{figure}
        \centering
        \begin{subfigure}[b]{0.30\textwidth}
                \centering
                \includegraphics[trim = 1mm 1mm 1mm 1mm,clip,width=\textwidth]{../KU-2013/Motivation_pics/Electron_Flow}
                \caption{Electron Flows in Semiconductors}
                \label{fig:Electron_Flow}
        \end{subfigure}
        ~ %add desired spacing between images, e. g. ~, \quad, \qquad etc.
          %(or a blank line to force the subfigure onto a new line)
        \begin{subfigure}[b]{0.30\textwidth}
                \centering
                \includegraphics[trim = 1mm 1mm 1mm 1mm,clip,width=\textwidth]{../KU-2013/Motivation_pics/Plasma-lamp}
                \caption{Plasma Modelation}
                \label{fig:Plasma_lamp}
        \end{subfigure}
        ~ %add desired spacing between images, e. g. ~, \quad, \qquad etc.
          %(or a blank line to force the subfigure onto a new line)
        \begin{subfigure}[b]{0.30\textwidth}
                \centering
                \includegraphics[trim = 1mm 1mm 1mm 1mm,clip,width=\textwidth]{../KU-2013/Motivation_pics/HeatTransfer_compositeMaterials}
                \caption{Heat Transfer in composite materials}
                \label{fig:HeatTransfer_compositeMaterials}
        \end{subfigure}
        \caption{Typical Application for the Semi-classical Boltzmann Equation, Modelation of Quantum transport phenomena}
	\label{fig:Semiclassical_Applications}
 \end{figure}
\end{frame}

\section{Basic Concepts}

\begin{frame} \frametitle{Phase space}
We use a simple case to start familirizing with the concept of Phase Space. 
Conside and ideal pendulum, or any linear oscilatory model.
 \begin{figure}
        \centering
        \begin{subfigure}[b]{0.60\textwidth}
                \centering
                \includegraphics[trim = 30mm 35mm 30mm 35mm,clip,width=\textwidth]{../KU-2013/BasicConcepts_pics/idealPendulum}
        \end{subfigure}%
        \caption{Ideal Pendulum ( in a bad-hand-made depiction)}
        \label{fig:idealPendulum}
 \end{figure}
\end{frame}

\begin{frame}
How does phase space looks?
 \begin{figure}
        \centering
        \begin{subfigure}[b]{0.60\textwidth}
                \centering
                \includegraphics[trim = 30mm 35mm 30mm 35mm,clip,width=\textwidth]{../KU-2013/BasicConcepts_pics/phaseSpace_1d_01}
        \end{subfigure}%
        \caption{Depiction of phase space of one-spatial dimension}
        \label{fig:phaseSpace_1d_01}
 \end{figure}
\end{frame}

\begin{frame}
We can easily identify 4 key position of our linear oscilator as depicted below,
 \begin{figure}
        \centering
        \begin{subfigure}[b]{0.60\textwidth}
                \centering
                \includegraphics[trim = 30mm 35mm 30mm 35mm,clip,width=\textwidth]{../KU-2013/BasicConcepts_pics/phaseSpace_1d_02}
        \end{subfigure}%
        \caption{Depiction of oscilator in phase space}
        \label{fig:phaseSpace_1d_02}
 \end{figure}
\end{frame}

\begin{frame}
If we take discrete pictures of our oscilator and plot its momentum and x-position we would observe,
 \begin{figure}
        \centering
        \begin{subfigure}[b]{0.60\textwidth}
                \centering
                \includegraphics[trim = 30mm 35mm 30mm 35mm,clip,width=\textwidth]{../KU-2013/BasicConcepts_pics/phaseSpace_1d_03}
        \end{subfigure}%
        \caption{Depiction of phase space of one-spatial dimension}
        \label{fig:phaseSpace_1d_03}
 \end{figure}
 Can we think of another examples?
\end{frame}

\begin{frame}
How would a shock tube problem would look like in shape space?
 \begin{figure}
        \centering
        \begin{subfigure}[b]{0.60\textwidth}
                \centering
                \includegraphics[trim = 30mm 35mm 30mm 35mm,clip,width=\textwidth]{../KU-2013/BasicConcepts_pics/phaseSpace_shocktube}
        \end{subfigure}%
        \caption{Depiction a uniformly discrete shock-tube problem in phase space}
        \label{fig:phaseSpace_1d_04}
 \end{figure}
\end{frame}

\begin{frame} \frametitle{Probability Distribution Function}
 \grassgreenbox{What is a probability distribution function?}
{
  Any function can be probability distribution function (PDF) as long it full fills two basic requirements:
  \begin{grassgreenitemize}
  \item Is positive function.
  \item The integral over it's domain must be equal to one.
  \end{grassgreenitemize}
}
\end{frame}


\begin{frame}
However today presentation is bias in favor of the Normal PDF,
 \begin{align*}
  Normal_{PDF}(x,\sigma,\mu) = \frac{1}{\sqrt{2 \pi } \sigma } e^{-\frac{(x-\mu )^2}{2 \sigma ^2}}
 \end{align*}
 where $\mu$ is the mean or expectation and $\sigma$ is the standar deviation. 
\end{frame}

\begin{frame}
  \begin{figure}
        \centering
        \begin{subfigure}[b]{0.45\textwidth}
                \centering
                \includegraphics[height = 40 mm,width = 50 mm,trim = 1mm 1mm 1mm 1mm]{../KU-2013/Mathematica_pics/Gaussian_PDF}
                \caption{Influence of the $\sigma$ parameter}
                \label{fig:Gaussian_PDF}
        \end{subfigure}%
        ~ %add desired spacing between images, e. g. ~, \quad, \qquad etc.
          %(or a blank line to force the subfigure onto a new line)
        \begin{subfigure}[b]{0.45\textwidth}
                \centering
                \includegraphics[height = 40 mm,width = 50 mm,trim = 1mm 1mm 1mm 1mm]{../KU-2013/Mathematica_pics/Gaussian_PDF02}
                \caption{Influence of the $\mu$ parameter}
                \label{fig:Gaussian_PDF02}
        \end{subfigure}
        \caption{Normal Distribution Function}
	\label{fig:NormalDistributionFunction}
 \end{figure}  
\end{frame}

\begin{frame}
 Distribution function like Normal Distributions can provide a lot of mininum information. This information can be retrived by examming with detail it cumulants or Integration Moments
 \begin{align*}
  & M_0 = \int_{a}^{b} f(x) dx \\
  & M_1 = \int_{a}^{b} x f(x) dx \\
  & M_2 = \int_{a}^{b} x^2 f(x) dx \\
  & M_n = \int_{a}^{b} x^n f(x) dx
 \end{align*}
\end{frame}

\begin{frame}
 Of course Probability distribution function are not only one-dimensional. The same principles used is classical statistics can be extended to 2d and 3d case (Just the equations become more complicated). 
 For e.g. Normal distribution in 2D becomes:
 \begin{align*}
  & BiNormal_{PDF}(x,y,\mu_1,\mu_2,\sigma_1,\sigma_2) = \\
  & \frac{\exp \left(\frac{1}{2} \left(-\frac{\left(x-\mu _1\right) \left(-\mu _2 \rho  \sigma _1+\mu _1 \sigma _2-\sigma _2 x+\rho  \sigma _1 y\right)}{\left(\rho ^2-1\right) \sigma _1^2 \sigma _2}-\frac{\left(y-\mu _2\right) \left(-\mu _1 \rho  \sigma _2+\mu _2 \sigma _1+\rho  \sigma _2 x+\sigma _1 (-y)\right)}{\left(\rho ^2-1\right) \sigma _1 \sigma _2^2}\right)\right)}{2 \pi  \sqrt{\sigma _1^2 \sigma _2^2-\rho ^2 \sigma _1^2 \sigma _2^2}}
 \end{align*}
We can think about it as the tensor product of two normal PDFs.
\end{frame}

\begin{frame}
  Graphically we have,
  \begin{figure}
        \centering
        \begin{subfigure}[b]{0.45\textwidth}
                \centering
                \includegraphics[height = 40 mm,width = 50 mm,trim = 1mm 1mm 1mm 1mm]{../KU-2013/Mathematica_pics/Gaussian_PDF04}
                \caption{Isotropic vehavior}
                \label{fig:Gaussian_PDF04}
        \end{subfigure}%
        ~ %add desired spacing between images, e. g. ~, \quad, \qquad etc.
          %(or a blank line to force the subfigure onto a new line)
        \begin{subfigure}[b]{0.45\textwidth}
                \centering
                \includegraphics[height = 40 mm,width = 50 mm,trim = 1mm 1mm 1mm 1mm]{../KU-2013/Mathematica_pics/Gaussian_PDF03}
                \caption{Anisotropic}
                \label{fig:Gaussian_PDF03}
        \end{subfigure}
        \caption{BiNormal Distribution Function}
	\label{fig:BiNormalDistributionFunction}
 \end{figure}  
\end{frame}

\begin{frame} \frametitle{Towards the Botlzmann Equation}
  \begin{figure}
        \centering
        \begin{subfigure}[b]{0.30\textwidth}
                \centering
                \includegraphics[height = 40 mm,width = 30 mm,trim = 1mm 1mm 1mm 1mm]{../KU-2013/Intro_pics/james_clerk_maxwell}
                \caption{Maxwell}
                \label{fig:james_clerk_maxwell}
        \end{subfigure}%
        ~ %add desired spacing between images, e. g. ~, \quad, \qquad etc.
          %(or a blank line to force the subfigure onto a new line)
        \begin{subfigure}[b]{0.30\textwidth}
                \centering
                \includegraphics[height = 40 mm,width = 30 mm,trim = 1mm 1mm 1mm 1mm]{../KU-2013/Intro_pics/ludwing_eduard_boltzmann}
                \caption{Boltzmann}
                \label{fig:ludwing_eduard_boltzmann}
        \end{subfigure}
        \caption{Pioneers in of Statistical Mechanics}
	\label{fig:Maxwell-Boltzmann}
 \end{figure}
\end{frame}

\begin{frame}
How would a shock tube problem would look like in shape space?
 \begin{figure}
        \centering
        \begin{subfigure}[b]{0.60\textwidth}
                \centering
                \includegraphics[trim = 30mm 35mm 30mm 35mm,clip,width=\textwidth]{../KU-2013/BasicConcepts_pics/phaseSpace_shocktube}
        \end{subfigure}%
        \caption{Depiction a uniformly discrete shock-tube problem in phase space}
        \label{fig:shocktube_dx}
 \end{figure}
\end{frame}

\begin{frame} \frametitle{Introduction to Boltzmann Equation}
 Here we consider Botlzmann initial the molecules of a Gas with N molecules enclosed in a recipiente of volume V.
 Following Boltzmann work, the state of a gas in the phase-space is characterized by a density PDF namely $f(\vec{x},\vec{v},t)$.
 \begin{equation}
  f(\vec{x},\vec{v},t) d\vec{x} d\vec{v} = f(\vec{x},\vec{v},t) dx_1 dx_2 dx_3 dv_1 dv_2 dv_3 
 \end{equation}
  Which represents at time t, the number of molecules in the volume element with position vector within the range $x + dx$ and with velocity vectors within the range $v + dv$.
\end{frame}

\begin{frame}
How would a shock tube problem would look like in shape space?
 \begin{figure}
        \centering
        \begin{subfigure}[b]{0.60\textwidth}
                \centering
                \includegraphics[trim = 30mm 35mm 30mm 35mm,clip,width=\textwidth]{../KU-2013/BasicConcepts_pics/phaseSpace_shocktube_dx}
        \end{subfigure}%
        \caption{Depiction a uniformly discrete shock-tube problem in phase space}
        \label{fig:shocktube}
 \end{figure}
\end{frame}

% Introduction
\subsection{Semi-classical Boltzmann-BGK}

\begin{frame}
	\frametitle{Boltzmann-BGK Equation for gas flows}
	Boltzmann Equation (BE), derived from statistical mechanics and based on kinetic theory, describes the evolution of the velocity distribution function, $f(\vec{x},\vec{c},t)$, for rarefied gases in phase-space. In this presentatio, the collission operator is replaced by the BGK operator to avoid the mathematical difficulty cuased by the nonlinear integral collision term,
	\begin{equation}
	\frac{\partial{f}}{\partial{t}} +
	\vec{c}\bullet\frac{\partial{f}}{\partial{\vec{x}}} +
	\vec{F}\bullet\frac{\partial{f}}{\partial{\vec{c}}} = 
	\left( \frac{\delta f}{\delta t}\right )^{BGK}_{coll} = -\frac{1}{\tau}(f-f^{eq})
	\label{eq:classicalBBGK}
	\end{equation}
	where $\tau$ stands for the molecular collision relaxation time.
\end{frame}

\begin{frame}
	\frametitle{Semi-classical BE-BGK for gas flows}
	Uehling and Uhlenbeck \cite{PhysRev.43.552} generalized the Boltzmann Equation to be used with quantum statistics, again their collision operator is replaced by the relaxation concept debeloped by Bhatnagar, Gross and Krook; which leads to our semi-classical Boltzmann-BGK equation,
	\begin{equation}
	\frac{\partial{f}}{\partial{t}} +
	\frac{\vec{p}}{m}\bullet\frac{\partial{f}}{\partial{\vec{x}}} +
	\nabla_\vec{x}{\phi}\bullet\frac{\partial{f}}{\partial{\vec{p}}} = 
	\left( \frac{\delta f}{\delta t}\right )^{BGK}_{coll} = -\frac{1}{\tau}(f-f^{eq})
	\label{eq:semiclassicalBBGK}
	\end{equation}
	where $\vec{p}$ is the momentum at the space time, $m$ the particle mass, and $\phi$ is a mean potential field.
\end{frame}

\begin{frame}
	\frametitle{Semi-classical BE-BGK for gas flows}
	However, let $\vec{c} = \vec{p} / m$ be the particle velocity and $F = \nabla_\vec{x}{\phi}$, then we can re-write our semiclassical formulation in eq. \ref{eq:semiclassicalBBGK}, into an analog of its classical counter part \cite{Yang2013},
	\begin{equation}
	\frac{\partial{f}}{\partial{t}} +
	\vec{c}\bullet\frac{\partial{f}}{\partial{\vec{x}}} +
	\vec{F}\bullet\frac{\partial{f}}{\partial{\vec{c}}} = 
	\left( \frac{\delta f}{\delta t}\right )^{BGK}_{coll} = -\frac{1}{\tau}(f-f^{eq})
	\end{equation}
	where $\tau$ stands for the molecular collision relaxation time.
\end{frame}

\begin{frame}
	\frametitle{Classical Equilibrium Distribution function}
	Following the work of C.T. Hsu \cite{ISI:000303761300021} and J.C. Huang \cite{Huang2011261} the equilibrium distribution function in three-dimensions for the classical gas, $f^{eq}(\vec{x},\vec{c},t)$, is given by
	\begin{equation}
	f^{eq}=f^{eq}(\vec{x},\vec{c},t)=n \left( \frac{1}{2 \pi RT} \right)^{\frac{3}{2}} \exp\left({\frac{(\vec{c}-\vec{u})^2}{2 R T}}
\right)
	\label{eq:classical_feq}
	\end{equation}
	Where $n(\vec{x},t)$, $\vec{u}(\vec{x},t)$, $T(\vec{x},t)$, the density, mean velocity and temperature of the gas.
\end{frame}


\begin{frame}
	\frametitle{Semi-classical Equilibrium Distribution function}
	Following the work of Uehling \& Uhlenbeck \cite{PhysRev.43.552} the equilibrium for the semi-classical distribution function, $f^{eq}(\vec{x},\vec{c},t)$, is given by
	\begin{equation}
	f^{eq}=f^{eq}(\vec{x},\vec{c},t)=\frac{1}{(\frac{1}{z})\exp\left({\frac{m}{2 k_B T}(\vec{c}-\vec{u})^2}\right)+\theta}
	\label{eq:semiclassical_feq}
	\end{equation}
	Where $z(\vec{x},t)$, $\vec{u}(\vec{x},t)$, $T(\vec{x},t)$, the quemical potential, mean velocity and temperature of the gas; $\theta$ is a parameter that specifies the type of particle statistics we will using. Here are consider:
	\[
		\begin{cases}
		\theta = +1, 	& \text{Fermi-Dirac particles.} \\
		\theta = \ 0,	& \text{Maxwell-Boltzmann or classical particles.} \\
		\theta = -1, 	& \text{Bose-Einstein particles.}
		\end{cases}
	\]
	% i.e. we are will be solving Boltzmann Equation for Classical and Quantun Statisticas in a parallel maner.
\end{frame}
	
\begin{frame}
	\frametitle{Moments of Semi-classical BE}
	The first four moments of the distribution function are,
	\begin{eqnarray}
	\int \frac{d\vec{p}}{h^3} f(\vec{p},\vec{x},t) &=& \int f d^3 c = \rho \\
	\int \frac{d\vec{p}}{h^3} \frac{\vec{p}}{m} f(\vec{p},\vec{x},t) &=& \int \vec{c} f d^3 c = \rho \vec{u} \\
	\int \frac{d\vec{p}}{h^3} \frac{\vec{p}^2}{2m} f(\vec{p},\vec{x},t) &=& \int \frac{\vec{c}^2}{2} f d^3 c = \rho E \\
	\int \frac{d\vec{p}}{h^3} \frac{(\vec{p}-m\vec{u})}{2m} f(\vec{p},\vec{x},t) &=& \int \frac{(\vec{c}-\vec{u})^2}{2} f d^3 c = \rho e 
	\end{eqnarray}
	where $\rho(\vec{x},t)$,  $\vec{u}(\vec{x},t)$ and $e(\vec{x},t)$ are density, mean velocity and internal specific energy of the gas particles respectively. Note that total Energy density can be also defined as $\rho E = 1/2 \rho \vec{u}^2 + \rho e$.
\end{frame}

\begin{frame}
	\frametitle{Hydrodynamic Moments of SBE}
	In analogy, if we where to evaluate the hydrodynamic limit a semi-classical gas we have,
	\begin{eqnarray}
	\int \frac{d\vec{p}}{h^3} f^{eq}(\vec{p},\vec{x},t) &=& \int f^{eq}d^3c = \rho \\
	\int \frac{d\vec{p}}{h^3} \frac{\vec{p}}{m} f^{eq}(\vec{p},\vec{x},t) &=& \int \vec{c} f^{eq}d^3c = \rho \vec{u} \\
	\int \frac{d\vec{p}}{h^3} \frac{\vec{p}^2}{2m} f^{eq}(\vec{p},\vec{x},t) &=& \int \vec{c}^2 f^{eq}d^3c = \rho E \\
	\int \frac{d\vec{p}}{h^3} \frac{(\vec{p}-m\vec{u})}{2m} f^{eq}(\vec{p},\vec{x},t) &=& \int (\vec{c}-\vec{u})^2 f^{eq}d^3c = \rho e 
	\end{eqnarray}
	Note also that we are using a Gauss Hermite quadrature due to the gaussian-like behavior of the semi-classical statistics at low temperatures.
\end{frame}


% Bibliography
\section{References}
\subsection{Bibliography}

\begin{frame}[allowframebreaks]
	\frametitle{References}
	\bibliographystyle{unsrt}
	%\bibliographystyle{amsalpha}
	\bibliography{Presentation_refs,DDOM_refs}
\end{frame}

\end{document}