\section{Implementing DDOM}
\subsection{The Method}

\begin{frame}
	\frametitle{Graphically speaking}
	
		\begin{figure}
			\centering
				\includegraphics[width=0.90\textwidth]{MB_DDOMvsDOM2}
			\caption{Quadrature point distribution on thermaly normalized distribution with galilean transformation vs. traditional method}
			\label{fig:MB_DDOMvsDOM2}
		\end{figure}
	
\end{frame}

\begin{frame}
	\frametitle{Graphically speaking}
		
		\begin{figure}
			\centering
				\includegraphics[width=0.90\textwidth]{MB_DDOMvsDOM1}
			\caption{Quadrature point distribution on thermaly normalized distribution with galilean transformation vs. traditional method}
			\label{fig:MB_DDOMvsDOM1}
		\end{figure}
		
\end{frame}

\begin{frame}
	\frametitle{Graphically speaking}

		\begin{figure}
			\centering
				\includegraphics[width=0.90\textwidth]{MB_DDOMvsDOM3}
			\caption{Quadrature point distribution on thermaly normalized distribution with galilean transformation vs. traditional method}
			\label{fig:MB_DDOMvsDOM3}
		\end{figure}
		
\end{frame}

\begin{frame}
	\frametitle{DDOM Basic Idea}
	To circumbent the problems professor C.T. Hsu, et al. \cite{ISI:000303761300021,Hsu201239} noted that the exponential in the equilibrium distribution function (\ref{eq:classical_feq}) can be made independent of $\vec{u}(\vec{x},t)$ and $T(\vec{x},t)$. by using the transformation,
	\begin{equation}
	\vec{C}^{*}  = \frac{(\vec{c}-\vec{u})}{\sqrt{\frac{2 k_B T}{m}}} = \frac{(\vec{c}-\vec{u})}{a}
	\label{eq:transformation}
	\end{equation}
	Where $\vec{C}^{*}$ will be defined as fixed discrete velocity points and as the abscissas of our GH quadrature.
\end{frame}

\begin{frame}
	\frametitle{1d SB-BGK}
	Consider a one-spatial dimensional Boltzmann-BKG equation and neglect any external force, Eq. (\ref{eq:simplyfied_semiclassicalBBGK}) reduces to,
	\begin{equation}
	\frac{\partial f(x,c,t)}{\partial t} + c \frac{\partial f(x,c,t)}{\partial x} = -\frac{f - f^{eq}}{\tau}
	\label{eq:sbbgk1d}
	\end{equation}
	and consider our semi-classical equilibrium distribution function,
	\begin{equation}
	f^{eq} (x,c,t) = \frac{1}{(1/z) exp( \frac{ m \left | c - u \right |^2}{2 k_B T} )+\theta}
	\end{equation}
	Where in traditional DOM $c$, is fixed values of discrete velocity poins and GH abscissas.
\end{frame}

\begin{frame}
	\frametitle{Integral Moments of $f$}
	Again, the first four moments fo the the distribution function are defined as,
	\begin{eqnarray}
	\int f(x,c,t) d c  &=& \rho \\
	\int c f(x,c,t) d c  &=& \rho \vec{u} \nonumber \\
	\frac{1}{2} \int c^2 f(x,c,t) d c  &=& \rho E \nonumber \\
	\frac{1}{2} \int (c-u)^2 f(x,c,t) d c  &=& \rho e \nonumber
	\end{eqnarray}
\end{frame}

\begin{frame}
	\frametitle{1d SB-BGK}
	Using DDOM by \cite{Hsu201239}, $c = c(x,t)$, and are computed for every point in our domain using the relation $c(x,t) = a(x,t) C^* + u(x,t)$, moreover it also implies,
	\begin{equation}
	\frac{\partial f(x,t)}{\partial t} + c(x,t) \frac{\partial f(x,t)}{\partial x} = -\frac{f - f^{eq}}{\tau}
	\label{eq:sbbgk1d_DDOM}
	\end{equation}
	and our semi-classical equilibrium distribution function also becomes,
	\begin{equation}
	f^{eq} (x,t) = \frac{1}{(1/z) exp( C^* )+\theta}
	\end{equation}
\end{frame}

\begin{frame}
	\frametitle{Thermal Normalization \& Galilean trasformation of the 1d Equations}
	Our thermal normalized definition of $\vec{C}^*$ in one-dimension goes a follows,
	\begin{align*}
	& C^* = \frac{(c-u)}{\sqrt{\frac{2k_BT}{m}}} & a = \sqrt{\frac{2k_BT}{m}} 
	\end{align*}
\end{frame}
	

\begin{frame}
	\frametitle{DDOM}
	The term $(\frac{\partial C^*}{\partial c} )$ is then introduced to perform this change of variables inside the moment integrals
	\begin{eqnarray}
		\int f^* \left(\frac{\partial C^*}{\partial c} \right ) d c  &=& \rho \\
		\int c f^* \left(\frac{\partial C^*}{\partial c} \right ) d c &=& \rho u \nonumber \\
		\frac{1}{2} \int c^2 f^* \left(\frac{\partial C^*}{\partial c} \right ) d c  &=& \rho E \nonumber \\
		\frac{1}{2} \int (c-u)^2 f^* \left(\frac{\partial C^*}{\partial c} \right ) d c &=& \rho e \nonumber
	\end{eqnarray}
	the term $\left(\frac{\partial C^*}{\partial c} \right )$ is identified as the Jacobian, $J$ and $c=c(x,t)$.
\end{frame}

\begin{frame}
	\frametitle{Normalization of 1d Equations}
	We proceed to normalize our one-dimension SB-BKG equation by following Muljadi and Yang \cite{Yang2013} procedure. Namely define, 
	\begin{align*}
	V_\infty &= \sqrt{\frac{2k_BT_\infty}{m}}, & t_\infty &= \frac{L}{V_\infty}
	\end{align*}
	Here $L$ $(L = V_\infty t_\infty)$ is identified as the characterisitc length.
\end{frame}

\begin{frame}
	Multiplying to equation (\ref{eq:sbbgk1d}) and it moments by $(\frac{V_\infty}{V_\infty})(\frac{t_\infty}{t_\infty})$, the following dimensionles relations follows,
	\begin{align*}
	(\hat t, \hat \tau) &= \frac{(t,\tau)}{t_\infty}, & (\hat u_x,\hat c_x,) = \frac{(u_x,c_x)}{V_\infty}, \\
	\hat T &= \frac{T}{T_\infty}, & \hat E = E / \left(\frac{m^3V_\infty^4}{h^2} \right ), \\
	\hat f &= f, &\hat {\rho u_x} = \rho u_x / \left(\frac{m^2V_\infty^3}{h^2} \right ), \\
	\hat x &= \frac{x}{L}, &\hat \rho = \rho / \left(\frac{m^2V_\infty^2}{h^2} \right ).
	\end{align*}
\end{frame}

\begin{frame}
	\frametitle{1d Normalized Equation}
	Our normalized one-dimensional Boltzmann-BKG equation becomes,
	\begin{equation}
	\frac{\partial \hat {f^*}}{\partial \hat t} + \hat c \frac{\partial \hat {f*}}{\partial \hat x} 
		= -\frac{\hat f - \hat {f^{eq}}}{\hat \tau}
	\end{equation}
	and the normalized semi-classical equilibrium distribution function is,
	\begin{equation}
	\hat f^{eq} (\hat x,\hat c,\hat t) = \frac{1}{(1/z) exp(\left | \hat c - \hat u \right |^2 / \hat T)+\theta}
\end{equation}
\end{frame}

\begin{frame}
	\frametitle{Finally apply DOM}
	Applying Discrete Ordinate Method, we render the following set of dynamic Boltzmann-BGK equations,
	\begin{equation}
	\frac{\partial {f_\sigma^*}}{\partial t} + c_\sigma \frac{\partial f_\sigma^*}{\partial x} 
		= -\frac{f_\sigma - f_\sigma^{eq}}{\tau}, \;\;\; \text{where } \sigma = 1,\dots,N_v
	\end{equation}
	Note that $c_\sigma(x,t) = a(x,t) C_\sigma^* + u(x,t)$. We also must render their companion semi-classical equilibrium distribution functions,
	\begin{equation}
	f_\sigma^{eq} (x,c_\sigma,t) = \frac{1}{(1/z) exp(\frac{c_\sigma-u}{T})^2 + \theta}
	\end{equation}
\end{frame}

\begin{frame}
	\frametitle{DDOM}
	And the moment integrals become,
	\begin{eqnarray}
		\sum_\sigma J W_\sigma \exp(c_\sigma^2) f^*(x,c_\sigma,t) &=&  \rho, \\
		\sum_\sigma J c_\sigma W_\sigma \exp(c_\sigma^2) f^*(x,c_\sigma,t) &=& \rho u, \nonumber \\
		\sum_\sigma J \frac{c_\sigma^2}{2} W_\sigma \exp(c_\sigma^2) f^*(x,c_\sigma,t) &=& \rho E, \nonumber \\
		\sum_\sigma J \frac{(c_\sigma-u)^2}{2} W_\sigma \exp(c_\sigma^2) f^*(x,c_\sigma,t) &=& \rho e  \nonumber
	\end{eqnarray}
	Where J is the Jacobian. Here $J = \sqrt{T(x,t)}$ .
\end{frame}
