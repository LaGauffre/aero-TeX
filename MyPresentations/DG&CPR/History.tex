\section*{History and Facts from the method}

\begin{frame}[allowframebreaks]
 Facts in Wang and Gao's publication on LCP method in 2009, \cite{Wang&Gao2009}.
 \begin{itemize}
  \item Huynh's Flux Reconstruction i
  \item the surge of activities regarding higher order methods in the CFD comunity is due to fact that some of this methods provide higher accuracy with less CPU time than the 1st or 2nd order methods for problems with both complex physics and geometry.
  \item for compressible flow simulation in aerospace applications, the leader of these high-order methods is arguably the Discontinuous Galerkin (DG) method.
  \item Van den Abeele et al. and Huynh independently showed that the SG and SD methods are independent of how solution points are selected. Only the flux points determine the characteristics of the methods. Therefore one does not need to use a staggered grid at all, and the solutions and flux points can coincide to improve efficiency.
  \item SV and SD methods allow large time steps than the DG method for stability.
  \item it is more convenient to considered the correction function $g(x)$ on computational domain $[-1,1]$
  \item if $g(x)$ is the right Radau Polynomial, the resulting scheme is actually the DG Method!
  \item if $g(x)$ is the Chebyshev-Lobatto points as its interior roots, the resulting scheme is the SG method or the SD/SV method in 1D. The Scheme, however, is mildy unstable, which was also found by Van den Abeele et al.
  \item if $g(x)$ has the Legendre-Gauss points as its interior roots, the scheme is stable. This suggest that using the legendre-Gauss points and the two end points as flux points resutls in a stable SG method in 1D.
  \item if $g(x)$ has a vanishing derivative at the right boundary as well as the interior legendre-lobatto points, the method results in a remarkably simple, yet stable scheme if one chooses the legendre-lobatto points as the solution points since the corrections for the interior solution points vanish. This scheme was called the $g_2$ scheme in Huynh 2007 paper. 
 \end{itemize}
\end{frame}

\begin{frame}[allowframebreaks]
 Facts from Huynh 2011 publication, \cite{Huynh2011}.
 \begin{itemize}
  \item Low order methods are Generally more robust and reliable; as a result the are employed in practical calculations.
  \item In contrast High-order methods are more complicated, less robust and expensive but can provide more accurate solutions.
  \item DG schemes using polynomails for the nodal point as basis functions known as nodal DG methods can be found in Hestaven and Warburton 2008.
  \item Another class of popular high-order methods introduced as Kopriva and Kolias (1996) as staggered grid chebyshev multidomain methods were extended to triangular meshes by Liu, Vinokur and Wang (2006) and were domain methods, they were extended called spectal difference methods. 
  \item with the introduction of Flux Reconstruction by Huynh in 2007, a unification of high order method such as DG, SG/SD as well as spectral volume were unified.
  \item FR versions are generally simpler and more economical than previous methods.
  \item One parameter family was found to be stable and super convergent vie Fourier Analysis. This framework was applied to solve diffusion problems in Huyng 2009.
  \item FR can be extended to quadrilateral mesh in straighforward manner via tensor products However for unstructured meshed this becomes nontrivial: tenso product no longer applies.
  \item Wang and Gao showed in 2009 that the derivate of the correction can be extended without the reconstruction concept. The resulting method was named Lifting Collocation Penalty (LCP).
  \item Wang and Gao provided encouraging numerical solutions for the 2D Euler and the Navier-Stokes equations on meshes of mixed elements (including quadrilateral and triangules). 3D simulation where provide by Haga, Gao and Wang in 2010.
  \item Due to the tight connection between FR and LCP, the involved authors combined the names and call them the CPR method (Correction via Reconstruction or Collocation Penalty via Reconstruction). 
 \end{itemize}  
\end{frame}

\begin{frame}[allowframebreaks]
 Facts in Yu and Wang review and comparison of CPR with other methods, 2013 \cite{Yu&Wang2013}. 
 \begin{itemize}
  \item Discontinous Galerkin (DG) methods are arguably nowadays the most and popular method in the Computational Fluid Dynamics (CFD) comunity due to the to its potentail of delivering high-accuracy (order of accuracy $\ge$ 3) with lower computational cost than the low order methods for problems involiving complex physics and geometries, such as aero-acoustic noise prediction and vortex dominated flows.
  \item However, DG methods are not unique. In fact there are two most kinds. The first is known as Quadrature DG methods (QDG) and was introduced first in 1989 by Cockbun and Shu \cite{Cockburn&Shu1989a,CockburnLin&Shu1989b}; A second version, the Nodal DG (NDG), was introduced by Hestaven and Warburton \cite{hesthaven2008}.
  \item Generally speacking QDG is more expensive that NDG, however both deliver the identical accuracy. For non-linear conservation Laws, QDG is more accurate as NDG may run intro problems associated with alias errors. 
  \item In order to avoid the explicit numerical integrals, some DG-like but differential-form based high-order methods have been developed. One succesfull high order approach of this kind is the staggered-grid (SG) multi-domain spectral method or the spectral difference method. Even on quadrilaterals or hexahedral elements, the SG/DG methods are one dimensional by design, even for high order elements. This is very significant as the computational cost is dramatically less than the DG approach in multiple dimensions.
  \item Later is was found that the computational cost of SG/SD can be further decreased as Huynh \cite{Huynh2007} and Van den Abbele et al \cite{Abbele} confirmed that SG/SD method only depends on the location of the flux points, and the staggered grid configuration, i.e. Staggered distribution of solutions points and flux points is not necessary. This suggest that the solution and flux points can coincide to further improve computational efficiency.
  \item In 2007, Huynh introduced the collocation based flux reconstruction (FR) approach for one-dimensional conservation laws
 the FR is similar to SG/SD in philosophy, but allows the flux polynomial to be reconstructed using minimization procedure.
  \item Depending on the how is the minimization procedure is defined, FR approach can unify existing methods such as DG, SG/SD and SV. 
  \item Moreover it can lead to several new schemes with favorable properties.
  \item Considering that LCP and FR approaches result in the same final formulation, Huynh and Wang decided to unify the FR and LCP under the Correction Procedure via Reconstruction or CPR in short.
 \end{itemize}
\end{frame}

\begin{frame}[allowframebreaks]
 Facts found in Shu's review on CPR, 2013 \cite{DuShu&Zhang2013}.
 \begin{itemize}
  \item WENO limiter is implemented on top of the CPR framework. Therefore the original CPR framework and its conservation properties are preserved.  
  \item CPR framework is conservative.
  \item CPR framework does not in general preserver positivity of the solution.
  \item There has been a surge of research activities around high-order methods capable of solving such as discontinuous galerkin (DG), stagered-grid (SG) and spectral volume (SV) and spectral difference (SD) methods.
  \item SD method can be viewed as the extension (or generalization) of the SG method to triangular meshes.
  \item CPR has some nice properties. By choosing the certain correction functions, the CPR framework can unify several well-know methods such as the DG, SG or the SV/SD methods and leads to simplified versions of these methods, at least for linear equations.
  \item CPR is based on Nodal form with an element-wise discontinuous piecewise polynomial solution space. According to Shu, it can be considered as the DG with a suitable numerical quadrature for the integration of nonlinear terms.
  \item In SG/SD method two groups of points are need, however in CPR only one group of grid points namely, the solutions points are needed.
  \item CPR framework is easier to understand and more eficient to implement.
  \item However CPR is only high-order linear scheme. It will generate spurious oscillations for problems containing strong discontinuities.
  \item However the CPR framework with the WENO limiter does not in general satisfy the positivity property.
 \end{itemize}
\end{frame}