\section{Numerical Results}
\subsection{Linear Advection without limiter}

\begin{frame} \frametitle{Linear Advection equations}
	Let us concider a linear advecation equation,
	\begin{figure}
		\centering
		\includegraphics[width=0.45\textwidth]{../DG&CPR/images/MDG_advection}
		\caption{Scalar advection of a sine wave}
		\label{fig:LinearAdvection}
	\end{figure}
\end{frame}

\begin{frame}
	Consider,
	\begin{align}
		\pd{u}{t} + \pd{u}{x} =& 0, \;\;\; \text{ for } x\in[0,2\pi] \\
		u_0 =& \sin x
	\end{align}
	and periodic boundary conditions. \\
	Let us make a comparison between Modal DG, Nodal DG, and CPR by using polynomials of degree: 3,4 and 5.
	Also we wish to compare the convergence of the solution by using a mesh refiment. Choose to divide the domain into 20, 40, 80, 160 and 320 elements. We compare each of the methods by using strong stability preserving Runge Kutta time integration of 3rd order and 3-stages.
\end{frame}

\begin{frame} \frametitle{Linear Advection equations}
	\begin{figure}
        \centering
        \begin{subfigure}[b]{0.31\textwidth}
                \centering
                \includegraphics[trim = 5mm 35mm 30mm 75mm,clip,width=\textwidth]{../DG&CPR/MDG/TestMDGresults}
                \caption{Modal DG}
                \label{fig:TestMDGresults}
        \end{subfigure}%
				~
				%~ %add desired spacing between images, e. g. ~, \quad, \qquad etc.
          %(or a blank line to force the subfigure onto a new line)
        \begin{subfigure}[b]{0.31\textwidth}
                \centering
                \includegraphics[trim = 5mm 35mm 30mm 75mm,clip,width=\textwidth]{../DG&CPR/NDG/TestNDGresults}
                \caption{Nodal DG}
                \label{fig:TestNDGresults}
        \end{subfigure}
				~
        \begin{subfigure}[b]{0.31\textwidth}
								\centering
                \includegraphics[trim = 5mm 35mm 30mm 75mm,clip,width=\textwidth]{../DG&CPR/CPR/TestCPRresults}
                \caption{CPR/FR}
                \label{fig:TestCPRresults}
        \end{subfigure}
				%
				\caption{Comparison of methods for linear advection problem}
				\label{fig:compareLinearAdvection_p3p4p5}
	\end{figure}
	Here, for the cases with polynomial degree of 3,4 and 5; the choosen CFL condition is set to 0.01, 0.001 and 0.0001 respectively.
	Notice that Nodal DG and CPR methods have identical accuracy in the linear case!
\end{frame}

\begin{frame} \frametitle{Linear Advection equations}
The record of CPU time employed in every case for each method was recorded,
	\begin{figure}
        \centering
        \begin{subfigure}[b]{0.31\textwidth}
                \centering
                \includegraphics[trim = 5mm 135mm 90mm 15mm,clip,width=\textwidth]{../DG&CPR/MDG/TestMDGresults}
                \caption{Modal DG}
                \label{fig:TestMDGresults_CPUtime}
        \end{subfigure}%
				~
				%~ %add desired spacing between images, e. g. ~, \quad, \qquad etc.
          %(or a blank line to force the subfigure onto a new line)
        \begin{subfigure}[b]{0.31\textwidth}
                \centering
                \includegraphics[trim = 5mm 135mm 90mm 15mm,clip,width=\textwidth]{../DG&CPR/NDG/TestNDGresults}
                \caption{Nodal DG}
                \label{fig:TestNDGresults_CPUtime}
        \end{subfigure}
				~
        \begin{subfigure}[b]{0.31\textwidth}
								\centering
                \includegraphics[trim = 5mm 135mm 90mm 15mm,clip,width=\textwidth]{../DG&CPR/CPR/TestCPRresults}
                \caption{CPR/FR}
                \label{fig:TestCPRresults_CPUtime}
        \end{subfigure}
				%
				\caption{Comparison of methods using Linear Advection Equation}
				\label{fig:compareLinearAdvection_p3p4p5_CPUtime}
	\end{figure}
	Notice that fastes implementation is by Nodal DG. It is not surpricing that Modal DG is the slowest due to the conversion of nodal to modal data in every time step. Also Notice that the CPR implementation is not far away from the Nodal DG.
\end{frame}

\begin{frame} \frametitle{Linear Advection equations}
What happend if we choose to increase the degree of our polynomail beyong the RK scheme accuracy?\\
Let us use a SSP-RK of 4th order with 5-stages and set polynomails degree to 6,
	\begin{figure}
        \centering
        \begin{subfigure}[b]{0.31\textwidth}
                \centering
                \includegraphics[trim = 20mm 12mm 170mm 12mm,clip,width=\textwidth]{../DG&CPR/MDG/MDG_IC2_p5}
                \caption{Modal DG}
                \label{fig:MDG_IC2_p5}
        \end{subfigure}%
				~
				%~ %add desired spacing between images, e. g. ~, \quad, \qquad etc.
          %(or a blank line to force the subfigure onto a new line)
        \begin{subfigure}[b]{0.31\textwidth}
                \centering
                \includegraphics[trim = 20mm 12mm 170mm 12mm,clip,width=\textwidth]{../DG&CPR/NDG/NDG_IC2_p5}
                \caption{Nodal DG}
                \label{fig:NDG_IC2_p5}
        \end{subfigure}
				~
        \begin{subfigure}[b]{0.31\textwidth}
								\centering
                \includegraphics[trim = 20mm 12mm 170mm 12mm,clip,width=\textwidth]{../DG&CPR/CPR/CPR_IC2_p5}
                \caption{CPR/FR}
                \label{fig:CPR_IC2_p5}
        \end{subfigure}
				%
				\caption{Solution obtained with 10 elements and CFL = 0.001}
				\label{fig:compareLinearAdvection_p5}
	\end{figure}
\end{frame}

\begin{frame} \frametitle{Linear Advection equations}
What happend if we choose to increase the degree of our polynomail beyong the RK scheme accuracy?\\
Let us use a SSP-RK of 4th order with 5-stages and set polynomails degree to 6,
	\begin{figure}
        \centering
        \begin{subfigure}[b]{0.31\textwidth}
                \centering
                \includegraphics[trim = 170mm 12mm 20mm 12mm,clip,width=\textwidth]{../DG&CPR/MDG/MDG_IC2_p5}
                \caption{Modal DG}
                \label{fig:MDG_IC2_p5_averages}
        \end{subfigure}%
				~
				%~ %add desired spacing between images, e. g. ~, \quad, \qquad etc.
          %(or a blank line to force the subfigure onto a new line)
        \begin{subfigure}[b]{0.31\textwidth}
                \centering
                \includegraphics[trim = 170mm 12mm 20mm 12mm,clip,width=\textwidth]{../DG&CPR/NDG/NDG_IC2_p5}
                \caption{Nodal DG}
                \label{fig:NDG_IC2_p5_averages}
        \end{subfigure}
				~
        \begin{subfigure}[b]{0.31\textwidth}
								\centering
                \includegraphics[trim = 170mm 12mm 20mm 12mm,clip,width=\textwidth]{../DG&CPR/CPR/CPR_IC2_p5}
                \caption{CPR/FR}
                \label{fig:CPR_IC2_p5_averages}
        \end{subfigure}
				%
				\caption{Solution obtained with 10 elements and CFL = 0.001}
				\label{fig:compareLinearAdvection_p5averages}
	\end{figure}
\end{frame}

\begin{frame} \frametitle{Linear Advection equations}
What happend if we choose to increase the degree of our polynomail beyong the RK scheme accuracy?\\
Let us use a SSP-RK of 4th order with 5-stages and set polynomails degree to 7,
	\begin{figure}
        \centering
        \begin{subfigure}[b]{0.31\textwidth}
                \centering
                \includegraphics[trim = 20mm 12mm 170mm 12mm,clip,width=\textwidth]{../DG&CPR/MDG/MDG_IC2_p7}
                \caption{Modal DG}
                \label{fig:MDG_IC2_p7}
        \end{subfigure}%
				~
				%~ %add desired spacing between images, e. g. ~, \quad, \qquad etc.
          %(or a blank line to force the subfigure onto a new line)
        \begin{subfigure}[b]{0.31\textwidth}
                \centering
                \includegraphics[trim = 20mm 12mm 170mm 12mm,clip,width=\textwidth]{../DG&CPR/NDG/NDG_IC2_p7}
                \caption{Nodal DG}
                \label{fig:NDG_IC2_p7}
        \end{subfigure}
				~
        \begin{subfigure}[b]{0.31\textwidth}
								\centering
                \includegraphics[trim = 20mm 12mm 170mm 12mm,clip,width=\textwidth]{../DG&CPR/CPR/CPR_IC2_p7}
                \caption{CPR/FR}
                \label{fig:CPR_IC2_p7}
        \end{subfigure}
				%
				\caption{Solution obtained with 10 elements and CFL = 0.001}
				\label{fig:compareLinearAdvection_p7}
	\end{figure}
\end{frame}

\begin{frame} \frametitle{Linear Advection equations}
What happend if we choose to increase the degree of our polynomail beyong the RK scheme accuracy?\\
Let us use a SSP-RK of 4th order with 5-stages and set polynomails degree to 7,
	\begin{figure}
        \centering
        \begin{subfigure}[b]{0.31\textwidth}
                \centering
                \includegraphics[trim = 170mm 12mm 20mm 12mm,clip,width=\textwidth]{../DG&CPR/MDG/MDG_IC2_p7}
                \caption{Modal DG}
                \label{fig:MDG_IC2_p7_averages}
        \end{subfigure}%
				~
				%~ %add desired spacing between images, e. g. ~, \quad, \qquad etc.
          %(or a blank line to force the subfigure onto a new line)
        \begin{subfigure}[b]{0.31\textwidth}
                \centering
                \includegraphics[trim = 170mm 12mm 20mm 12mm,clip,width=\textwidth]{../DG&CPR/NDG/NDG_IC2_p7}
                \caption{Nodal DG}
                \label{fig:NDG_IC2_p7_averages}
        \end{subfigure}
				~
        \begin{subfigure}[b]{0.31\textwidth}
								\centering
                \includegraphics[trim = 170mm 12mm 20mm 12mm,clip,width=\textwidth]{../DG&CPR/CPR/CPR_IC2_p7}
                \caption{CPR/FR}
                \label{fig:CPR_IC2_p7_averages}
        \end{subfigure}
				%
				\caption{Solution obtained with 10 elements and CFL = 0.001}
				\label{fig:compareLinearAdvection_p7averages}
	\end{figure}
\end{frame}

\subsection{Nonlinear Advection without limiter}

\begin{frame} \frametitle{Burgers' equation}
	Let us now concider Burgers' equation,
	\begin{figure}
		\centering
		\includegraphics[width=0.45\textwidth]{../DG&CPR/images/NDG_Burgers}
		\caption{Nonlinear advection of a sine wave}
		\label{fig:NonlinearAdvection}
	\end{figure}
\end{frame}

\begin{frame}
	Consider,
	\begin{align}
		\pd{u}{t} +& \pd{f(u)}{x} = 0, \;\;\; \text{ for } x\in[0,2\pi] \\
		f(u) =& \frac{u^2}{2} \;\; \text{and} \;\; u_0 = \frac{1}{2} + \sin x
	\end{align}
	and periodic boundary conditions. \\
	Let us make a comparison between Modal DG, Nodal DG, and CPR by using polynomials of degree 3,4 and 5. However because of the lack of a limiter accuracy of the methods cannot be measured. Instead we wish to understand how each method reacto to discontinuities. 
	In the following tests, we employ a strong stability preserving Runge Kutta time integration of 4th order and 5-stages.
\end{frame}

\begin{frame} \frametitle{Linear Advection equations}
Let us plot the L1 Error for all the nodes in the domain for and final time of 0.95 and CFL = 0.001,
	\begin{figure}
        \centering
        \begin{subfigure}[b]{0.31\textwidth}
                \centering
                \includegraphics[trim = 5mm 1mm 5mm 1mm,clip,width=\textwidth]{../DG&CPR/MDG/TestMDGresults2}
                \caption{Modal DG}
                \label{fig:TestMDGresults2}
        \end{subfigure}%
				~
				%~ %add desired spacing between images, e. g. ~, \quad, \qquad etc.
          %(or a blank line to force the subfigure onto a new line)
        \begin{subfigure}[b]{0.31\textwidth}
                \centering
                \includegraphics[trim = 5mm 1mm 5mm 1mm,clip,width=\textwidth]{../DG&CPR/NDG/TestNDGresults2}
                \caption{Nodal DG}
                \label{fig:TestNDGresults2}
        \end{subfigure}
				~
        \begin{subfigure}[b]{0.31\textwidth}
								\centering
                \includegraphics[trim = 5mm 1mm 5mm 1mm,clip,width=\textwidth]{../DG&CPR/CPR/TestCPRresults2}
                \caption{CPR/FR}
                \label{fig:TestCPRresults2}
        \end{subfigure}
				%
				\caption{Comparison of Errors using Burgers' equation}
				\label{fig:compareLinearAdvection_p3p4p5}
	\end{figure}
	Similar partern can be obtained with any degree of base fucntions polynomials. Notice the total loose of accuaracy in the neighbourhood of the discontinuity.
\end{frame}

\begin{frame} \frametitle{Burgers' equations}
The figures below depict the behavior of each methods at a time 1.5, for the fully developed discontinuity. Here we use 3rd order base polynomials with 20 elements and CFL = 0.02.
	\begin{figure}
        \centering
        \begin{subfigure}[b]{0.31\textwidth}
                \centering
                \includegraphics[trim = 20mm 12mm 170mm 12mm,clip,width=\textwidth]{../DG&CPR/MDG/MDG_IC3_p3}
                \caption{Modal DG}
                \label{fig:MDG_IC3_p3}
        \end{subfigure}%
				~
				%~ %add desired spacing between images, e. g. ~, \quad, \qquad etc.
          %(or a blank line to force the subfigure onto a new line)
        \begin{subfigure}[b]{0.31\textwidth}
                \centering
                \includegraphics[trim = 20mm 12mm 170mm 12mm,clip,width=\textwidth]{../DG&CPR/NDG/NDG_IC3_p3}
                \caption{Nodal DG}
                \label{fig:NDG_IC3_p3}
        \end{subfigure}
				~
        \begin{subfigure}[b]{0.31\textwidth}
								\centering
                \includegraphics[trim = 20mm 12mm 170mm 12mm,clip,width=\textwidth]{../DG&CPR/CPR/CPR_IC3_p3}
                \caption{CPR/FR}
                \label{fig:CPR_IC3_p3}
        \end{subfigure}
				%
				\caption{Comparison of methods using Burgers' equation}
				\label{fig:compareBurgers_p3}
	\end{figure}
	Notice that The three methods manage to converge to the solution at the specified time, however Nodal DG seems to experiment high runge phenomenon at the right side of the discontinuity, and the phenomenon seems to increase with the increase of the polynomials degree.
\end{frame}

\begin{frame} \frametitle{Burgers' equation}
	\begin{figure}
        \centering
        \begin{subfigure}[b]{0.31\textwidth}
                \centering
                \includegraphics[trim = 170mm 12mm 20mm 12mm,clip,width=\textwidth]{../DG&CPR/MDG/MDG_IC3_p3}
                \caption{Modal DG}
                \label{fig:MDG_IC3_p3_averages}
        \end{subfigure}%
				~
				%~ %add desired spacing between images, e. g. ~, \quad, \qquad etc.
          %(or a blank line to force the subfigure onto a new line)
        \begin{subfigure}[b]{0.31\textwidth}
                \centering
                \includegraphics[trim = 170mm 12mm 20mm 12mm,clip,width=\textwidth]{../DG&CPR/NDG/NDG_IC3_p3}
                \caption{Nodal DG}
                \label{fig:NDG_IC3_p3_averages}
        \end{subfigure}
				~
        \begin{subfigure}[b]{0.31\textwidth}
								\centering
                \includegraphics[trim = 170mm 12mm 20mm 12mm,clip,width=\textwidth]{../DG&CPR/CPR/CPR_IC3_p3}
                \caption{CPR/FR}
                \label{fig:CPR_IC3_p3_averages}
        \end{subfigure}
				%
				\caption{Comparison of methods using Burgers' equation}
				\label{fig:compareBurgers_p3averages}
	\end{figure}
\end{frame}

\begin{frame} \frametitle{Burgers' equations}
The figures below depict the behavior of each methods at a time 1.5, for the fully developed discontinuity. Here we use 5th order base polynomials with 20 elements and CFL = 0.002.
	\begin{figure}
        \centering
        \begin{subfigure}[b]{0.31\textwidth}
                \centering
                \includegraphics[trim = 20mm 12mm 170mm 12mm,clip,width=\textwidth]{../DG&CPR/MDG/MDG_IC3_p5}
                \caption{Modal DG}
                \label{fig:MDG_IC3_p5}
        \end{subfigure}%
				~
				%~ %add desired spacing between images, e. g. ~, \quad, \qquad etc.
          %(or a blank line to force the subfigure onto a new line)
        \begin{subfigure}[b]{0.31\textwidth}
                \centering
                \includegraphics[trim = 20mm 12mm 170mm 12mm,clip,width=\textwidth]{../DG&CPR/NDG/NDG_IC3_p5}
                \caption{Nodal DG}
                \label{fig:NDG_IC3_p5}
        \end{subfigure}
				~
        \begin{subfigure}[b]{0.31\textwidth}
								\centering
                \includegraphics[trim = 20mm 12mm 170mm 12mm,clip,width=\textwidth]{../DG&CPR/CPR/CPR_IC3_p5}
                \caption{CPR/FR}
                \label{fig:CPR_IC3_p5}
        \end{subfigure}
				%
				\caption{Comparison of methods using Burgers' equation}
				\label{fig:compareBurgers_p5}
	\end{figure}
\end{frame}

\begin{frame} \frametitle{Burgers' equation}
	\begin{figure}
        \centering
        \begin{subfigure}[b]{0.31\textwidth}
                \centering
                \includegraphics[trim = 170mm 12mm 20mm 12mm,clip,width=\textwidth]{../DG&CPR/MDG/MDG_IC3_p5}
                \caption{Modal DG}
                \label{fig:MDG_IC3_p5_averages}
        \end{subfigure}%
				~
				%~ %add desired spacing between images, e. g. ~, \quad, \qquad etc.
          %(or a blank line to force the subfigure onto a new line)
        \begin{subfigure}[b]{0.31\textwidth}
                \centering
                \includegraphics[trim = 170mm 12mm 20mm 12mm,clip,width=\textwidth]{../DG&CPR/NDG/NDG_IC3_p5}
                \caption{Nodal DG}
                \label{fig:NDG_IC3_p5_averages}
        \end{subfigure}
				~
        \begin{subfigure}[b]{0.31\textwidth}
								\centering
                \includegraphics[trim = 170mm 12mm 20mm 12mm,clip,width=\textwidth]{../DG&CPR/CPR/CPR_IC3_p5}
                \caption{CPR/FR}
                \label{fig:CPR_IC3_p5_averages}
        \end{subfigure}
				%
				\caption{Comparison of methods using Burgers' equation}
				\label{fig:compareBurgers_p5averages}
	\end{figure}
	Notice that Modal DG, was not able to converge to the solution at the specified time. We notice that even with smaller CFL number it was not possible to converge to the correct solution. However Nodal DG and CPR seems to perform well even with higher CFL conditions.
\end{frame}

\subsection{Linear system without limiter}

\begin{frame} \frametitle{Roe average for Euler system}
	Let us now a 1D Linear system of equations,
	\begin{figure}
		\centering
		\includegraphics[width=0.65\textwidth]{../DG&CPR/images/RoeEuler}
		\caption{Linear System}
		\label{fig:linearSystem}
	\end{figure}
\end{frame}

\subsection{BBGK without limiter}

\begin{frame} \frametitle{BBGK equation}
	Let us test now the BBGK for classical gas dynamics,
	\begin{figure}
		\centering
		\includegraphics[width=0.55\textwidth]{../DG&CPR/images/BBGK}
		\caption{Boltzmann-BGK PDF initial condition}
		\label{fig:BBGK_IC}
	\end{figure}
\end{frame}


