\begin{frame} \frametitle{Numerical Results}
	\begin{figure}
        \centering
        \begin{subfigure}[b]{0.45\textwidth}
                \centering
                \includegraphics[trim = 0mm 0mm 0mm 0mm,clip,width=\textwidth]{../FEM_grid/Mesh2d_pics/Q4}
                \caption{2D Quadrilateral Mesh}
                \label{fig:Q4}
        \end{subfigure}%
				~ %add desired spacing between images, e. g. ~, \quad, \qquad etc.
          %(or a blank line to force the subfigure onto a new line)
        \begin{subfigure}[b]{0.45\textwidth}
                \centering
                \includegraphics[trim = 0mm 0mm 0mm 0mm,clip,width=\textwidth]{../FEM_grid/Mesh2d_pics/T3}
                \caption{3D Triangular Mesh}
                \label{fig:T3}
				\end{subfigure}
				%\caption{Typical Application for the Classical Boltzmann Equation}
				\label{fig:myCFD_meshes}
	\end{figure}
\end{frame}

\begin{frame}
	\begin{figure}
        \centering
        \begin{subfigure}[b]{0.45\textwidth}
                \centering
                \includegraphics[trim = 0mm 0mm 0mm 0mm,clip,width=\textwidth]{../FEM_grid/Mesh2d_pics/AG}
                \caption{2D Mesh using Polar Coordinates}
                \label{fig:PolarCoordiantesMesh}
        \end{subfigure}%
				~ %add desired spacing between images, e. g. ~, \quad, \qquad etc.
          %(or a blank line to force the subfigure onto a new line)
        \begin{subfigure}[b]{0.45\textwidth}
                \centering
                \includegraphics[trim = 0mm 0mm 0mm 0mm,clip,width=\textwidth]{../FEM_grid/Mesh2d_pics/AG2}
                \caption{2D Mesh using Elliptical Coordiantes}
                \label{fig:EllipticalCoordiantesMesh}
				\end{subfigure}
				%\caption{Typical Application for the Classical Boltzmann Equation}
				\label{fig:AnalyticCoordiantes}
	\end{figure}
\end{frame}

\begin{frame}
	\begin{figure}
        \centering
        \begin{subfigure}[b]{0.45\textwidth}
                \centering
                \includegraphics[trim = 0mm 0mm 0mm 0mm,clip,width=\textwidth]{../FEM_grid/Mesh2d_pics/ForwardStep2}
                \caption{Mesh2D Result}
                \label{fig:forwardstep1}
        \end{subfigure}%
				~ %add desired spacing between images, e. g. ~, \quad, \qquad etc.
          %(or a blank line to force the subfigure onto a new line)
        \begin{subfigure}[b]{0.45\textwidth}
                \centering
                \includegraphics[trim = 0mm 0mm 0mm 0mm,clip,width=\textwidth]{../FEM_grid/Mesh2d_pics/ForwardStep}
                \caption{Built editor}
                \label{fig:forwardstep2}
				\end{subfigure}
				%\caption{Typical Application for the Classical Boltzmann Equation}
				\label{fig:NACA}
	\end{figure}
\end{frame}

\begin{frame}
	\begin{figure}
        \centering
        \begin{subfigure}[b]{0.45\textwidth}
                \centering
                \includegraphics[trim = 0mm 0mm 0mm 0mm,clip,width=\textwidth]{../FEM_grid/Mesh2d_pics/NACA2}
                \caption{DistMesh Result}
                \label{fig:naca1}
        \end{subfigure}%
				~ %add desired spacing between images, e. g. ~, \quad, \qquad etc.
          %(or a blank line to force the subfigure onto a new line)
        \begin{subfigure}[b]{0.45\textwidth}
                \centering
                \includegraphics[trim = 0mm 0mm 0mm 0mm,clip,width=\textwidth]{../FEM_grid/Mesh2d_pics/NACA}
                \caption{Built editor}
                \label{fig:naca2}
				\end{subfigure}
				%\caption{Typical Application for the Classical Boltzmann Equation}
				\label{fig:NACA}
	\end{figure}
\end{frame}

\begin{frame} \frametitle{Future Work}
	\begin{itemize}
		\item Implement BC node list and types
		\item Implement GCNS file reader and writer
		\item Implement Burgers 2D with periodic BC's
	\end{itemize}
\end{frame}